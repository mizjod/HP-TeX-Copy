\documentclass[10pt]{memoir}
\setstocksize{220mm}{155mm} 	        
\settrimmedsize{220mm}{155mm}{*}	
\settypeblocksize{170mm}{116mm}{*}	
\setlrmargins{18mm}{*}{*}
\setulmargins{*}{*}{1.2}
%\setlength{\headheight}{5pt}
\checkandfixthelayout[lines]
\linespread{1.1}

\setlength{\footmarkwidth}{1.3em}
\setlength{\footmarksep}{0em}
\setlength{\footparindent}{1.3em}
\footmarkstyle{\textsuperscript{#1} }
\usepackage{fnpos}
\makeFNbottom

\usepackage[teiexport=tidy,poetry=verse]{ekdosis}
\usepackage{sanskrit-poetry}
\usepackage{xcolor}
\usepackage[english]{babel}
\usepackage{babel-iast,xparse,xcolor,libertine}
\babelfont[iast]{rm}[Renderer=Harfbuzz, Scale=1.5]{AdishilaSan}
%\babelfont[english]{rm}[Scale=0.9]{Adobe Text Pro}
%\babelfont[english]{sf}{Linux Biolinum}
\babeltags{dev = iast}
\babeltags{eng = english}

\SetHooks{
  lemmastyle=\bfseries,
  refnumstyle=\selectlanguage{english}\color{blue}\bfseries, 
}
\newif\ifinapparatus
  \DeclareApparatus{default}[
  lang=english,
  sep = {] },
  delim=\hskip 0.75em,
  rule=none,
]
\DeclareApparatus{notes}[
  lang=english,
  sep = {},
  delim=\hskip 0.75em,
  rule=\rule{0.7in}{0.4pt},
]

\DeclareShorthand{conj}{\texteng{\emph{conj.}}}{ego}
\DeclareShorthand{emend}{\texteng{\emph{em.}}}{ego}
\setlength{\vrightskip}{-10pt}
%\setlength{\vgap}{3mm}
\verselinenumfont{\footnotesize\selectlanguage{english}\normalfont}

\NewDocumentCommand{\skp}{m}{}
%\NewDocumentCommand{\skm}{m}{\unless\ifinapparatus#1-\fi}
\NewDocumentCommand{\skm}{m}{\unless\ifinapparatus#1\fi} % modified by MD 2022-05-31

\setlength{\vrightskip}{-10pt}

%\DeclareShorthand{egoscr}{\selectlanguage{english}\emph{scripsi}}{ego}
%\DeclareShorthand{egomute}{\unskip}{ego}
%\DeclareShorthand{nosscr}{\selectlanguage{english}\emph{nos scribere}}{nos}

%\input{Manuscript-Sigla}
%\input{Abbreviations}
% changes/additions 2023-08-15 MD:
\TeXtoTEIPat{\lineom {#1}{#2}}{<note type="omission">#1 omitted in <ref>#2</ref></note>}
\TeXtoTEI{graus}{hi}[rend="grey"]
\TeXtoTEIPat{\startgray}{<note type="altrecension">}
\TeXtoTEIPat{\endgray}{</note>}


% additions/changes 2023-06-05 mm:
%\TeXtoTEIPat{\lineom {#1}}{<note type="omission">Line omitted in <ref>#1</ref></note>}
\TeXtoTEIPat{\NotIn {#1}}{<note type="omission">Stanza omitted in <ref>#1</ref></note>}

% additions 2023-04-16 MD:
\TeXtoTEIPat{\,}{ }

% additions 2023-04-13 mm:
\TeXtoTEIPat{\begin {versinnote}}{<lg>}
  \TeXtoTEIPat{\end {versinnote}}{</lg>}

% additions 2023-04-05 MD:
\TeXtoTEIPat{\begin {testimonia}[#1]}{<note type="testimonia" target="##1">}
  \TeXtoTEIPat{\end {testimonia}}{</note>}
\TeXtoTEI{devnote}{s}[xml:lang="sa-deva"]
																								
																								
													 
%\TeXtoTEIPat{\anm {#1}}{<note type="memo">(#1)</note>}
\TeXtoTEI{anm}{note}[type="memo"] %% change 2023-04-16 MD
%\TeXtoTEIPat{\Anm {#1}}{<note type="memo">(#1)</note>} % only in HP4-edition.tex
\TeXtoTEIPat{\Anm {#1}}{} %% change 2023-04-16 MD
\TeXtoTEIPat{\startverse}{} %%% marked for change 2023-04-13 mm
\TeXtoTEIPat{\endverse}{} %%% marked for change 2023-04-13 mm
\TeXtoTEIPat{\newpage}{}
\TeXtoTEIPat{\marma}{}
\TeXtoTEIPat{\marmas}{ }

%%% modify environments and commands
%%% TEI mapping
% additions/changes 2022-06-07 mm:
\TeXtoTEI{grau}{hi}[rend="grey"]
\TeXtoTEIPat{ \& }{ &amp; }

% additions/changes 2022-06-01 mm:
\TeXtoTEI{skp}{seg}[type="deva-ignore"]
\TeXtoTEI{skm}{seg}[type="ltn-ignore"]

\TeXtoTEIPat{\rlap {#1}}{#1}

% additions/changes 2022-04-06 mm:
\TeXtoTEI{sgwit}{ref}
\TeXtoTEI{textdev}{s}[xml:lang="sa-deva"]
\TeXtoTEIPat{\begin {col}[#1]}{<div type="colophon" xml:id="#1"><p>}
  \TeXtoTEIPat{\end {col}}{</p></div>}
\TeXtoTEIPat{\begin {ava}[#1]}{<note type="avataranika" target="##1">}
  \TeXtoTEIPat{\end {ava}}{</note>}

												   
\TeXtoTEIPat{\outdent}{}
\TeXtoTEIPat{\startaltrecension}{<note type="altrecension">}
\TeXtoTEIPat{\endaltrecension}{</note>}
\TeXtoTEIPat{\begin {alttlg}[#1]}{<lg xml:id="#1">}
  \TeXtoTEIPat{\end {alttlg}}{</lg>}
\TeXtoTEIPat{\\[+!]}{}

% additions/changes 2022-03-12 mm:
\TeXtoTEIPat{\begin {tlg}[#1]}{<lg xml:id="#1">}
  \TeXtoTEIPat{\end {tlg}}{</lg>}

\TeXtoTEIPat{\begin {translation}[#1]}{<note type="translation" target="##1">}
  \TeXtoTEIPat{\end {translation}}{</note>}
\TeXtoTEIPat{\begin {philcomm}[#1]}{<note type="philcomm" target="##1">}
  \TeXtoTEIPat{\end {philcomm}}{</note>}
\TeXtoTEIPat{\begin {sources}[#1]}{<note type="sources" target="##1">}
  \TeXtoTEIPat{\end {sources}}{</note>}


\TeXtoTEIPat{\begin {marma}[#1]}{<note type="marma" target="##1">}
  \TeXtoTEIPat{\end {marma}}{</note>}

\TeXtoTEIPat{\begin {jyotsna}[#1]}{<note type="jyotsna" target="##1">}
  \TeXtoTEIPat{\end {jyotsna}}{</note>}

\EnvtoTEI{description}{list}
\EnvtoTEI{itemize}{list}
\TeXtoTEIPat{\item [#1]}{<label>#1</label>\item}

\TeXtoTEI{tl}{l}
\TeXtoTEI{myfn}{note}[type="myfn"]
\TeXtoTEIPat{\getsiglum {#1}}{<ref target="##1">#1</ref>}

\TeXtoTEI{SetLineation}{}
\TeXtoTEI{noindent}{}
\TeXtoTEI{subsection*}{}

\TeXtoTEI{rlap}{}

% end additions/changes
% \TeXtoTEIPat{\skp {#1}}{#1}
% \TeXtoTEIPat{\skm {#1}}{}

\TeXtoTEIPat{\begin {prose}}{<p>}
  \TeXtoTEIPat{\end {prose}}{</p>}

\TeXtoTEIPat{\begin {tlate}}{<p>}
  \TeXtoTEIPat{\end {tlate}}{</p>}

\TeXtoTEI{emph}{hi}
\TeXtoTEI{bigskip}{}
% \TeXtoTEI{/}{|}
\TeXtoTEI{tl}{l}
\TeXtoTEIPat{english}{}
%\TeXtoTEIPat{-}{ } %% change 2023-04-16 MD
%\TeXtoTEIPat{°}{} %% change 2023-04-16 MD
\TeXtoTEIPat{\textcolor {#1}{#2}}{<hi rend="#1">#2</hi>}

% \TeXtoTEIPat{\eyeskip}{}
% \TeXtoTEIPat{\aberratio}{}
% \TeXtoTEIPat{\ad}{}
\TeXtoTEIPat{\add}{<hi rend="italic">add.</hi>} %% change 2023-04-16 MD
% \TeXtoTEIPat{\ann}{}
\TeXtoTEIPat{\ante}{<hi rend="italic">ante</hi> } %% change 2023-04-16 MD
\TeXtoTEIPat{\post}{<hi rend="italic">post</hi> } %% change 2023-04-16 MD
% \TeXtoTEIPat{\codd}{}
% \TeXtoTEIPat{\conj }{}
% \TeXtoTEIPat{\contin}{}
% \TeXtoTEIPat{\corr}{}
% \TeXtoTEIPat{\del}{}
% \TeXtoTEIPat{\dub}{}
% \TeXtoTEIPat{\emend }{}
% \TeXtoTEIPat{\expl}{}
% \TeXtoTEIPat{\ȩxplicat}{}
% \TeXtoTEIPat{\fol}{}
% \TeXtoTEIPat{\gloss}{}
% \TeXtoTEIPat{\ins}{}
% \TeXtoTEIPat{\im}{}
% \TeXtoTEIPat{\inmargine}{}
% \TeXtoTEIPat{\intextu}{}
% \TeXtoTEIPat{\indist}{}
% \TeXtoTEIPat{\iteravit}{}
% \TeXtoTEIPat{\lectio}{}
% \TeXtoTEIPat{\leginequit}{}
% \TeXtoTEIPat{\legn}{}
% \TeXtoTEIPat{\illeg}{<hi rend="italic">illeg.</hi>}
\TeXtoTEIPat{\illeg}{<gap reason="illeg."/>} %%% change 2023-04-11 mm
% \TeXtoTEIPat{\om}{<hi rend="italic">om.</hi>}
\TeXtoTEIPat{\om}{<gap reason="om."/>} %%% change 2023-04-11 mm
% \TeXtoTEIPat{\primman}{}
% \TeXtoTEIPat{\prob}{}
% \TeXtoTEIPat{\rep}{}
% \TeXtoTEIPat{\sequentia}{}
% \TeXtoTEIPat{\supralineam}{}
% \TeXtoTEIPat{\interlineam}{}
\TeXtoTEIPat{\vl}{<hi rend="italic">v.l.</hi>}
% \TeXtoTEIPat{\vide}{}
% \TeXtoTEIPat{\videtur}{}
% \TeXtoTEIPat{\crux}{}
% \TeXtoTEIPat{\cruxxx}{}
\TeXtoTEIPat{\unm}{<hi rend="italic">unm.</hi>}


% List of Scholars
\DeclareScholar{nos}{nos}[
forename=HPP,
surname=Team]


% Nullify \selectlanguage in TEI as it has been used in
% \DeclareWitness but should be ignored in TEI.
\TeXtoTEI{selectlanguage}{}


%%% Local Variables:
%%% mode: latex
%%% TeX-master: t
%%% End:


%%%%%%%%%%% below added by MD %%%%%%%%%%%

\NewDocumentEnvironment{ava}{O{}}{
  \begin{ekdverse}
    \hspace{-\vgap}}{
  \end{ekdverse}
  \vskip 0.6\baselineskip
}
\NewDocumentEnvironment{col}{O{}}{
  \medskip
  \setvnum{col}
%  \selectlanguage{iast}
  \begin{ekdverse}
    \hspace{-\vgap}}{
  \end{ekdverse}
}

        
% modifications/additions by MM 2022-06-07:
\NewDocumentEnvironment{altava}{O{}}{
  \begin{ekdverse}
    \hspace{-\vgap}}{
  \end{ekdverse}
  \vskip 0.6\baselineskip
}   

% end additions

\SetTEIxmlExport{autopar=false}

\NewDocumentEnvironment{tlg}{O{}}{
  \begin{ekdverse}}{
  \end{ekdverse}
  \vskip 0.6\baselineskip}

% additions/changes 2022-08-22 mm:
\NewDocumentEnvironment{alttlg}{O{}}{
  %\stopvline
  %\addtocounter{saved@poemline}{-1}
  \setvnum{\hindsection.\arabic{saved@poemline}*\arabic{poemline}}
%  \selectlanguage{iast}
  \begin{ekdverse}[type=altrecension]
    \color{gray}
  }{
  \end{ekdverse}
  \vskip 0.6\baselineskip
  %\addtocounter{saved@poemline}{1}
  %\startvline
  %\setvnum{\hindsection.\arabic{poemline}}
  %\selectlanguage{iast}
}

\NewDocumentCommand{\tl}{m}{#1}

\def\startverse{\begin{ekdverse}}
\def\endverse{\end{ekdverse}\vskip 0.75\baselineskip}
\def\startgray{\color{gray}} % NEW! 2023-06-16
\def\endgray{\color{black}} % NEW! 2023-06-16

\def\startaltrecension{
	\stopvline
	\addtocounter{saved@poemline}{-1}
	\setvnum{\hindsection.\arabic{saved@poemline}*\arabic{poemline}}
	\begin{ekdverse}[type=altrecension]
%	\indentpattern{1111}
%	\begin{patverse}
	\color{gray}
	\small % added 2023-07-02 MD 
	}
\def\endaltrecension{
%	\end{patverse}
	\end{ekdverse}
	\vskip 0.75\baselineskip
    \addtocounter{saved@poemline}{1}
	\startvline   
	\setvnum{\hindsection.\arabic{poemline}}
	\normalsize % added 2023-07-02 MD 
	}

%%%%%%

\newcommand{\myfn}[1]{\footnote{\texteng{#1}}}
\renewcommand{\thefootnote}{\texteng{\arabic{footnote}}}
\newcommand{\nagari}[1]{\textdev{\scriptsize #1}}
%\newcommand{\devnote}[1]{\selectlanguage{iast}{\scriptsize #1}\selectlanguage{english}}
\newcommand{\devnote}[1]{\emph{#1}}
\newcommand{\outdent}{\hspace{-\vgap}}
\newcommand{\sgwit}[1]{{\footnotesize (\getsiglum{#1})}}
\newcommand{\NotIn}[1]{\texteng{\footnotesize (om. \getsiglum{#1})}}
\newcommand{\lineom}[2]{\texteng{\footnotesize (#1 om. \getsiglum{#2})}}
\newcommand{\grau}[1]{\textcolor{gray}{#1}}
\newcommand{\graus}[1]{\small\grau{#1}\normalsize}
\newcommand{\Anm}[1]{\texteng{\footnotesize (#1)}}
\newcommand{\anm}[1]{\texteng{\footnotesize [#1]}}

\def\om{\texteng{\emph{om.\kern-1pt}}}
\def\illeg{\texteng{\emph{illeg.\kern-1pt}}} 
\def\damaged{\texteng{\emph{damaged}}} 
\def\unm{\texteng{\emph{unm.\ }}}
\def\gap{\texteng{\emph{gap}}}
%\def\recte{\texteng{r.\:}}
%\def\for{\texteng{for }}
%\def\sic{\texteng{\emph{sic}}}
%\def\oder{\texteng{\emph{or\ }}}
\def\ante{\texteng{\normalfont\emph{ante\ }}}
\def\add{\texteng{\normalfont\emph{add.}}}
\def\post{\texteng{\normalfont\emph{post\ }}}
\def\antecorr{\texteng{\textsubscript{ac}}}% noch nicht in Preamble-Print.tex
\def\postcorr{\texteng{\textsubscript{pc}}}% noch nicht in Preamble-Print.tex
\def\marma{\textsuperscript{\#}}
\def\marmas{\textsuperscript{\#} }
\def\crux{\textsuperscript{\textdagger}}

%%% Gr1,4b,6
\usepackage{textgreek}
\DeclareWitness{N3}{\texteng{\textalpha\textsubscript{1}}}{NGMPP B 62-20}[]
\DeclareWitness{J5}{\texteng{\textalpha\textsubscript{2}}}{Jodhpur 02235}[]
\DeclareWitness{G4}{\texteng{\textalpha\textsubscript{3}}}{GOML 18885}[]% Telugu script
\DeclareWitness{N24}{\texteng{\textalpha\textsubscript{4}}}{NGMPP G 190-16}[]
\DeclareWitness{Gr1r}{\texteng{\textalpha{*}}}{Gr1 reconstructed}[]

\DeclareWitness{C6}{\texteng{\textbeta\textsubscript{1}}}{Lalchand M-2089}[]
\DeclareWitness{P11}{\texteng{\textbeta\textsubscript{2}}}{}[]

\DeclareWitness{V3}{\texteng{\textbeta\textsubscript{\textomega}}}{Sampurnananda Library Sarasvati Bhavan 29899}[]

%%% Gr2

\DeclareWitness{N23}{\texteng{\textgamma\textsubscript{1}}}{NGMPP G 25-2}[]
        \DeclareHand{N23ac}{N23}{\texteng{\textgamma\rlap{\textsubscript{1}}\textsuperscript{ac}}}[]
        \DeclareHand{N23pc}{N23}{\texteng{\textgamma\rlap{\textsubscript{1}}\textsuperscript{pc}}}[]
\DeclareWitness{J7}{\texteng{\textgamma\textsubscript{2}}}{Jodhpur 02241}[]
%\DeclareWitness{V6}{\texteng{V\textsubscript{6}}}{Sampurnananda Library Sarasvati Bhavan 29991}[]
\DeclareWitness{K1}{\texteng{K\textsubscript{1}}}{Raghunātha Temple Library 4383}[settlement=Jammu]
        \DeclareWitness{K1ac}{\texteng{K\rlap{\textsubscript{1}}\textsuperscript{ac}\space}}{}[]
        \DeclareWitness{K1pc}{\texteng{K\rlap{\textsubscript{1}}\textsuperscript{pc}\space}}{}[]


%%% Gr3

\DeclareWitness{V19}{\texteng{\textdelta\textsubscript{1}}}{Sampurnananda Library Sarasvati Bhavan 30069}[]
\DeclareWitness{K3}{\texteng{\textdelta\textsubscript{2}}}{Privat collection}
\DeclareWitness{C7}{\texteng{\textdelta\textsubscript{3}}}{Lalchand M-6494}[]
%\DeclareWitness{C1}{\texteng{C\textsubscript{1}}}{Lalchand M-2080}[]%L1 And C1 very close (and come from same region)
%\DeclareWitness{P23}{\texteng{P\textsubscript{23}}}{}[]
%\DeclareWitness{L1}{\texteng{L\textsubscript{1}}}{SOAS RE 43454}[settlement=Jammu]

\DeclareWitness{J6}{\texteng{\textdelta\textsubscript{\textomega}}}{Jodhpur 02237}[]

%%% Gr4c

\DeclareWitness{P15}{\texteng{\textepsilon\textsubscript{1}}}{}[]
\DeclareWitness{N19}{\texteng{\textepsilon\textsubscript{2}}}{NGMPP E-1528-1 / E-1527-7(4)}[]
\DeclareWitness{V15}{\texteng{\textepsilon\textsubscript{3}}}{Sampurnananda Library Sarasvati Bhavan 30051}[]
        \DeclareHand{V15ac}{V15}{\texteng{\textepsilon\rlap{\textsubscript{3}}\textsuperscript{ac}}}[]
        \DeclareHand{V15pc}{V15}{\texteng{\textepsilon\rlap{\textsubscript{3}}\textsuperscript{pc}}}[]
\DeclareWitness{J14}{\texteng{\textepsilon\textsubscript{4}}}{Jodhpur 02239}[]
\DeclareWitness{J11}{\texteng{\textepsilon\textsubscript{5}}}{Jodhpur 23532}[]
        \DeclareHand{J11ac}{J11}{\texteng{\textepsilon\rlap{\textsubscript{4}}\textsuperscript{i.t.}}}[]
        \DeclareHand{J11pc}{J11}{\texteng{\textepsilon\rlap{\textsubscript{4}}\textsuperscript{mg.}}}[alternative reading written by the first hand in margin or interlinearly (J11)] 
%\DeclareWitness{L2}{\texteng{L\textsubscript{2}}}{Wellcome Collection O.36]}
\DeclareWitness{M1}{\texteng{M\textsubscript{1}}}{P-5682/4}[]

\DeclareWitness{V1}{\texteng{\texteta\textsubscript{1}}}{Sampurnananda Library Sarasvati Bhavan 30109}[]
        \DeclareHand{V1ac}{V1}{\texteng{\texteta\rlap{\textsubscript{1}}\textsuperscript{ac}}}[]
        \DeclareHand{V1pc}{V1}{\texteng{\texteta\rlap{\textsubscript{1}}\textsuperscript{pc}}}[]

%%% Gr4d

\DeclareWitness{J10}{\texteng{\texteta\textsubscript{2}}}{MSPP Jodhpur 2230}[]
        \DeclareHand{J10ac}{J10}{\texteng{\texteta\rlap{\textsubscript{2}}\textsuperscript{pc}}}[]
        \DeclareHand{J10pc}{J10}{\texteng{\texteta\rlap{\textsubscript{2}}\textsuperscript{pc}}}[]

\DeclareWitness{N9}{\texteng{\texteta\textsubscript{\textomega 1}}}{NGMPP A62-33}[]
\DeclareWitness{V17}{\texteng{\texteta\textsubscript{\textomega 2}}}{Sampurnananda Library Sarasvati Bhavan 30053}[]
\DeclareWitness{N26}{\texteng{\texteta\textsubscript{\textomega 3}}}{NGMPP}[]
%\DeclareWitness{J15}{\texteng{\textepsilon\textsubscript{\textomega 4}}}{Jodhpur 9732A}[]

%%%

\DeclareWitness{Jyo}{\texteng{\textchi}}{Brahmānanda's version}[]
%\DeclareWitness{Tue}{\texteng{Tü}}{Ma I 339}[]

\DeclareWitness{ceteri}{\texteng{cett.}}{ceteri}[]

%%% Group Sigla

% \DeclareShorthand{Gr1}{\selectlanguage{english}Gr\textsubscript{1}}{N3,J5,G4}
%\DeclareShorthand{GrA1}{\texteng{A}}{N3,C6,V3}
%\DeclareShorthand{GrA2}{\texteng{A}}{N3,P11,C6,V3}

\DeclareShorthand{Gr2}{\texteng{\textGamma}}{N23,J7}
%\DeclareShorthand{Gr2}{\texteng{%
%	\textbeta\textsubscript{1}%
%	\textbeta\textsubscript{2}%
%	}}{N23,J7}
\DeclareShorthand{Gr3a}{\texteng{\textDelta}}{V19,K3,C7}
\DeclareShorthand{Gr4b}{\texteng{%
	\textbeta\textsubscript{1}%
	\textbeta\textsubscript{2}%
	}}{C6,P11}
\DeclareShorthand{GrB}{\textBeta}{C6,P11,V3}
\DeclareShorthand{Gr4c}{\textEpsilon}{P15,N19,V15}

\DeclareShorthand{Gr4d}{\texteng{%
	\texteta\textsubscript{1}%
	\texteta\textsubscript{2}%
	}}{V1,J10}
\DeclareShorthand{Gr6}{\texteng{\textOmega}}{V3,J6,N9,V17}

\makepagestyle{HPed}
\makeoddhead{HPed}{\small\texteng{HP3 edition}}{}{\small\texteng{\today}}
\makeevenhead{HPed}{\small\texteng{HP3 edition}}{}{\small\texteng{\today}}
\makeoddfoot{HPed}{}{\small\texteng{\thepage}}{}
\makeevenfoot{HPed}{}{\small\texteng{\thepage}}{}
\def\hindsection{3}

% N3,(P11,)C6,V3, N23,J7, V19,K3,C7(except Vajrolī), P15(up to 13a),N19,V15, V1,J10,Jyo
% J6,V17,N9 (for the Khecaryabhyāsakrama only)
% not included: C1(lost),P23,P11(errorneus),N26
% discarded: C6,C8,N17,J11,J15


\begin{document}
\pagestyle{HPed} %

%\begin{otherlanguage}{iast}
\begin{ekdosis}
\begin{ekdverse}

%[hp03_001]
\pada{\app{\lem[wit={ceteri}]{saśaila}
		\rdg[wit={V3}]{saśaile}
		\rdg[wit={K3},alt={\om}]{\skp{\om}}}% lost K3
	\app{\lem[wit={ceteri}]{vana}
		\rdg[wit={N23}]{vane}
		\rdg[wit={K3},alt={\om}]{\skp{\om}}}% lost K3
	\app{\lem[wit={ceteri}]{dhātrīṇāṃ}
		\rdg[wit={C6}]{dhātṝṇāṃ}
		\rdg[wit={K3},alt={\om}]{\skp{\om}}}} % lost K3
\pada{\app{\lem[wit={ceteri}]{yathādhāro}
		\rdg[wit={K3},alt={\om}]{\skp{\om}}} % lost K3
	\app{\lem[wit={ceteri}]{'hināyakaḥ}
		\rdg[wit={J7}]{himālayaḥ}
		\rdg[wit={K3},alt={\om}]{\skp{\om}}}/} \\+
\pada{sarveṣāṃ  \app{\lem[wit={ceteri}]{yoga}
		\rdg[wit={C6,V3}]{haṭha}
		\rdg[wit={K3},alt={\om}]{\skp{\om}}}%
	\app{\lem[wit={ceteri},postwit={(\getsiglum{K3}\,\textsubscript{s.l.})}]{tantrāṇāṃ}
		\rdg[wit={K3},postwit={\textsubscript{i.t.}}]{śāstrāṇā}}}
\pada{tathādhāro hi kuṇḍalī//}\\! %3.1

%[hp03_002]
\pada{suptā guruprasādena} % prasānena N23
\pada{\app{\lem[wit={N3,P11,V3,V15,J10,Jyo}]{yadā jāgarti kuṇḍalī}
		\rdg[wit={C6,P15,N19,V1}]{yathā jāgarti kuṇḍalī}
		\rdg[wit={Gr2,Gr3a}]{bodhitā sukhadā bhavet}}/}\\+
\pada{\app{\lem[wit={N3,C6,V3,Gr2,V15,Jyo}]{tadā}% +J17
		\rdg[wit={Gr3a,P15,N19,V1,J10}]{tathā}}
	\app{\lem[wit={ceteri}]{sarvāṇi padmāni}
		\rdg[wit={J10}]{padmāni sarvāṇi}
		\rdg[wit={V19}]{pi sarvapadmāni}}}
\pada{bhidyante granthayo'pi ca//}\\!  %3.2 % bidyante K3; grathīyā N23, graṃthiyo N3

%[hp03_003]
\pada{\app{\lem[wit={ceteri}]{prāṇasya}
		\rdg[wit={K3,N19}]{praṇamya}
		\rdg[wit={C6}]{prāṇa}}
	śūnya% śūnyā J7, śunya N3
	\app{\lem[wit={ceteri}]{padavī}
		\rdg[wit={K3,N19,V15,V1}]{padavīṃ}}}\myfn{\getsiglum{Gr4c} jump to \devnote{śūnyapadavī} in the next verse.} % padavīṃ V1?
\pada{\app{\lem[wit={N3,V3,Gr2,J10,Jyo}]{tadā}
		\rdg[wit={C6,Gr3a}]{tathā}
		\rdg[wit={V1}]{yathā}}
rāja\app{\lem[wit={ceteri}]{pathāyate}% raja° C6
		\rdg[wit={V1}]{padāyate}}/}\\+
\pada{\app{\lem[wit={N3,C6,V3,J7,J10,Jyo}]{tadā}% +C6pc
		\rdg[wit={V19,C7,V1}]{tathā} % +C6ac; tathac V19
		\rdg[wit={K3}]{yathā}
		\rdg[wit={N23}]{yadā}}
	cittaṃ nirālambaṃ} % °lambhaṃ K3
\pada{\app{\lem[wit={N3,C6,V3,Gr2,J10,Jyo}]{tadā}
		\rdg[wit={Gr3a,V1}]{tathā}} 
		kālasya vañcanam//}
		\lineom{bcd}{Gr4c} \anm{eye-skip}\\!  %3.3

\outdent
\app{\lem[wit={N3,C6,Gr2,K3,C7}]{śūnyapadavīti kim} % +J17
		\rdg[wit={J10}]{atha śūnyapadavīm iti kim ucyate}}/
	\NotIn{V3,V19,Gr4c,V1,Jyo}
%	\sgwit{N3,C6,Gr2,K3,C7,J10}

%[hp03_004]
\pada{suṣumṇā śūnyapadavī} % pavī K3, padavīva N3; sukhu° N23
\pada{brahma\app{\lem[wit={N23,V1}]{randhra}
		\rdg[wit={ceteri}]{randhraṃ}}%
	mahā\app{\lem[wit={P15,V1,J10,Jyo}]{pathaḥ}
		\rdg[wit={J5,C6,V3,Gr2,Gr3a,N19}]{pathaṃ}
		\rdg[wit={N3,V15}]{pathāḥ}}/}\\+
\pada{\app{\lem[wit={N3,C6,V3,Gr4c,V1,J10,Jyo}]{śmaśānaṃ}
		\rdg[wit={V19}]{śmaśāne} % C8ac? °naṃ pc?
		\rdg[wit={J7,K3,C7}]{śmaśānī}% #
		\rdg[wit={N23}]{aiśānī}} śāmbhavī
	\app{\lem[wit={N3,Gr2,V19,C7,P15,V15,Jyo}]{madhya}
		\rdg[wit={C6,V3,V1,J10}]{madhyaṃ}
		\rdg[wit={N19}]{madhye}
		\rdg[wit={K3}]{mudrā}}}%
\pada{\app{\lem[wit={N3,C6,V3,Gr2,V19,P15,V15,Jyo},alt={mārgaś cety eka}]{mārgaś cety eka} % māgraśceteka P15, māga° V15; catye°? V19,
		\rdg[wit={V1}]{mārgeś cety eka}
		\rdg[wit={N19}]{mārgapratyeka}
		\rdg[wit={K3,C7}]{mārgaḥ śūnyeva}}%
	\app{\lem[wit={J7,J10,Jyo}]{vācakāḥ}
		\rdg[wit={N23}]{vācakā}
		\rdg[wit={N3,C6}]{vācakaḥ}
		\rdg[wit={V3}]{vācaka}
		\rdg[wit={J5,Gr3a,Gr4c,V1}]{vācakam}}//}
		\lineom{a}{Gr4c}\\!  %3.4

%[hp03_005]
\pada{tasmāt sarvaprayatnena}
\pada{\app{\lem[wit={N3,Gr2,P15,V15,J10,Jyo},alt={prabodhayitum}]{prabodhayitu} % °yītum N3
		\rdg[wit={C6,V3,N19,V1}]{prabodhayatum}
%		\rdg[wit={J8}]{prabodhayatim}
		\rdg[wit={Gr3a}]{tāṃ bodhayituṃ}}%
	\app{\lem[wit={ceteri},alt={īśvarīṃ}]{m īśvarīm}
		\rdg[wit={V3,N23,N19}]{īśvarī}% °rīṃ J8
		\rdg[wit={V19}]{īśvaraṃ}}/}\\+
\pada{brahma\app{\lem[wit={ceteri}]{dvāra}
		\rdg[wit={P15,N19}]{dvāre}}%
	\app{\lem[wit={ceteri}]{mukhe}
		\rdg[wit={N23}]{mukha}
		\rdg[wit={P15}]{sukhe}}
	\app{\lem[wit={ceteri}]{suptāṃ}
		\rdg[wit={V3}]{supto}}}
\pada{mudrā\app{\lem[wit={N3,P11,V3,Gr4c,V1,J10,Jyo}]{bhyāsaṃ samācaret}% V3 bhyāsa; +P11
		\rdg[wit={C6,Gr2,Gr3a}]{bhyāsena bodhayet}}//}\\!  %3.5


\newpage
%[hp03_006]
\pada{mahāmudrā mahābandho} % mudrāṃ N3
\pada{mahāvedhaś ca khecarī/}\\+% °vedhoś N19
\pada{\app{\lem[wit={N23,K3,C7,J10}]{uḍḍiyānaṃ}
		\rdg[wit={C6,V3}]{uḍiyānaṃ}% +J17
		\rdg[wit={N3,J7,N19}]{uḍḍīyāṇaṃ}
		\rdg[wit={V1}]{uḍḍīyāṇo}
		\rdg[wit={Jyo}]{uḍyānaṃ}
		\rdg[wit={P15,V15}]{uḍyāṇa}
		\rdg[wit={V19},alt={\om}]{\skp{\om}}}
  mūla\app{\lem[wit={Gr2,K3,C7},alt={mūlabandho}]{bandho}% muḷa° V15 % ra-vipulā
		\rdg[wit={C6}]{°bandhas}
		\rdg[wit={J10}]{°bandhaḥ}
		\rdg[wit={N3,J5}]{°bandhaṃ}
		\rdg[wit={V3}]{°bandha}
		\rdg[wit={N19,V15,Jyo}]{°bandhaś ca}
		\rdg[wit={V1}]{°bandhāś ca}
		\rdg[wit={V19},alt={\om}]{\skp{\om}}}}
\pada{\app{\lem[wit={J7,Gr3a,N19,V1,Jyo}]{bandho}% + 3 testimonies
		\rdg[wit={N23}]{bandhā}
		\rdg[wit={N3,C6,V3,P15,V15,J10}]{tato}} % ##
	\app{\lem[wit={ceteri}]{jālandharā}
		\rdg[wit={N23}]{jāladharā}
		\rdg[wit={V1}]{jālaṃjarā}}bhidhaḥ//}
		\lineom{bc}{V19}\\!  %3.6


%[hp03_007]
\pada{karaṇī  % karanī N23
	\app{\lem[wit={ceteri}]{viparītākhyā}
		\rdg[wit={N19}]{viparītākhyaṃ}
		\rdg[wit={P15}]{viparītāni}}}
\pada{\app{\lem[wit={ceteri}]{vajrolī} % varjrolī N23, vajroli V15, varjālī P15, vajrālī N3, vajolī N19
		\rdg[wit={V19}]{vajro}} śakticālanam/} \\+
\pada{\app{\lem[wit={N3,J5}]{idaṃ mudrādi}
		\rdg[wit={P15,N19}]{idaṃ tu mudrā}
		\rdg[wit={V3,V15,V1,J10}]{idaṃ ca mudrā}
		\rdg[wit={Jyo}]{idaṃ hi mudrā}
		\rdg[wit={C6,Gr2,Gr3a}]{etad dhi mudrā}% etadvi? N23
		}daśakaṃ}
\pada{jarā\app{\lem[wit={ceteri}]{maraṇa}
		\rdg[wit={V3}]{marṇavi}
		\rdg[wit={N23}]{maṇa}}%
	\app{\lem[wit={ceteri}]{nāśanam}
		\rdg[wit={Gr3a}]{varjitaṃ}}//}\\!  %3.7 % \lineom{cd}{P11}!!


%[hp03_008]
\pada{\app{\lem[wit={ceteri}]{ādinātho}
		\rdg[wit={V19,C7}]{ādīśvaro}
		\rdg[wit={K3}]{ādyeśvaro}}ditaṃ
	\app{\lem[wit={ceteri},alt={divyam}]{divya}
		\rdg[wit={J10}]{sarvaṃ}}m}
\pada{aṣṭaiśvarya\app{\lem[wit={ceteri}]{pradāyakam}% +P11
	\rdg[wit={C6}]{phalapradaṃ}}/}% aṣṭaiḥ N3
	\myfn{In \getsiglum{V15} this hemistich is found after pādas ab of the next verse.} \\+
\pada{vallabhaṃ sarva% sarve J17
	\app{\lem[wit={ceteri}]{siddhānāṃ}% +J5
		\rdg[wit={N3,V3}]{siddhīnāṃ}
		\rdg[wit={V15}]{vidyānāṃ}}}
\pada{durlabhaṃ marutām api//}\\!  %3.8

%[hp03_009]
\pada{gopanīyaṃ prayatnena} \pada{yathā ratnakaraṇḍakam/} \\+
\pada{kasyacin naiva % kasyacci V3
	\app{\lem[wit={ceteri}]{vaktavyaṃ}
		\rdg[wit={V1}]{vaktavyā}
		\rdg[wit={V3,P15}]{kartavyaṃ}}}
\pada{\app{\lem[wit={ceteri}]{kulastrīsurataṃ}
		\rdg[wit={V1}]{kulastrīṣu rataṃ}
		\rdg[wit={V3},post={\unm}]{kulastrīsukharataṃ}}
	\app{\lem[wit={ceteri}]{yathā}
		\rdg[wit={J10}]{tathā}}//}\\!  %3.9

\endverse
\startaltrecension{}
%[hp03_009_1]
\pada{vajrolī
	\app{\lem[wit={C6}]{tv amarolī} % §§
		\rdg[wit={V3}]{amarolīś}
		\rdg[wit={J10}]{°r amaroliś}} ca}
\pada{\app{\lem[wit={C6,V3}]{sahajolī}
		\rdg[wit={J10}]{sahajolis}} tridhā
	\app{\lem[wit={C6}]{matāḥ}
		\rdg[wit={J10}]{mataḥ}
		\rdg[wit={V3}]{magaḥ}}/} \\+
\pada{\app{\lem[wit={V3}]{eteṣāṃ}% +J17
		\rdg[wit={C6}]{etāsāṃ} % ##?
		\rdg[wit={J10}]{eteṣā}} lakṣaṇaṃ
	\app{\lem[wit={J10}]{vakṣye}
		\rdg[wit={V3}]{vakṣe}}}
\pada{kartavyaṃ ca viśeṣataḥ//} \sgwit{C6,V3,J10}\\!  % 3.9*1  NOT IN P11!
\endaltrecension
\startverse



%%%%%%%%%%%%%%
%\newpage
\outdent
	\app{\lem[wit={V3,N23,N19,Jyo}]{atha mahāmudrā}
		\rdg[wit={C6}]{tatha mahāmudrā}
		\rdg[wit={P11,V1,J10}]{tatra mahāmudrā}% ##
		\rdg[wit={P15}]{tatra mahāmudrā yathā}
		\rdg[wit={V15}]{atha tatra mahāmudrā}
		\rdg[wit={N3,J7,Gr3a},alt={\om}]{\skp{\om}}}/

%[hp03_010]
\pada{pādamūlena vāmena} % °mūle<<na>> C8
\pada{\app{\lem[wit={ceteri}]{yoniṃ} %%% CHECK! MD
		\rdg[wit={N3,V3}]{yoni}% yoniṃ J8
		\rdg[wit={N19}]{yoniḥ}}
	\app{\lem[wit={ceteri}]{saṃpīḍya}
		\rdg[wit={P15,N19}]{pīḍya}} dakṣiṇam/}\\+ % °ṇāṃ N3
\pada{\app{\lem[wit={ceteri}]{pādaṃ}
		\rdg[wit={J10}]{pāda}
		\rdg[wit={V3}]{padaṃ}
		\rdg[wit={Jyo}]{prasā°}}
	\app{\lem[wit={ceteri}]{prasāritaṃ}
		\rdg[wit={V3}]{prasaritaṃ}
		\rdg[wit={V1}]{prasāditaṃ}
		\rdg[wit={Jyo}]{°ritaṃ padaṃ}}
	\app{\lem[wit={J7,K3,C7,V15,V1,J10}]{dhṛtvā}
		\rdg[wit={N3,C6,V3,N23,V19,P15,N19,Jyo}]{kṛtvā}}}
\pada{karābhyāṃ
	\app{\lem[wit={ceteri}]{pūrayen}% so in Amaraugha
		\rdg[wit={K3}]{pūrayet}
		\rdg[wit={J10}]{dhārayen}
		\rdg[wit={Jyo}]{dhārayed}}
	\app{\lem[wit={N3,P11,V3,N19}]{mukhe}% so in Amaraugha
		\rdg[wit={C6,Gr2,V19,C7,P15,V15,V1,J10}]{mukham}
		\rdg[wit={K3}]{sukham}
		\rdg[wit={Jyo}]{dṛḍhaṃ}}\marma//}\\!  %3.10

\newpage
%[hp03_011]
\pada{\app{\lem[wit={ceteri}]{kaṇṭhe}% +J8
		\rdg[wit={V3,Gr3a,N19}]{kaṇṭha}}
	\app{\lem[wit={J5,J7,V19,P15,V15,J10,Jyo}]{bandhaṃ}
		\rdg[wit={C6,V3,N23,V1}]{bandha}
		\rdg[wit={N19}]{bandhaḥ}
		\rdg[wit={K3}]{bandhe}
		\rdg[wit={N3}]{budha}
		\rdg[wit={C7}]{madhye}} samāropya}
\pada{\app{\lem[wit={ceteri},alt={dhārayed}]{dhāraye}
		\rdg[wit={V19}]{dhānayed}}d vāyum ūrdhvataḥ/}\\+ % °ta N3
\pada{\app{\lem[wit={ceteri}]{yathā}
		\rdg[wit={V1}]{pathi}} % prefer yathā ? And then read with the next line?
	\app{\lem[wit={N3,P11,V3,P15,V15,J10}]{daṇḍāhataḥ} % + J7pc; daṃḍohataḥ C8, °hata V3
		\rdg[wit={C6,Gr2,Gr3a,N19,V1,Jyo}]{daṇḍahataḥ}} sarpo}
\pada{\app{\lem[wit={ceteri}]{daṇḍākāraḥ} % kāraṃ J7ac, kāraḥ J7pc
		\rdg[wit={N19}]{daṇḍakāraḥ}}
	\app{\lem[wit={ceteri}]{prajāyate}
		\rdg[wit={V1}]{prayujyate}
		\rdg[wit={C7},postwit={(lost up to 3.19c saṃsthāpya; one folio missing)},alt={\om}]{\skp{\om}}}//}\\!  %3.11


%[hp03_012]
\pada{\app{\lem[wit={P11,J7,V19,K3,V15,V1,J10,Jyo}]{ṛjvībhūtā}
		\rdg[wit={C6}]{ṛjvībhūtvā}
		\rdg[wit={N3,V3}]{rujvībhūtvā}% °bhūtatathā śaktiḥ J5
		\rdg[wit={N19}]{rajvībhūtā}
		\rdg[wit={P15}]{vajrībhūtā}
		\rdg[wit={N23}]{ṛ\,\_\,bhūtrā}}
	\app{\lem[wit={ceteri}]{tathā}
		\rdg[wit={N19}]{yathā}}
	\app{\lem[wit={ceteri}]{śaktiḥ}
		\rdg[wit={V3,V19,N19,V1}]{śakti}}}
\pada{kuṇḍalī sahasā bhavet/}\\+ % kuṃḍali N23
\pada{\app{\lem[wit={N3,C6,V3,Gr2,P15,N19,V1}]{tadāsau}
		\rdg[wit={V19,K3}]{tathāsau}
		\rdg[wit={V15,J10,Jyo}]{tadā sā}} % tada N17
	\app{\lem[wit={ceteri}]{maraṇā}% +J17
		\rdg[wit={P15}]{maraṇa}
		\rdg[wit={V1}]{maraṇī}
		\rdg[wit={V3}]{ramaṇā}
		\rdg[wit={J10}]{maṇā}}%
	\app{\lem[wit={ceteri}]{vasthā}
		\rdg[wit={J7,V19,K3,V1}]{vasthāṃ}
		\rdg[wit={P15}]{sthā}}}
\pada{\app{\lem[wit={ceteri}]{jāyate}
		\rdg[wit={P15}]{yāyate}
		\rdg[wit={Gr2,V19,K3}]{harate}}
	\app{\lem[wit={N3,C6,V3,V19,K3,V1,J10,Jyo}]{dvipuṭā}
		\rdg[wit={N23}]{dvipūtā}
		\rdg[wit={P15,N19}]{nṛpuṭā}
		\rdg[wit={V15}]{tripuṭā}
		\rdg[wit={J7}]{vapurā}}%
	\app{\lem[wit={N3,P11,V3,N19,J10}]{śritā}
		\rdg[wit={J7}]{śrayāṃ}
		\rdg[wit={V19,K3,Jyo}]{śrayā}
		\rdg[wit={N23}]{śrayī}
		\rdg[wit={V1}]{ā[śr]i\,..}
		\rdg[wit={P15}]{smṛtā}
		\rdg[wit={V15}]{sanāṃ}
		\rdg[wit={C6}]{hi sā}}\marma//}\\!  %3.12

%\newpage
%[hp03_013]
\myfn{\getsiglum{V19,K3} have a different order for the following 4 verses: 16 \rightarrow\ 15 \rightarrow\ 13 \rightarrow\ 14.\\
\getsiglum{P15} is lost after \devnote{tataḥ śanaiḥ śanai}.}%
\pada{tataḥ
	\app{\lem[wit={ceteri}]{śanaiḥ śanair eva} % śanai(1) V3
		\rdg[wit={N23}]{śanaiḥ śanair yeca}}}
\pada{\app{\lem[wit={ceteri},alt={recayen}]{recaye}
		\rdg[wit={N19}]{recayan}}% °yana N19
	\app{\lem[wit={ceteri},alt={na tu}]{n na tu} % recayet tanu! K3
		\rdg[wit={V3}]{na ca}
		\rdg[wit={Jyo}]{naiva}} vegataḥ/}\\+
\pada{\app{\lem[wit={ceteri}]{iyaṃ}
		\rdg[wit={V3}]{idaṃ}} khalu mahāmudrā}
\pada{mahā\app{\lem[wit={ceteri}]{siddhaiḥ}% siddhai V19
		\rdg[wit={N19,V15}]{siddhiḥ}}
	\app{\lem[wit={J5,Jyo}]{pradarśitā}
		\rdg[wit={N3}]{pradarśanā}
		\rdg[wit={ceteri}]{praśasyate}% prasaśyate V3
		\rdg[wit={N19}]{prajāyate}}//}\\!  %3.13

%[hp03_014]
\pada{\app{\lem[wit={ceteri}]{mahā}
		\rdg[wit={J10}]{mahān}}%
	\app{\lem[wit={N3,C6,V3,V1,J10,Jyo}]{kleśādayo}
		\rdg[wit={J7}]{kleśā yato}
		\rdg[wit={N23}]{kleśa yato}
		\rdg[wit={P11}]{kleśāyatā}
		\rdg[wit={N19}]{kleśā yathā}
		\rdg[wit={V15}]{kleśa yathā}
		\rdg[wit={V19,K3}]{kleśā mahā}}\marmas
	\app{\lem[wit={ceteri}]{doṣā}
		\rdg[wit={J10,Jyo}]{doṣāḥ}
		\rdg[wit={J7}]{doṣa}}}
\pada{\app{\lem[wit={N3}]{hīyaṃte}
		\rdg[wit={J5}]{hrīyaṃte}
		\rdg[wit={V3,J10,Jyo}]{kṣīyante}
		\rdg[wit={C6,P11,Gr2,V19,K3,V15,V1}]{jīryante} % +SouthIndMss; jīryate K3, jīyaṃte V15; jaryante G4
		\rdg[wit={N19}]{jāyante}}
		maraṇādayaḥ/}\\+ % °ādaya N19,V3
\pada{mahā\app{\lem[wit={C6,V3,V15,J10,Jyo}]{mudrāṃ}
		\rdg[wit={V1}]{mudrā[ś]}
		\rdg[wit={N3,Gr2,N19}]{mudrā}
		\rdg[wit={V19,K3},postwit={(Pādas c--d omitted)},alt={\om}]{\skp{\om}}}
	\app{\lem[wit={ceteri}]{ca} % +J5
		\rdg[wit={N3}]{tu}} % +K3
	\app{\lem[wit={ceteri}]{tenaiva}
		\rdg[wit={N23}]{tenai} % haplo
		\rdg[wit={V15}]{tenetāṃ}
		\rdg[wit={V19,K3},alt={\om}]{\skp{\om}}}}
\pada{vadanti \app{\lem[wit={ceteri}]{vibudho}
		\rdg[wit={J7}]{vividho}
		\rdg[wit={V19,K3},alt={\om}]{\skp{\om}}}ttamāḥ//} \lineom{cd}{V19,K3}\\!  %3.14  % ttamaḥ N3, ttamā V3

%\newpage
%[hp03_015]
\pada{\app{\lem[wit={ceteri}]{candrāṅge} % °rṃge
		\rdg[wit={V1}]{cāndrāṅge}
		\rdg[wit={N19}]{candrāṃgaṃ}
		\rdg[wit={V19,K3}]{candrāṃśaṃ}}
	\app{\lem[wit={ceteri}]{tu}
		\rdg[wit={V3,J10}]{ca}} samabhyasya}
\pada{\app{\lem[wit={ceteri}]{sūryāṅge}
		\rdg[wit={V1}]{sūryāṅge°}
		\rdg[wit={N19}]{sūryāṃgaṃ}
		\rdg[wit={V19,K3}]{sūryāṃśaṃ}}
	\app{\lem[wit={N3,C6,V3,Gr2,J10,Jyo}]{punar abhyaset}
		\rdg[wit={V19,K3,N19,V15}]{tu samabhyaset}
		\rdg[wit={V1}]{°ṣu samabhyaset}}/}\\+
\pada{yāvat \app{\lem[wit={N3,C6,V3,Gr2,N19,V1,Jyo}]{tulyā} % tulyāṃ V1?
		\rdg[wit={J10}]{saṃkhyā}
		\rdg[wit={V19,K3}]{tayor}
		\rdg[wit={V15},alt={\om}]{\skp{\om}}}
	\app{\lem[wit={ceteri}]{bhavet}
		\rdg[wit={J7,V1}]{bhavat}
		\rdg[wit={V15},alt={\om}]{\skp{\om}}}
	\app{\lem[wit={N3,C6,V3,Gr2,V1,Jyo}]{saṃkhyā}
		\rdg[wit={N19}]{saṃkṣā}
		\rdg[wit={J10}]{tulyā}
		\rdg[wit={V19,K3}]{sāmyaṃ}
		\rdg[wit={V15},alt={\om}]{\skp{\om}}}}
\pada{tato mudrāṃ % mudrā C8
	\app{\lem[wit={ceteri}]{visarjayet}
		\rdg[wit={V19}]{visaryayet}
		\rdg[wit={V3}]{vivarjayet}
		\rdg[wit={V15},alt={\om}]{\skp{\om}}}//} \lineom{cd}{V15}\\!  %3.15



\newpage
%[hp03_016]
\pada{\app{\lem[wit={ceteri}]{na hi pathyam apathyaṃ vā}
		\rdg[wit={J10}]{nāpathyaṃ na hi pathyaṃ ca}
		\rdg[wit={N19},post={(3 akṣaras missing)}]{na hi madhyaṃ vā}}}
\pada{rasāḥ sarve'pi nīrasāḥ/}\\+ % rasā C6; nīrasā V3
\pada{\app{\lem[wit={ceteri}]{api bhuktaṃ} % muktaṃ N23?
		\rdg[wit={N19,V15}]{ahimuktaṃ}} viṣaṃ
	\app{\lem[wit={ceteri}]{ghoraṃ}
		\rdg[wit={V1}]{khāraṃ}}}
\pada{\app{\lem[wit={ceteri},alt={pīyūṣam}]{pīyūṣa}
		\rdg[wit={V3}]{piyuṣam}}m
	\app{\lem[wit={ceteri}]{iva}
		\rdg[wit={K3}]{api}}
	\app{\lem[wit={ceteri}]{jīryate}
		\rdg[wit={Gr2,Jyo}]{jīryati}
		\rdg[wit={V19}]{jīrjyate}}//}\\!

%[hp03_017] %\NotIn{P11}
\pada{kṣaya\app{\lem[wit={ceteri}]{kuṣṭha} % kṣayaṃ N19
		\rdg[wit={V1}]{kuṣṭhaṃ}}%
	\app{\lem[wit={ceteri}]{gudā}
		\rdg[wit={V19,N19,V15}]{mudā}}varta}%
\pada{\app{\lem[wit={ceteri}]{gulmājīrṇa}
		\rdg[wit={C6}]{gulmajīrṇa}
		\rdg[wit={Gr2}]{gulmaplīha}}%
	\app{\lem[wit={ceteri}]{purogamāḥ}
		\rdg[wit={V3}]{purogamā}
		\rdg[wit={V19,K3}]{jvarās tathā}}/}\\+
\pada{\app{\lem[wit={ceteri}]{tasya doṣāḥ} % doṣā V19,V3
		\rdg[wit={V1,J10}]{doṣāḥ sarve}}
		kṣayaṃ yānti} % jāṃti V19
\pada{mahāmudrāṃ \app{\lem[wit={ceteri}]{tu yo'bhyaset}
		\rdg[wit={K3}]{yo bhyaset}
		\rdg[wit={C6,V15}]{ca yo bhyaset}
		\rdg[wit={V3}]{yomabhyaset}}//}\\!  %3.17 


%[hp03_018] %\NotIn{P11}
\pada{\app{\lem[wit={ceteri}]{kathiteyaṃ} % °ya V15
		\rdg[wit={J5,V3,N19}]{kathitoyaṃ}} mahāmudrā}
\pada{\app{\lem[wit={G4,V3,N19,V15,V1,J10,Jyo},post={(nṛṇā \getsiglum{V15,V1})}]{mahāsiddhikarī nṛṇām}% +G4,N24,M3; keep!
		\rdg[wit={N3,C6,Gr2,V19,K3}]{jarāmṛtyuvināśinī} % +M1; °vināśanaṃ G7
		\rdg[wit={J5}]{nṛṇāṃ mṛtyuvināśinī}}\marma/}\\+
\pada{\app{\lem[wit={ceteri}]{gopanīyā}
		\rdg[wit={V3,N19}]{gopanīyaṃ}
		\rdg[wit={J10}]{gopanīyāṃ}} prayatnena}
\pada{na
	\app{\lem[wit={ceteri}]{deyā}
		\rdg[wit={V3}]{deyaṃ}}
	yasya kasyacit//}\\!   % yasyā J7ac %3.18 

%%%%%%%%%%%%%%
%\newpage
\outdent
\app{\lem[wit={ceteri}]{atha}
		\rdg[wit={Gr2,K3},alt={\om}]{\skp{\om}}}
	mahābandhaḥ/ % baṃdha N3,V3,N19

%[hp03_019]
\pada{\app{\lem[wit={N3,V19,K3,V15,V1,Jyo}]{pārṣṇiṃ}
%		\rdg[wit={J8}]{pārṣmiṃ}
		\rdg[wit={C6,V3,J7,N19,J10}]{pārṣṇi}
		\rdg[wit={N23}]{yāṣi}}
	\app{\lem[wit={ceteri}]{vāmasya}
		\rdg[wit={J10}]{bhāgena}} pādasya}
\pada{yonisthāne % yonī V15, yoniḥ N19
	\app{\lem[wit={ceteri}]{niyojayet}
		\rdg[wit={N19}]{yojayet}}/}\\+
\pada{vāmorūpari saṃsthāpya}
\pada{\app{\lem[wit={ceteri}]{dakṣiṇaṃ}
		\rdg[wit={V3}]{dakṣaṇaṃ}} caraṇaṃ
	\app{\lem[wit={ceteri}]{tathā}%
		\rdg[wit={C7}]{tataḥ}}//}\myfn{%
		In \getsiglum{N19,V15} this and the following two hemistiches are found after \ref{III22}ab. Probably they were omitted by eye-skip due to \devnote{niyojayet} and inserted at a wrong place.}\\! % V15 dhārayitvā; N19,J11 cālayitvā
		%3.19

%\newpage
%[hp03_020]
\pada{pūrayitvā
	\app{\lem[wit={N3,C6,Gr2,Gr3a}]{mukhe}
		\rdg[wit={V3,V1,J10,Jyo}]{tato}
		\rdg[wit={N19,V15}]{tathā}}
	\app{\lem[wit={ceteri}]{vāyuṃ}
		\rdg[wit={V3,Gr2}]{vāyu}}}
\pada{hṛdaye \app{\lem[wit={ceteri}]{cibukaṃ}% V19 corr. from cibuke tathā dṛḍhaṃ
		\rdg[wit={V15}]{sasvanaṃ}
		\rdg[wit={N19}]{svasanaṃ}}
	\app{\lem[wit={ceteri}]{dṛḍhaṃ}
		\rdg[wit={C6}]{tathā}}/}\\+ % V19 had both tathā (cancelled?) and dṛḍhaṃ
\pada{\app{\lem[wit={N3,V3}]{nibhṛtya} % = Amaraugha
		\rdg[wit={C6}]{nibhṛtaṃ}
		\rdg[wit={N19,V15}]{nivṛtya}
		\rdg[wit={Gr2,V19,C7,V1,Jyo}]{niṣpīḍya} % niḥ° Gr2
		\rdg[wit={K3}]{nipīḍya}
		\rdg[wit={J10}]{nikṣipya}} yonim ākuñcya}
\pada{mano madhye niyojayet//}\\+  % nijojayet N23
\pada{\app{\lem[wit={N3,J5,Gr3a},post={(tu for ca \getsiglum{K3})}]{recayec ca śanair eva}
	\rdg[wit={C6,Gr2,V1,Jyo}]{dhārayitvā yathāśakti}
	\rdg[wit={P11,V3,V15,J10}]{dhārayitvā yathāśaktyā}
	\rdg[wit={N19}]{cālayitvā yathāśaktyā}}
\pada{\app{\lem[wit={N3,J5,Gr3a}]{mahābandho'yam ucyate}
	\rdg[wit={C6,P11,V3,Gr2,N19,V15,V1,J10,Jyo}]{recayed anilaṃ śanaiḥ}}//}\label{mahabandha}\\! % śanai V3 \sgwit{N3,J5,Gr3a}

\newpage

%\newpage
%[hp03_021]
\startgray
\pada{\app{\lem[wit={ceteri}]{matam atra}
		\rdg[wit={V1}]{matam etat}
		\rdg[wit={V3}]{matāntare}
		\rdg[wit={J10}]{matārettamaṃtra}}
	\app{\lem[wit={ceteri}]{tu}
		\rdg[wit={Gr2}]{ca}} keṣāṃcit} % V3 ci
\pada{\app{\lem[wit={ceteri}]{kaṇṭha}
		\rdg[wit={J10}]{kaṇṭhe}}%
	\app{\lem[wit={ceteri}]{bandhaṃ}
		\rdg[wit={V3}]{bandha}
		\rdg[wit={N23}]{yaṃ}}
	\app{\lem[wit={P11,V3,N19,V15}]{visarjayet}
		\rdg[wit={V1,J10,Jyo}]{vivarjayet}
		\rdg[wit={C6,Gr2}]{tu varjayet}}/}
		\label{III22}\\+
\pada{\app{\lem[wit={C6}]{rājadantabilaṃ tatra}
		\rdg[wit={V3}]{rājadantabilaṃ jatra}
		\rdg[wit={N19,V15}]{rājadantabalaṃ haṃti}% rājaddaṃntabaḷaṃ V15
		\rdg[wit={Gr2}]{rājadantadvayaṃ tatra}
		\rdg[wit={V1,Jyo}]{rājadantasthajihvāyā(ṃ)}% ṃ om. V1
		\rdg[wit={J10}]{rājadantasya jihvāyāṃ}}\marma}
\pada{\app{\lem[wit={C6,N19,V15}]{jihvayottambhayed}
		\rdg[wit={V3,Gr2}]{jihvayottaṃbhaved}
		\rdg[wit={V1}]{bandhaś ca staṃbhayed}
		\rdg[wit={J10,Jyo}]{bandhaḥ śasto bhaved}}
	\app{\lem[wit={C6,V3,Gr2,N19,V15,Jyo}]{iti}
		\rdg[wit={J10}]{dhitaḥ}
		\rdg[wit={V1}]{dhi tat}}//}\myfn{In \getsiglum{N19,V15} the 2nd hemistich is found betweem 3.28 and 3.29.} % [MD: reference!]
		%\sgwit{C6,P11,V3,Gr2,N19,V15,V1,J10,Jyo}
		\NotIn{N3,J5,Gr3a}\\! %3.22
\endgray


%[hp03_022]
\pada{\app{\lem[wit={ceteri}]{ayaṃ}
		\rdg[wit={J5}]{asaṃ}
		\rdg[wit={N3}]{amuṃ}} 
	\app{\lem[wit={ceteri}]{khalu}% +J8
		\rdg[wit={V3}]{ṣalu}
		\rdg[wit={V1,J10}]{kila}
		\rdg[wit={J5}]{yogī}
		\rdg[wit={N3}]{yoga}
		}
	mahā\app{\lem[wit={ceteri}]{bandho}
		\rdg[wit={J10}]{bandhaḥ}
		\rdg[wit={N3}]{bandhaṃ}}}
\pada{\app{\lem[wit={ceteri}]{mahā}
		\rdg[wit={N23}]{sahā}
		\rdg[wit={J10}]{sarva}}siddhi%
	\app{\lem[wit={ceteri}]{pradāyakaḥ} % +J5,G4,N24; dā  om. in V1
	\rdg[wit={N3}]{pradāyakaṃ}}/}\\+
\pada{kāla\app{\lem[wit={ceteri}]{pāśa} % pāsa V3
		\rdg[wit={N23}]{pāśaṃ}}%
		mahā\app{\lem[wit={ceteri}]{bandha}
		\rdg[wit={J5,N23}]{bandho}
		\rdg[wit={N19}]{baddho}}}%
\pada{\app{\lem[wit={ceteri}]{vimocana}
		\rdg[wit={V3}]{mocayec ca}}% J8ac moccaye
	\app{\lem[wit={ceteri}]{vicakṣaṇaḥ}
		\rdg[wit={V3}]{vicakṣaṇam}}//}\label{III23}\myfn{\getsiglum{Jyo} has a different verse order: \ref{III24}ab \rightarrow\ \ref{III23}abcd \rightarrow\ \ref{III24}cd.}
			\lineom{N3}
%		\sgwit{C6,V3,Gr2,Gr3a,N19,V15,V1,J10,Jyo}\\!  %3.23


%[hp03_023]
\pada{savyāṅge \app{\lem[wit={N3,J5,J7,N19,V15,V1,J10}]{ca samabhyasya}% °sye N3
		\rdg[wit={Jyo}]{tu samabhyasya}
		\rdg[wit={C6,P11,V3}]{pūrvam abhyasya}
		\rdg[wit={N23},alt={\om}]{\skp{\om}}}}
\pada{\app{\lem[wit={C6,V15,V1},alt={dakṣiṇāṅge sam°}]{dakṣiṇāṅge sama}
		\rdg[wit={N3,J5,N19}]{dakṣāṅge ca sam°}% +G7,
		\rdg[wit={N23}]{sam°}
		\rdg[wit={V3,J7,Jyo}]{dakṣāṅge punar}% +M3,G4?
		\rdg[wit={P11}]{dakṣiṇāṅge punar}
		\rdg[wit={J10}]{dakṣiṇe punar}}bhyaset//}\label{III21}
		\NotIn{Gr3a}\myfn{\getsiglum{N3,J5} have this verse here. \getsiglum{Gr3a} omits. The other mss have it after \ref{mahabandha}.}\\!
%		\sgwit{C6,V3,Gr2,N19,V15,J10,V1,Jyo}\\!


%[hp03_024]
\pada{ayaṃ \app{\lem[wit={ceteri}]{ca}
		\rdg[wit={J7,K3,C7,Jyo}]{tu}}
	sarvanāḍīnām} % nāḍīṣu J17
\pada{\app{\lem[wit={ceteri}]{ūrdhvaṃ}
		\rdg[wit={N3,V1,N23,Jyo}]{ūrdhva}}% -ṃ ū- V1
	\app{\lem[wit={N3,P11,V3,V15,V1,J10}]{gativibodhakaḥ}
		\rdg[wit={N19}]{gatinibodhakaḥ} % C8ac/pc difficult to read; °kāḥ N19
		\rdg[wit={Jyo}]{gatinirodhakaḥ}
		\rdg[wit={C6,Gr2,Gr3a}]{gamanabodhakaḥ}}/}\\+ % V3 bodhaka
\pada{triveṇīsaṅgamaṃ dhatte} % °veṇīṃ, dhartte N23; °veṇi V15
\pada{kedāraṃ
	\app{\lem[wit={ceteri}]{prāpayen manaḥ} % prāpyate J5; mana V3
		\rdg[wit={V1}]{prāpaye naraḥ}
		\rdg[wit={N19}]{prāpaye naraṃ}}\marma//}\label{III24}\\!  %3.24


%%%%%%%%%%%%%%
\newpage

\outdent
\app{\lem[wit={C6,C7,N19}]{atha mahāvedhaḥ} % °vedha N19,V3
		\rdg[wit={Gr2,K3}]{mahāvedhaḥ}}/
		\sgwit{Gr2,K3,C7,N19} %\NotIn{V19,N3}


%[hp03_025]
\pada{rūpalāvaṇya\app{\lem[wit={ceteri}]{saṃpannā}
		\rdg[wit={J10}]{saṃpannaṃ}
		\rdg[wit={N23}]{saṃpattī}
		\rdg[wit={V19}]{saṃyuktā}}}
\pada{yathā \app{\lem[wit={ceteri}]{strī puruṣaṃ}
		\rdg[wit={V19}]{nārī patiṃ}} vinā/}\\+
\pada{mahāmudrā%
	mahā\app{\lem[wit={C6,J7,Gr3a,Jyo}]{bandhau}
		\rdg[wit={N3,J5,P11,V3,N23,N19,J10}]{bandho}
		\rdg[wit={V1}]{bandha}
		\rdg[wit={V15},alt={\om}]{\skp{\om}}}}
\pada{\app{\lem[wit={J7,Gr3a,J10,Jyo}]{niṣphalau}
		\rdg[wit={C6,N23}]{niṣphalo}% niḥphalo N23
		\rdg[wit={J5}]{niṣkalaḥ}
		\rdg[wit={N3}]{miṣkalā}
		\rdg[wit={N19}]{mahābaṃdhaṃ} % °baṃdha N19
		\rdg[wit={V3,V1}]{mahāvedha(ṃ)}
		\rdg[wit={V15},alt={\om}]{\skp{\om}}} % °vedhaṃ J8
	\app{\lem[wit={C6,Gr2,Gr3a,Jyo}]{vedhavarjitau} % vetha N23
		\rdg[wit={J5,P11}]{vedhavarjitaḥ}
		\rdg[wit={N3}]{vedhavarttina}
		\rdg[wit={J10}]{vedhavarttitau}
		\rdg[wit={N19,V1}]{vinā tathā}
		\rdg[wit={V3}]{vinānyathā}
		\rdg[wit={V15},alt={\om}]{\skp{\om}}}//} \lineom{cd}{V15}\\!  %3.25

\endverse
\startaltrecension
\outdent
iti mahābandhaḥ/ \sgwit{V1}

\outdent
atha mahāvedhaḥ/ \sgwit{V3,V15,J10,Jyo}
\endaltrecension
\startverse


%\newpage
%[hp03_026]
\pada{\app{\lem[wit={N3,N19,V15,V1,Jyo}]{mahābandha}
		\rdg[wit={J7},post={(followed by a double daṇḍa and corrected to °vedhaḥ)}]{mahābandhaḥ}
		\rdg[wit={P11,N23}]{mahābandho}
		\rdg[wit={C6,Gr3a}]{mahāvedhe}
		\rdg[wit={V3,J10}]{mahāvedha}}%
	\app{\lem[wit={ceteri}]{sthito}
		\rdg[wit={N23}]{sthite}
		\rdg[wit={J10}]{sthitau}} yogī}
\pada{kṛtvā pūraka%
	\app{\lem[wit={C6,J7,V15,V1,Jyo},alt={ekadhīḥ}]{m ekadhīḥ}
		\rdg[wit={N3}]{ekadhī}
		\rdg[wit={V19,N19}]{ekadhā}
		\rdg[wit={K3,C7}]{ekayā}
		\rdg[wit={N23}]{eva dhīḥ}
		\rdg[wit={V3}]{eva dhī}
		\rdg[wit={J5}]{eva ca dhā}}/}\\+
\pada{\app{\lem[wit={J7,Gr3a,V15,Jyo}]{vāyūnāṃ}
		\rdg[wit={V1}]{vāyunāṃ}
		\rdg[wit={N3,J5,C6,V3,N23,N19,J10}]{vāyunā}}
	\app{\lem[wit={ceteri}]{gatim āvṛtya}
		\rdg[wit={N3}]{gam āvṛtya}
		\rdg[wit={J5,V15}]{gatim ākṛṣya}}} % gatin? V15
\pada{nibhṛtaṃ\marmas kaṇṭha%
	\app{\lem[wit={ceteri}]{mudrayā}
		\rdg[wit={J10}]{mudrāyā}
		\rdg[wit={N23},postwit=\texteng{(jumped to pāda c after gatim)},alt={\om}]{\skp{\om}}}//}\\!  %3.26

%[hp03_027]
\pada{\app{\lem[wit={ceteri}]{samahasta}
		\rdg[wit={N3}]{samahāsta}% nyastahasta J5
		\rdg[wit={J10}]{samahastā}
		\rdg[wit={N23}]{samahaste}
		\rdg[wit={C6}]{samau hasta}}%
	\app{\lem[wit={V3,N23,V19,J10,Jyo}]{yugo}% yuga J5
		\rdg[wit={C6,J7,K3,C7,V15,V1}]{yugau}
		\rdg[wit={N3,N19}]{yuge}} bhūmau}
\pada{\app{\lem[wit={N3,P11,V3,J7,Gr3a,J10}]{sphijau}
		\rdg[wit={N23,Jyo}]{sphicau}
		\rdg[wit={V1}]{sphidau} % rather sphiṭṭau?
		\rdg[wit={C6}]{sphītau}
		\rdg[wit={N19}]{dvijāt}
		\rdg[wit={V15}]{dvijā}}
	\app{\lem[wit={ceteri},alt={saṃtāḍayec}]{saṃtāḍaye} % saṃtāḍanec N23; °yechanaiḥ J10
		\rdg[wit={V1}]{saṃ[c]ālayec}
		\rdg[wit={V15}]{nutāḍayec}}c chanaiḥ/}\\+ % V3 om. ḥ; chūnaiḥ N19
\pada{\app{\lem[wit={ceteri}]{puṭadvayaṃ}
		\rdg[wit={N23}]{jaṃghāyuṭadvayam}}
	\app{\lem[wit={ceteri}]{samākramya}% +K3
		\rdg[wit={J5,J7,C7}]{samākṛṣya}% +J5,N24
		\rdg[wit={N23}]{ākṛṣya}
		\rdg[wit={Jyo}]{atikramya}}}
\pada{\app{\lem[wit={C6,J7,Gr3a,V1,J10,Jyo}]{vāyuḥ}
		\rdg[wit={N3,P11,V3,N23,N19,V15}]{vāyu}}
		sphurati % V3 sphuraṃti
	\app{\lem[wit={N3,J5,N19,J10}]{satvaraṃ}% +M1,M3,G7 ## ratvaraṃ P11
		\rdg[wit={V3}]{tatvaraṃ}
		\rdg[wit={V1}]{tatparaṃ}
		\rdg[wit={C6}]{tatparaḥ}
		\rdg[wit={Gr2,V19,V15,Jyo}]{madhyagaḥ}
		\rdg[wit={K3,C7}]{madhyamaḥ}
		}/}\\+
\graus{\pada{bandhenānena
	\app{\lem[wit={J7}]{yogīndraḥ}\rdg[wit={N23}]{yogīndra}}}
	\pada{sādhayet sarvam īpsitam//} \sgwit{Gr2}}\\!  %3.27


%[hp03_028]
\pada{somasūryāgni\app{\lem[wit={N19,Jyo}]{saṃbandho}% M1
		\rdg[wit={V3,V1}]{sambandhā}
		\rdg[wit={N3,J5,C6,Gr2,J10}]{sambandhāj}
		\rdg[wit={Gr3a,V15}]{saṃdhānaṃ}}}\marmas
\pada{jāyate % jāyata? V15
	\app{\lem[wit={N3,P11,Jyo}]{cāmṛtāya vai}
		\rdg[wit={C6,Gr2,N19,V15}]{cāmṛtāyate}
		\rdg[wit={Gr3a}]{vāmṛtāyate}
		\rdg[wit={V1}]{cāmṛtāye vaiḥ}
		\rdg[wit={V3}]{ca mṛtāya vai}
		\rdg[wit={J10}]{ca mṛturjayaḥ}}\marma/}\\+
\pada{\app{\lem[wit={ceteri}]{mṛtāvasthā}
		\rdg[wit={N23}]{mṛtāmasthā}}
	\app{\lem[wit={ceteri}]{samutpannā}
		\rdg[wit={N23},alt={\om}]{\skp{\om}}}}
\pada{tato \app{\lem[wit={N3,C6,V3,J7,Gr3a,N19}]{mṛtyubhayaṃ kutaḥ}
		\rdg[wit={V15,V1,J10,Jyo}]{vāyuṃ virecayet} % °cayat V15
		\rdg[wit={N23}]{vāyuṃ nirundhayet kumbhakena}}\marma//}\\!  %3.28

\newpage
%[hp03_029]
\pada{\app{\lem[wit={ceteri}]{mahāvedho}
		\rdg[wit={V15}]{mahābaṃdho}}%
		'ya\app{\lem[wit={N3,C6,V3,J7,K3,C7,J10,Jyo},alt={abhyāsān}]{m abhyāsā} % V19 °vedhopamanabhyā°?
		\rdg[wit={N23}]{abhyāsāt}
		\rdg[wit={V19}]{anabhyāsān}
		\rdg[wit={V1}]{abhyāso}
		\rdg[wit={N19,V15}]{abhyasto}}n}
\pada{\app{\lem[wit={ceteri}]{mahā} % K3 unclear
		\rdg[wit={N23}]{sarva}}siddhipradāyakaḥ/}\\+ % siddhiḥ J10pc; dāyaka V3
\pada{\app{\lem[wit={ceteri}]{valī}% vaḷī V15
		\rdg[wit={J10},post={\unm}]{valīta}
		\rdg[wit={N23,V1}]{vali}}%
	\app{\lem[wit={ceteri}]{palita}
		\rdg[wit={J7}]{palīta}}%
	\app{\lem[resp=emend,(cf.\,\getsiglum{C8})]{roga}
		\rdg[wit={Gr3a}]{vega} % corrupt from roga? C8
		\rdg[wit={N3,C6,V3,Gr2,N19,V15,V1}]{vedha} % J7ac
		\rdg[wit={Jyo}]{vepa}
		\rdg[wit={J10}]{bandha} % J7pc
		}\marma%
	\app{\lem[wit={ceteri}]{ghnaḥ}
		\rdg[wit={N3,V3}]{ghnaṃ}
		\rdg[wit={N23}]{ghna}}}
\pada{sevyate % savyate N19, sevyato K3
	\app{\lem[wit={ceteri}]{sādhakottamaiḥ}
		\rdg[wit={V3}]{sādhakottamaṃ}}//}\\!  %3.29


%\newpage
%[hp03_030]
\pada{\app{\lem[wit={ceteri}]{etat trayaṃ mahā}% °traya N23; V3,V19 etatrayaṃ; J10 sahā
		\rdg[wit={V15}]{mahāmudrātrayaṃ}
		\rdg[wit={V1},post={\unm}]{mahāmudrātrayatraṃ}}%
	\app{\lem[wit={N3,C6,V3,Gr2,N19,V15,V1,Jyo}]{guhyaṃ}
		\rdg[wit={Gr3a}]{guptaṃ}
		\rdg[wit={J10}]{mudrā}}}
\pada{jarāmṛtyu\app{\lem[wit={ceteri}]{vināśanam}
		\rdg[wit={J10}]{vināśinī}}/}\\+
\pada{vahnivṛddhikaraṃ % buddhivṛddhi C6
	\app{\lem[wit={N3,C6,Gr2}]{caiva}
		\rdg[wit={V3,N19,V15,J10}]{caivam}
		\rdg[wit={Jyo}]{caiva hy}
		\rdg[wit={Gr3a}]{caitad}
		\rdg[wit={V1}]{viśvam}}}
\pada{aṇimādi\app{\lem[wit={ceteri}]{guṇa}
		\rdg[wit={N19}]{gaṇa}}%
	\app{\lem[wit={ceteri}]{pradam}
		\rdg[wit={N23}]{pradī}}//}\\!  %3.30

%\newpage
%[hp03_031]
\pada{aṣṭadhā kriyate % aṣṭādi C6
	\app{\lem[wit={N3,C6}]{caitad}% ## caihad G7?
		\rdg[wit={Gr3a,N19,Jyo}]{caiva}% +N24, taitva J5
		\rdg[wit={P11,Gr2}]{caivaṃ}% M3
		\rdg[wit={V3,V1,J10}]{caikaṃ}
		\rdg[wit={V15}]{caika}}}
\pada{\app{\lem[wit={ceteri}]{yāme yāme}
		\rdg[wit={V15}]{yāmayāme}
		\rdg[wit={V1}]{yāmaṃ yamāṃ}} dine dine/}\\+
\pada{\app{\lem[wit={ceteri}]{puṇya}
		\rdg[wit={V15}]{puṇyaṃ}
		\rdg[wit={J10}]{sarva}}%
	\app{\lem[wit={N3,J7,Gr3a,N19,Jyo}]{saṃbhāra}
		\rdg[wit={V3}]{sahāra}
		\rdg[wit={V1,J10}]{saṃcāra}
		\rdg[wit={V15}]{saṃsāra}
		\rdg[wit={C6}]{saṃdhāna}
		\rdg[wit={N23},alt={\om}]{\skp{\om}}}%
	\app{\lem[wit={V3,Gr2,N19}]{sambhāvi}
		\rdg[wit={N3,J5}]{saṃbhāvī}
		\rdg[wit={V1}]{sabhāvī}
		\rdg[wit={J10}]{saṃdhāyī}
		\rdg[wit={C6,V15,Jyo}]{saṃdhāyi}
		\rdg[wit={Gr3a}]{saṃpādi}}}
\pada{\app{\lem[wit={ceteri}]{pāpaugha}
		\rdg[wit={J7}]{pāprogha}
		\rdg[wit={N23}]{padhau\,\_\,dhava}}%
		bhiduraṃ sadā//}\\!  %3.31


%[hp03_032]
\pada{samyak\app{\lem[wit={ceteri},alt={śikṣāvatām}]{śikṣāvatā} % sikṣī V19; °catām N23
		\rdg[wit={C6}]{śikṣāvatā}
		\rdg[wit={J5,N19}]{śiṣyāvatām}
		\rdg[wit={J10}]{jijñāsatām}}%
	\app{\lem[wit={J5,P11,Gr2,Gr3a,V15},alt={eva}]{m eva} % M3?
		\rdg[wit={N3,V3,N19,V1,J10,Jyo}]{evaṃ}
		\rdg[wit={C6}]{bhavyaṃ}}}
\pada{svalpaṃ % V1 svaplaṃ
	prathama\app{\lem[wit={J5,C6,N23,Gr3a,N19,V15,V1}]{sādhane}
		\rdg[wit={N3}]{sādhanaiḥ}
		\rdg[wit={V3,J7,J10,Jyo}]{sādhanaṃ}}/}\\+ % pratyama N23; J10ac sādhana?
\pada{vahnistrīpatha% stri N19; paṭha N3
	\app{\lem[wit={ceteri},alt={sevānām}]{sevānā}
		\rdg[wit={N19}]{sevācanām}
		\rdg[wit={J10}]{sevanām}
		\rdg[wit={V1}]{sevanam}
		\rdg[wit={N23}]{sevenam}
		\rdg[wit={Jyo},alt={\om}]{\skp{\om}}}}% °nāṃmādau J7
\pada{m ādau varjana%
	\app{\lem[wit={N3,J5,C6,N19,V15,V1}]{ādiśet}
		\rdg[wit={V3}]{ādṛśyet}
		\rdg[wit={Gr2,Gr3a,J10},alt={ācaret}]{m ācaret}% +M3,G7
		\rdg[wit={Jyo},alt={\om}]{\skp{\om}}}//}\myfn{%
	\getsiglum{Gr2} adds here:
	\nagari{mahāmudrā mahābandho mahāvedhaś ca nityaśaḥ/ % °bandhā N23
	etat trayaṃ prayatnena caturvāraṃ karoti yaḥ/
	ṣaṇmāsābhyantare mṛtyuṃ jayaty eva na saṃśayaḥ//} % mṛtyu N23
	 (= Śivasaṃhitā xx)}
	\lineom{cd}{Jyo}\myfn{\getsiglum{Jyo} has this line just before 1.61.}\\!  %3.32

%%%%%%%%%%%%%%%%%%%%%%%%
%\newpage
\outdent
\app{\lem[wit={ceteri}]{atha}
\rdg[wit={Gr2,K3},alt={\om}]{\skp{\om}}} khecarī/

\endverse
\startaltrecension{}
%[hp03_032_1]
\pada{nāsanaṃ
	\app{\lem[wit={V3,J6,N9}]{siddhasadṛśaṃ}
%		\rdg[wit={J8ac}]{siddhe sadṛśaṃ}
		\rdg[wit={V17}]{sadṛśaṃ siddhaṃ}}} % proof that J8 is a copy of V3? Halanta of k in the above line was mistaken for e?(V3=9v4, J8=16r6)
\pada{na \app{\lem[resp=emend]{kumbhaṃ}
		\rdg[wit={V3,N9}]{kumbha}
		\rdg[wit={V17}]{kumbhaḥ}
		\rdg[wit={J6}]{kuṃbhaka}}
	\app{\lem[resp=emend]{kevalopamam} % =J15
		\rdg[wit={V3,N9}]{kevalokanam}
		\rdg[wit={V17}]{kevalo mataḥ}
		\rdg[wit={J6}]{samonilaṃ}}/}\\+
\pada{na khecarīsamā mudrā} \pada{na nādasadṛśo layaḥ//}
\sgwit{Gr6} \anm{= 1.43}\\!
\endaltrecension
\startverse

\newpage
%[hp03_033]
\pada{\app{\lem[wit={V3,N23,V19,C7,J6,V15,V1,Jyo}]{chedana}% +J17
		\rdg[wit={J10,N9,V17}]{chedanaṃ}% +J8
		\rdg[wit={C6}]{chedanaiś}
		\rdg[wit={K3}]{khedana}
		\rdg[wit={N19}]{vedana}
		\rdg[wit={J7}]{rasanā}
		\rdg[wit={N3},alt={\illeg}]{}}%
	\app{\lem[wit={J7,Gr3a,J6,N19,V15,V1,Jyo}]{cālanadohaiḥ}
		\rdg[wit={N23}]{cālajadohaiḥ}
		\rdg[wit={V17}]{cālanaṃ dohaiḥ}
		\rdg[wit={V3,J10,N9}]{cālanaṃ dohau}
		\rdg[wit={C6}]{cālanair dāsyai}
		\rdg[wit={N3},alt={\illeg}]{}}
\app{\lem[wit={C6,J7,N19,V1,Jyo}]{kalāṃ}
		\rdg[wit={N23}]{kalaṃ}
		\rdg[wit={N3}]{kalāḥ}
		\rdg[wit={V3,N9,V17}]{kalā}
		\rdg[wit={J10}]{kāla}
		\rdg[wit={Gr3a,J6}]{jihvāṃ} % jihvā C7, jihva V19
		\rdg[wit={V15}]{krameṇa}}
	\app{\lem[wit={Gr2}]{tu}
		\rdg[wit={Gr3a}]{vai}
		\rdg[wit={V15}]{jihvāṃ}
		\rdg[wit={N3,C6,V3,N19,V1,J10,N9,V17}]{krameṇa}
		\rdg[wit={Jyo}]{krameṇātha}
		\rdg[wit={J6},alt={\om}]{\skp{\om}}}
	\app{\lem[wit={Gr2,J6},alt={saṃvardhayet}]{saṃvardhaye}
		\rdg[wit={V15,V1,J10}]{pravardhayet}
		\rdg[wit={N3,C6,V3,Gr3a,N19,N9,V17,Jyo}]{vardhayet}}%
	\app{\lem[wit={C6,J7,Gr3a,J6,N19,V15,Jyo},alt={tāvat}]{t tāvat}
		\rdg[wit={J10}]{tāt}
		\rdg[wit={N3,V3,N23,V1,N9,V17},alt={\om}]{\skp{\om}}}/}\\+
\pada{\app{\lem[wit={Gr2,Gr3a,J6}]{yāvad iyaṃ}% yāvaṃd iya
		\rdg[wit={N3,C6,V15,V1,J10,Jyo}]{sā yāvad}% yāva N3, pāvad J10
		\rdg[wit={N19}]{sā}
		\rdg[wit={V3,N9,V17}]{yāvad}}
	bhrū\app{\lem[wit={ceteri}]{madhyaṃ} % bhū N23, bhṛ N3
		\rdg[wit={V19,V1}]{madhya}}
	\app{\lem[wit={ceteri}]{spṛśati}
		\rdg[wit={N23}]{sparśati}
		\rdg[wit={N3}]{viśa}}
\app{\lem[wit={Gr2,C7,J6,V15,J10,Jyo}]{tadā khecarīsiddhiḥ}
		\rdg[wit={N3,C6,V3,N9,V17}]{tadānīṃ khecarīsiddhiḥ}% °nī siddhi C6
		\rdg[wit={N19}]{tadānīṃ hi khecarīsiddhiḥ}
		\rdg[wit={V1}]{tadānī siddhiḥ}
		\rdg[wit={V19}]{tadā khecarī bhavati}}//}\myfn{Based on \getsiglum{Gr2,Gr3a}. The metre is Upagīti.\\
	\getsiglum{N19,V15} (Gīti?):
		\nagari{chedanacālanadohaiḥ kalāṃ krameṇa (pra)vardhayet tāvat/
		sā yāvad bhrūmadhyaṃ spṛśati tadānīṃ hi khecarīsiddhiḥ//}\\
	\getsiglum{J10} (Āryā):
		\nagari{chedanacālanadohaiḥ kalāṃ krameṇa pravardhayet tāvat/
		sā yāvad bhrūmadhyaṃ spṛśati tadā khecarīsiddhiḥ//}\\
	\getsiglum{V3} (Anuṣṭubh):
		\nagari{chedanaṃ cālanaṃ dohau, kalākrameṇa vardhayet/
		yāvad bhrūmadhyaṃ spṛśati tadānīṃ khecarīsiddhiḥ(!)//}\\
	Perhaps to read:
	\nagari{chedanacālanadohaiḥ, kalāṃ kramād vardhayet tāvat/
	sā yāvad bhrūmadhyaṃ, spṛśati tadā khecarīsiddhiḥ//}
	}\\!  %3.33
\endverse

%\newpage
% Khecaryabhyāsakrama

\startaltrecension{}

%[hp03_033_01] % KhV 1.46
\pada{\app{\lem[wit={J6,V15,V17,Jyo}]{snuhī} % snuhi V15,N9
		\rdg[wit={V3}]{śnuhi}}pattranibhaṃ śastraṃ} % putra J6ac
\pada{sutīkṣṇaṃ snigdhanirmalam/}\\+
\pada{samādāya tatas tena}
\pada{romamātraṃ \app{\lem[wit={V3,V17}]{samucchidet} % cci J15
		\rdg[wit={V17}]{samucchidat}
		\rdg[wit={V15,N9,Jyo}]{samucchinet}
		\rdg[wit={J6}]{samucchiṃdyāt}}//} \sgwit{V15,Gr6,Jyo}\\!

%[hp03_033_02] % KhV 1.47
\pada{\app{\lem[wit={V15,Gr6}]{kṛtvā}
		\rdg[wit={Jyo}]{tataḥ}}
	\app{\lem[wit={V3,J6,N9}]{saindhavapathyādi}
		\rdg[wit={V17,Jyo}]{saindhavapathyābhyāṃ}
		\rdg[wit={V15}]{saindhavapakṣyādi}}}%
\pada{\app{\lem[wit={V3,J6,V15,N9,Jyo}]{cūrṇitābhyāṃ} % cūṇi° V15
		\rdg[wit={V17}]{cūrṇaṃtābhyāṃ}}
	\app{\lem[wit={V3,J6,V15,N9,Jyo}]{pragharṣayet}
		\rdg[wit={V17}]{ca gharṣayet}}/}\\+
\pada{punaḥ saptadine prāpte}
\pada{romamātraṃ \app{\lem[wit={V3,N9}]{samucchidet}
		\rdg[wit={V17}]{samucchidat}
		\rdg[wit={Jyo}]{samucchinet}
		\rdg[wit={V15}]{punaḥ chidet}
		\rdg[wit={J6}]{samutthiyāt}}//} \sgwit{V15,Gr6,Jyo}\\!

%[hp03_033_03] % KhV 1.48
\pada{evaṃ krameṇa \app{\lem[wit={V3,J6,V17,Jyo}]{ṣaṇmāsaṃ}
		\rdg[wit={N9}]{ṣaṇmāse}
		\rdg[wit={V15}]{ṣaṇmāsān}}}
\pada{\app{\lem[wit={V3,J6,V15,V17}]{nitya}
		\rdg[wit={Jyo}]{nityaṃ} % +J15
		\rdg[wit={N9}]{netya}}%
	\app{\lem[wit={V3,V15,N9,V17}]{yuktaṃ}
		\rdg[wit={Jyo}]{yuktaḥ}
		\rdg[wit={J6}]{muktaṃ}} samācaret/}\\+
\pada{\app{\lem[wit={Gr6,Jyo}]{ṣaṇmāsād}
		\rdg[wit={V15}]{ṣaṇmāse}}
	rasanā\app{\lem[wit={V3,J6,N9,Jyo}]{mūla}
		\rdg[wit={V15,V17}]{mūlaṃ}}}%
\pada{\app{\lem[wit={V3}]{śarābandhaṃ}
		\rdg[wit={J6}]{śarabaṃdhaṃ}
		\rdg[wit={N9}]{śarābadho}
		\rdg[wit={V15,Jyo}]{śirābandhaḥ}
		\rdg[wit={V17}]{śirobandha}}
	\app{\lem[wit={Gr6}]{vinaśyati}
		\rdg[wit={V15,Jyo}]{praṇaśyati}}//} \sgwit{V15,Gr6,Jyo}\\!

%\newpage
%[hp03_033_04] % KhV 1.49
\pada{atha vāgīśvarīdhāma}% śvaro? V17
\pada{śiro vastreṇa veṣṭayet/}\\+
\pada{śanair utkarṣayed yogī}
\pada{kālavelā% kālā N9
	\app{\lem[wit={V3,J6,V17}]{vidhānavit}
		\rdg[wit={N9}]{vidhānataḥ}}//} \sgwit{Gr6}\\!

%\newpage
%[hp03_033_05] % from long recension of the Yogabīja
\pada{vitasti\app{\lem[wit={V3,J6}]{pramitaṃ}
		\rdg[wit={N9}]{pratima}
		\rdg[wit={V17}]{prathame}}
	\app{\lem[wit={V3,N9,V17}]{dairghyaṃ}% dairghaṃ N9
		\rdg[wit={J6},post={(°rghye \textit{pc})}]{dairghya}}}
\pada{vistāraṃ caturaṅgulam/}\\+
\pada{mṛdulaṃ dhavalaṃ proktaṃ}
\pada{veṣṭitāmbaralakṣaṇam//} \sgwit{Gr6}\\!

\newpage
%[hp03_033_06] % KhV 1.50
\pada{\app{\lem[wit={V3,J6,N9}]{punaḥ}
		\rdg[wit={V17}]{puṇāḥ}} ṣaṇmāsamātreṇa}
\pada{punaḥ saṃkarṣaṇāt priye/}\\+ % °karṣaṇa N9
\pada{bhrūmadhyāvadhi vardheta} % vaddheta V17
\pada{tiryakkarṇa% tiryakarṇa V3,N9
	bilāvadhi//} \sgwit{Gr6}\\!

%\newpage
%[hp03_033_07] % KhV 1.51ab, 1.52ab
\pada{adhastā\app{\lem[wit={J6,N9},alt={cibukaṃ mūlaṃ}]{c cibukaṃ mūlaṃ} % cf. J3
		\rdg[wit={V17}]{cikukaṃ mūlaṃ}
		\rdg[wit={V3}]{cibukamūla}}}
\pada{prayāti \app{\lem[wit={V3,J6}]{kramakāritā}
		\rdg[wit={N9}]{tramakārikā}
		\rdg[wit={V17}]{yuktakāritā}}/}\\+
\pada{krośād ūrdhvaṃ % ṃ om. V17, krośāharddhaṃ J8
	\app{\lem[wit={V3,J6,N9}]{ca}
		\rdg[wit={V17},alt={\om}]{\skp{\om}}}
	\app{\lem[wit={V3,N9,V17}]{kramati}
		\rdg[wit={J6}]{krāmati}}}
\pada{tiryak\app{\lem[wit={V3,J6,V17}]{saṃkhyā}
		\rdg[wit={N9}]{saṃsthā}}vadhi priye//} \sgwit{Gr6}\\!

%[hp03_033_08] % = HP6 and HP10
\pada{punaḥ saṃvatsarād devi}
\pada{dvitīyā caiva līlayā/}\\+ % dvitiyā V3,J15; lilayā N9,J15
\pada{brahma\app{\lem[wit={V3,J6,V17},alt={randhrāntam}]{randhrānta}
		\rdg[wit={N9}]{raṃdhraṃ tam}}m āvṛtya}
\pada{\app{\lem[wit={V3,J6,V17}]{tiṣṭhet}
		\rdg[wit={N9}]{viṣṭaitet}}
	\app{\lem[wit={V3,J6,V17}]{paramavandite}
		\rdg[wit={N9}]{paramavidite}}//} \sgwit{Gr6}\\!

%\newpage
%[hp03_033_09]% = HP6 and HP10
\pada{sva\app{\lem[wit={V3,J6,N9}]{tālu}
		\rdg[wit={V17}]{tālaṃ}}mūlaṃ saṃghṛṣya}
\pada{saptavāsaram ātmani/}\\+ % atmani N9
\pada{svagurūktaprakāreṇa}
\pada{malaṃ sarvaṃ viśoṣayet//} % viśeṣayet J8
\sgwit{Gr6}\\!

%\newpage
%[hp03_033_10]% = HP10 (33.10ab = KhV 1.56cd)
\pada{aṅgulyagreṇa saṃghṛṣya}
\pada{jihvāṃ tatra niveśayet/}\\+
\pada{śanaiḥ śanair mastakāc ca} % one śanair om. N9
\pada{mahāvajrakapāṭabhit//} \sgwit{Gr6}\\! % bhīt J15

%\newpage
%[hp03_033_11]% = KhV 1.34
\pada{pūrva\app{\lem[wit={N9,V17}]{bīja}
		\rdg[wit={V3}]{vīya}
		\rdg[wit={J6}]{vīrya}}yutāṃ vidyāṃ}
\pada{\app{\lem[wit={V3,J6,N9}]{vyākhyātām ati}
		\rdg[wit={V17}]{vikhyātām api}}durlabhām/}\\+
\pada{asyāḥ \app{\lem[wit={V3,J6}]{ṣaḍaṅgaṃ}
		\rdg[wit={V17}]{ṣaḍaṅgaḥ}
		\rdg[wit={N9}]{ṣaḍaṃhva}} kurvīta}
\pada{tayā ṣaṭcakra% ṣaḍvakra? V17
	bhinnayā//} \sgwit{Gr6}\\!

%\newpage
% metre: Rathoddhatā
%[hp03_033_12]% = HP6 and HP10
\pada{khe nirastasakalakriyākrame}\\+ % cirastasakalā N9
\pada{\app{\lem[resp=emend,alt={yā citiś},postwit=\texteng{(cf.\,Yoginīhṛdaya)}]{yā citi} % J15 yojitaś, many other mss °ṇa cittaś (unmetrical)
		\rdg[wit={N9,V17}]{yācitaś}
		\rdg[wit={V3}]{yā cittaś}
		\rdg[wit={J6}]{°ṇa cittaś}}ś carati
	\app{\lem[wit={V3,J6,N9}]{śāśvatodaye}
		\rdg[wit={V17}]{svāścatodaye}}/}\\+
\pada{sā śivatva\app{\lem[wit={V3,N9,V17}]{samavāya} % śivatta N9
		\rdg[wit={J6}]{samavāyi}}%
	\app{\lem[wit={J6,N9,V17}]{kāriṇī}
		\rdg[wit={V3}]{kariṇī}}}\\+
\pada{khecarī ca
	\app{\lem[wit={V3,N9,V17}]{bhava}
		\rdg[wit={J6}]{bhavati}}khedahāriṇī//} \sgwit{Gr6}\\!

%\newpage
%[hp03_033_13]% = HP6 and HP10, ~ KhV 1.54
\pada{krameṇaiva
	\app{\lem[wit={V3,N9,V17}]{prakartavyā}
		\rdg[wit={J6}]{pravartavyā}}}%
\pada{\app{\lem[wit={V3,J6,N9}]{bhyāsena}
		\rdg[wit={V17}]{bhyāsana}}
	vara\app{\lem[wit={V3,N9}]{varṇini}
		\rdg[wit={J6}]{varṇinī}
		\rdg[wit={V17}]{varṇiṇī}}/}\\+
\pada{yugapad
	\app{\lem[wit={V3,J6,N9},post={(yatete \getsiglum{J6}\textsuperscript{ac})}]{yatate} % yatete J6ac
		\rdg[wit={V17}]{utpadyate}} tasya}
\pada{śarīraṃ vilayaṃ vrajet//} \sgwit{Gr6}\\!

\newpage
%[hp03_033_14]% = HP6 and HP10, KhV 1.55ab
\pada{tasmāc chanaiḥ śanaiḥ kāryo}% one śanaiḥ om. N9
\pada{\app{\lem[wit={V3,J6}]{'bhyāso na}
		\rdg[wit={N9,V17}]{bhyāsena}} yugapat priye/}\\+
\pada{\app{\lem[wit={J6,N9,V17}]{evaṃ}
		\rdg[wit={V3}]{eva}} varṣatrayaṃ kṛtvā}
\pada{brahma\app{\lem[wit={V3,J6,N9}]{dvāraṃ}
		\rdg[wit={V17}]{dvāre}}
	\app{\lem[wit={J6,V17},alt={viśed}]{viśe}
		\rdg[wit={N9}]{vaśed}
		\rdg[wit={V3}]{biśe}}d dhruvam//} \sgwit{Gr6}\\!

%\newpage
% metre: Śārdūlavikrīḍita
%[hp03_033_15]% = HP6 and HP10
\pada{saṭcakrāṇi vibhidya śakti\app{\lem[wit={J6}]{bhujagīṃ}
		\rdg[wit={V3}]{bhujaṃgī}
		\rdg[wit={N9,V17}]{bhujaṃgīṃ}}
	\app{\lem[wit={V3,N9,V17}]{protthāpya}
		\rdg[wit={J6}]{protthāya}} mūlasthitāṃ}\\+ % stitāṃ N9
\pada{bhittvā granthitrayaṃ ca % tra of trayaṃ treated as a single consonant?
paścimaśirāprākārarūpaṃ mahat/}\\+
\pada{nītvā prāṇam ataḥ śirobilam alaṃ % nitvā V17; anaḥ? J6; ala N9
	nirmathya cittena
	\app{\lem[wit={V3}]{tat}
		\rdg[wit={J6,N9,V17}]{tal}}}\\+
\pada{liṅgaṃ yaḥ % ya V17, yāḥ N9
\app{\lem[wit={J6,V17}]{pibatī}
		\rdg[wit={V3,N9}]{pibate}}ndumaṇḍala%
	\app{\lem[wit={V3,J6,N9}]{galan}% galat V3,J15
		\rdg[wit={V17}]{gataṃ}}\
	\app{\lem[wit={V3,V17}]{muktaḥ sa sākṣācchivaḥ}
		\rdg[wit={J6}]{muktaś ca sākṣācchivaḥ}
		\rdg[wit={N9}]{muktaḥ kṣamāddhivaḥ}}//} \sgwit{Gr6}\\!

%\newpage
% metre: Sragdharā
%[hp03_033_16]% = HP6 and HP10
\pada{nityaṃ
	\app{\lem[wit={V3,N9,V17}]{yas tūrdhva}
		\rdg[wit={J6}]{yasphūrja}}%
	\app{\lem[resp=emend,postwit={(=\,\getsiglum{N26})}]{jihvo yadi}
		\rdg[wit={V3}]{jihvogradi}
		\rdg[wit={J6}]{jihvāgrayā}
		\rdg[wit={N9}]{jihvā yadi}
		\rdg[wit={V17}]{jihvāṃ yadi}}
pibati pumān
saptadhārāmṛ\app{\lem[wit={V3,V17}]{taughaṃ}
	\rdg[wit={N9}]{tauṣaṃ}
	\rdg[wit={J6}]{tauccaṃ}}}\\+
\pada{\app{\lem[wit={V3,N9,V17}]{susvādaṃ}
	\rdg[wit={J6}]{su[kha]daṃ}}
	\app{\lem[wit={V3,J6,N9}]{śītalāṅgaṃ}
	\rdg[wit={V17}]{śītalāṅge}}
	durita\app{\lem[wit={V3,J6,N9}]{bhaya}
	\rdg[wit={V17}]{maya}}haraṃ % rahaṃ N9
	kṣutpipāsānivāri/}\\+ % nivāritaṃ J6ac
\pada{piṇḍasthairyaṃ hi tasmād
	\app{\lem[wit={V3,J6,N9}]{bhavati}
	\rdg[wit={V17}]{avati}}
	\app{\lem[wit={N9}]{mṛtapathā} % mṛtta N9
	\rdg[wit={V3,V17}]{mṛtayathā}
	\rdg[wit={J6}]{mṛtaṃ yathā}}
	mṛtyu\app{\lem[wit={V3,J6,V17}]{rogād} % mṛtya V17
		\rdg[wit={N9}]{śeṣād}}
	\app{\lem[wit={V3,N9,V17}]{bhavanti}
		\rdg[wit={J6}]{bhavati}}}\\+ % mṛtya V17
\pada{\app{\lem[wit={J6,V17}]{daurbhāgyaṃ}
	\rdg[wit={V3,N9}]{daurbhyāgyaṃ}}
yāti nāśaṃ prasarati
	\app{\lem[wit={V3,J6,N9}]{sakalaṃ}
	\rdg[wit={V17}]{sakala}} yāti
	\app{\lem[wit={V3,N9}]{kālaṃ}
	\rdg[wit={J6,V17}]{kālo}} bhramitvā//} \sgwit{Gr6}\\!


%[hp03_033_17]% = HP6 and HP10
\pada{\app{\lem[wit={J6}]{tīkṣṇakaṃ}
	\rdg[wit={V3}]{tīkṣṇake}% ṇaṃke V17
	\rdg[wit={N9}]{tīkṣṇako}} harate vyādhiṃ} % vyādhiḥ N9
\pada{\app{\lem[wit={V3,J6,V17}]{kaṭukaṃ kuṣṭhanāśanam}
	\rdg[wit={N9}]{kaṭukuṭivināśanaṃ}}/}\\+ % u of kaṭuka resembles to a virāma in V3,N9
\pada{\app{\lem[wit={V3,J6,V17}]{ghṛta}
		\rdg[wit={N9}]{dhṛtvā}}svādūpamaṃ caiva}   % caivāmaratvaṃ J6
\pada{amaratvaṃ \app{\lem[wit={V3,V17}]{labhed}
		\rdg[wit={J6,N9}]{labhate}} dhruvam//} \sgwit{Gr6}\\! % labhe V3

%\newpage
%[hp03_033_18]% = HP6
\pada{madhusvādūpamaṃ
	\app{\lem[wit={V3,J6,N9}]{caiva}
	\rdg[wit={V17}]{caivaṃ}}}
\pada{śāstram
	\app{\lem[wit={V3,N9}]{udgirate}
	\rdg[wit={J6}]{udgirati}
	\rdg[wit={V17}]{udgire}}
	\app{\lem[wit={J6,N9}]{bahu}
	\rdg[wit={V3,V17}]{bahuḥ}}/}\\+
\pada{\app{\lem[resp=emend]{laḍḍu}
	\rdg[wit={Gr6}]{laḍu}}%
	\app{\lem[wit={V3,N9}]{ṣaṇḍakapādyāni}
	\rdg[wit={J6}]{khaṃḍakapādyāni}
	\rdg[wit={V17}]{piṇḍakaṣādyāni}}}
\pada{pakvānnāni anekaśaḥ//} \sgwit{Gr6}\\!

%[hp03_033_19]% = HP6
\pada{divyakalpaṃ
	\app{\lem[wit={ceteri}]{ramen}
	\rdg[wit={J6}]{krīḍen}} nityaṃ} % rame J8
\pada{utkṛṣṭo jāyate dhruvam/}\\+
\pada{tanmayatvam avāpnoti}
\pada{\app{\lem[wit={J6,V17}]{kośa}
	\rdg[wit={V3}]{kauśa}
	\rdg[wit={N9}]{kauṣṭa}}kārīva kīṭakaḥ//} \sgwit{Gr6}\\!

\endaltrecension
\startverse

\newpage
%[hp03_034]
\pada{kapālakuhare jihvā} % kalāpa C6
\pada{\app{\lem[wit={ceteri}]{praviṣṭā viparītagā}
	\rdg[wit={N3}]{pra\,+\,+\,+\,+\,+\,+}}/}\\+
\pada{bhruvor  % bhṛ° N3, bhrū° V3, bhrūvaur N23
	\app{\lem[wit={ceteri}]{antargatā}
	\rdg[wit={N3}]{aṃtagatā}
	\rdg[wit={C6}]{madhye gatā}}
	\app{\lem[wit={ceteri}]{dṛṣṭir}
		\rdg[wit={N3,N23,V15}]{dṛṣṭi}}}
\pada{mudrā bhavati
	\app{\lem[wit={ceteri}]{khecarī}
		\rdg[wit={V1}]{carī}}//}\myfn{\getsiglum{Jyo} has this verse at the very beginning of the Khecarī-section.}\\!  %3.34

%\newpage
%[hp03_035]
\pada{\app{\lem[wit={ceteri}]{kalāṃ}
		\rdg[wit={N23}]{kalā}
		\rdg[wit={J10}]{kālaṃ}} % +J6 sonst wie lemmata
	\app{\lem[wit={P11,V19,K3,N19,V15,V1,Jyo}]{parāṅmukhīṃ}
		\rdg[wit={V3,Gr2,C7,J10}]{parāṅmukhī}
		\rdg[wit={C6}]{avāṅmukhī}}
	\app{\lem[wit={C6,V3,Gr3a,N19,V15,V1,Jyo}]{kṛtvā}
		\rdg[wit={J10}]{kṛtya}
		\rdg[wit={Gr2}]{nītvā}}}
\graus{\pada{\app{\lem[wit={V3,Gr3a,V15,Jyo}]{tripathe}% tṛpathe V19
		\rdg[wit={N19}]{tripathaṃ}
		\rdg[wit={C6}]{tripatha}}
	\app{\lem[wit={V15}]{parivartayet}
		\rdg[wit={C6,V3,V19,N19}]{parivarjayet} % dv<<ā>>ravarjayet C6
		\rdg[wit={K3,C7}]{parivardhayet}
		\rdg[wit={Jyo}]{pariyojayet}}/}\\+  % om. in V1,J10
\pada{\app{\lem[wit={C6,Gr3a,N19,V15,Jyo}]{sā} % sāṃ N19
		\rdg[wit={V3}]{sa}}
	\app{\lem[wit={C6,V3,K3,C7,N19,V15,Jyo}]{bhavet khecarī}
		\rdg[wit={V19}]{bhat ṣecarī}} mudrā}
\pada{vyomacakraṃ tad ucyate/}\\+
\pada{rasanām ūrdhvagāṃ kṛtvā}\myfn{The Pādas in grey scale are not found in \getsiglum{Gr2,V1,J10}, but in \getsiglum{C6,V3,Gr3a,N19,V15,Jyo}.
\getsiglum{N3} omits the whole verse. \getsiglum{J5,G4} have this verse without the grey-scaled part.}}
\pada{kṣaṇārdhaṃ
	\app{\lem[wit={ceteri}]{yadi}
		\rdg[wit={J10,Jyo}]{api}} tiṣṭhati/}\\+
\pada{\app{\lem[wit={V3,J7,N19,V15,V1,J10}]{kṣaṇena}
		\rdg[wit={N23}]{kṣaṇe [ca]}
		\rdg[wit={Gr3a}]{viṣayair} % = YCM; viṣayai V19
		\rdg[wit={P11,Jyo}]{viṣair vi°}
		\rdg[wit={C6}]{duḥkhair vi°}}
		mucyate
	\app{\lem[wit={ceteri}]{yogī}\rdg[wit={J7},alt={\om}]{\skp{\om}}}}
\pada{\app{\lem[wit={ceteri}]{vyādhi}
		\rdg[wit={J7}]{vyādhijanma}}mṛtyujarādibhiḥ//} \NotIn{N3}\\!
		%3.35

%\newpage
%[hp03_036]
\pada{na \app{\lem[wit={ceteri}]{rogo}
		\rdg[wit={V1}]{roga}
		\rdg[wit={J10}]{rogān}}
	maraṇaṃ
	\app{\lem[wit={ceteri}]{tasya}
		\rdg[wit={Jyo}]{tandrā}}}
\pada{na nidrā na \app{\lem[wit={ceteri}]{kṣudhā tṛṣā}% +P11
		\rdg[wit={C7}]{kṣudhā nandaṭ}
		\rdg[wit={C6,V19},post={(tṛkhā \getsiglum{V19})}]{tṛṣā kṣudhā}}/}\\+
\pada{na
	\app{\lem[wit={ceteri}]{ca}
		\rdg[wit={V3}]{bhra}
		\rdg[wit={C7},alt={\om}]{\skp{\om}}} mūrchā
	\app{\lem[wit={ceteri},alt={bhavet}]{bhave}
		\rdg[wit={J10}]{bhave}
		\rdg[wit={C7}]{tu bhavet}}t tasya}
\pada{\app{\lem[wit={ceteri}]{yo mudrāṃ vetti}
	\rdg[wit={N3},alt={\illeg}]{}}
	\app{\lem[wit={J7,V19,K3,N19,V15,V1,J10,Jyo}]{khecarīm}
		\rdg[wit={N3,C6,V3,N23,C7}]{khecarī}}//}\myfn{In \getsiglum{J7} this verse is found after \ref{III38}.}\\!  %3.36


\endverse
%\newpage
\startaltrecension{}\normalsize\color{black}
%[hp03_036_1]
\pada{\app{\lem[wit={N3,P11,Jyo}]{pīḍyate}
		\rdg[wit={C6,V3,J10}]{bādhyate}
		\rdg[wit={V1}]{chādyate}}
	na \app{\lem[wit={V3,V1,J10,Jyo}]{sa}
	\rdg[wit={N3,J5,C6,P11}]{ca}} rogeṇa} % rogena V3
\pada{\app{\lem[wit={C6,V3,V1,J10}]{lipyate na sa}
		\rdg[wit={Jyo}]{lipyate na ca}
		\rdg[wit={J5,P11}]{na ca lipyati}
		\rdg[wit={N3}]{na ca lipyata}} karmaṇā/}\\+
\pada{\app{\lem[wit={N3,P11,V3,V1,J10,Jyo}]{bādhyate}% +P11
	\rdg[wit={C6}]{khādyate}}
	na \app{\lem[wit={C6,V3,V1,J10,Jyo}]{sa}
	\rdg[wit={N3,J5,P11}]{ca}} kālena}
\pada{\app{\lem[wit={N3,C6,P11,V3,J10,Jyo}]{yo mudrāṃ vetti}% mudrā N3
		\rdg[wit={V1}]{yasya mudrāsti}}
	\app{\lem[wit={N3,J10,Jyo}]{khecarīṃ}
		\rdg[wit={C6,P11,V3,V1}]{khecarī}}//}
	\sgwit{N3,C6,V3,V1,J10,Jyo}\\!  %3.37
\endaltrecension
\startverse

%[hp03_037]
\pada{\app{\lem[wit={ceteri}]{cittaṃ}
		\rdg[wit={V19}]{citte}
		\rdg[wit={N3}]{ci\,+}}
	carati khe % carati khe is also illegible in N3
	\app{\lem[wit={ceteri}]{yasmāj}
		\rdg[wit={V3}]{yasyā}
		\rdg[wit={N3}]{+\,.āj}}}
\pada{jihvā carati khe gatā/}\\+ % jihva J8; jihvā jihva V19; gatāḥ C8
\pada{\app{\lem[wit={N3,P11,V15,Jyo}]{tenaiṣā}
		\rdg[wit={V3,N19,V1,J10}]{tenaiva}
		\rdg[wit={C6,J7,Gr3a}]{teneyaṃ}} khecarī
	\app{\lem[wit={N3,P11,V15,Jyo}]{nāma}
		\rdg[wit={ceteri}]{mudrā}}}
\pada{\app{\lem[wit={N3,P11,V15,Jyo}]{mudrā}
		\rdg[wit={C6,V3,J7,Gr3a,N19,V1,J10}]{sarva}}
	siddhair namaskṛtā//}\label{III38} \NotIn{N23}
	\anm{= 4.25*4}\\!  %3.38  % siddhai V19, siddhir C8ac

\newpage
%[hp03_038]
\pada{\app{\lem[wit={ceteri}]{khecaryā}
		\rdg[wit={V3}]{khecaryāṃ}} mudritaṃ yena} % mudritā ye tu C6
\pada{\app{\lem[wit={ceteri}]{vivaraṃ}
		\rdg[wit={K3}]{viviraṃ}
		\rdg[wit={C6,V1}]{vicaran/raṃ}}
	\app{\lem[wit={ceteri}]{lambiko}
		\rdg[wit={K3,C7}]{lampiko}
		\rdg[wit={N3},alt={\illeg}]{}}rdhvataḥ/}\\+
\pada{\app{\lem[wit={C6,V3,N23,N19,V15,V1,J10}]{tasya na}% naḥ V15
		\rdg[wit={N3}]{+\,[s]ya na}
		\rdg[wit={J7,Gr3a,Jyo}]{na tasya}}
		kṣarate binduḥ} % bindu N23
\pada{kāminyā\app{\lem[wit={ceteri}]{śleṣitasya}
		\rdg[wit={Gr2,K3}]{liṅgitasya}
		\rdg[wit={C6}]{liṅgitena}}% saṃślitasya P11
		ca//}\label{III39}\\!  %3.39

% Gr4b have the pātāla-stanza here.

%\newpage
%[hp03_039]
\pada{\app{\lem[wit={ceteri}]{calito}% +P11; patito C6
		\rdg[wit={N23}]{calitā}
		\rdg[wit={V19}]{calate}
		\rdg[wit={N3}]{calato}}'pi yadā binduḥ} % bindu K3, biṃḥ N19
\pada{\app{\lem[wit={V3,Gr2,V19,J10},alt={saṃprāptaś}]{saṃprāpta}
		\rdg[wit={N3,C6,K3,C7,N19,V15,V1,Jyo}]{saṃprāpto}}%
	\app{\lem[wit={Gr2,V19},alt={ca hutāśanaṃ}]{ś ca hutāśanaṃ}% = VM
		\rdg[wit={K3,C7}]{pi hutāśanaṃ}
		\rdg[wit={V3,J10}]{cāgnimaṇḍalaṃ}
		\rdg[wit={C6}]{vahnimaṇḍalaṃ}% +P11,M1,M3(°le)
		\rdg[wit={J5,V15,V1,Jyo}]{yonimaṇḍalam}% +G7? illeg. G4
		\rdg[wit={N3}]{yogimaṃḍalaṃ}
		\rdg[wit={N19}]{yonimaṃgalaṃ}}/}\\+
\pada{\app{\lem[wit={ceteri},alt={vrajaty}]{vraja} % vṛjaṃty N3
		\rdg[wit={N23}]{jajaty}
		\rdg[wit={C7}]{vrajan}}%
	\app{\lem[wit={ceteri},alt={ūrdhvaṃ}]{ty ūrdhvaṃ} % ṃ om. J7
		\rdg[wit={C7}]{pūrvaṃ}
		\rdg[wit={N3}]{ū\,+}}
	\app{\lem[resp=emend,postwit={(C2)}]{hataḥ śaktyā}
		\rdg[wit={Jyo}]{hṛtaḥ śaktyā}
		\rdg[wit={N23}]{hatāchantkā}
		\rdg[wit={C6,V3,J7,N19,V15,V1,J10}]{haṭhāc chaktyā}% havāt N19
		\rdg[wit={J5}]{haṭhāt saktyā}
		\rdg[wit={K3,C7}]{hi tacchaktyā}% the best?
		\rdg[wit={V19}]{hi tadbhuktyā}
		\rdg[wit={N3},alt={\illeg}]{}}}\marma
\pada{\app{\lem[wit={J5,G4,C6,N19,V15,Jyo}]{nibaddho}% +J5,G4
		\rdg[wit={V1}]{nibadhno}
		\rdg[wit={Gr2,K3,C7}]{niruddho}
		\rdg[wit={V3,J10}]{nirodho}
		\rdg[wit={V19}]{viruddhe}
		\rdg[wit={N3},alt={\illeg}]{}}
	\app{\lem[wit={N3,C6,Gr2,Gr3a,N19,V15,V1,Jyo}]{yoni}
		\rdg[wit={P11,V3,J10}]{yoga}}mudrayā//}%
\myfn{\getsiglum{C6} has this verse between \ref{III42}ab and cd.}% P11 has it at a regular place.
\\!  %3.40

\endverse
\startaltrecension{}\normalsize
%[hp03_039_1]
\pada{kapālakuhare jihvā}
\pada{\app{\lem[wit={V3,V15}]{kalā}
		\rdg[wit={Gr2,N19}]{kāla}
		\rdg[wit={N3}]{kālā}
		\rdg[wit={C6}]{kṛtvā}}% +P11
	\app{\lem[wit={N3,C6,V3,V15}]{saṃdhāna}
		\rdg[wit={N19}]{saṃdhāra}
		\rdg[wit={Gr2}]{saṃhāra}}%
	\app{\lem[wit={N3,Gr2,N19,V15}]{mudrayā}
		\rdg[wit={V3}]{varjitā}}/} % varjitāḥ
%		\NotIn{Gr3a,V1,J10,Jyo}
		\sgwit{N3,C6,V3,Gr2,N19,V15}\myfn{%
\getsiglum{V3,J10} have a different order for this hemistich and the following verses: \ref{III43} \rightarrow\ \ref{III40_1} (not in \getsiglum{J10}) \rightarrow\ \ref{III41} \rightarrow\ \ref{III42}.}\label{III40_1}\\!
\endaltrecension
\startverse
% In the 6-chp version it is followed by \devnote{brahmarandhragatā nityaṃ tasya siddhir na dūrataḥ}.


%\newpage
%[hp03_040]
\pada{\app{\lem[wit={ceteri}]{ūrdhva}
		\rdg[wit={N19}]{ūrdhvaṃ}}%
	\app{\lem[wit={J7,V19,C7,Jyo}]{jihvaḥ}
		\rdg[wit={N23}]{jihva}
		\rdg[wit={N3,C6,V3,K3,N19,V15,V1,J10}]{jihvā}}
	\app{\lem[wit={N3,J5,V3,J10}]{sthito}% +J5
		\rdg[wit={C6,P11,Gr2,Gr3a,V15,V1,Jyo}]{sthiro}
		\rdg[wit={N19}]{sito}} bhūtvā}
\pada{somapānaṃ karoti yaḥ/}\\+ % saḥ for yaḥ in C6; karo + + N3
\pada{māsārdhena na saṃdeho} % tu for na in C6; 5 akṣaras illeg. N3
\pada{mṛtyuṃ jayati yogavit//}\label{III41}\\!  %3.41

%\newpage
%[hp03_041]
\pada{nityaṃ % nitya V19
	somakalā\app{\lem[wit={ceteri}]{pūrṇaṃ}
		\rdg[wit={N23,N19}]{pūrṇa}
		\rdg[wit={J10}]{pūrṇe}}}
\pada{śarīraṃ yasya
	\app{\lem[wit={C6,Gr2,Gr3a,N19,Jyo}]{yoginaḥ}
		\rdg[wit={V3}]{yoginaṃ}
		\rdg[wit={V15,V1,J10}]{dehinaḥ}}/}\\+
\pada{takṣakeṇāpi % tatkṣa° N19; °nāpi C7
	\app{\lem[wit={C6,J7,V19,V15,J10,Jyo}]{daṣṭasya}
		\rdg[wit={V3,N23,V1}]{dṛṣṭasya}% +J17
		\rdg[wit={N19}]{daṃṣṭrasya}
		\rdg[wit={K3,C7}]{dagdhasya}}}
\pada{viṣaṃ tasya na
	\app{\lem[wit={ceteri}]{sarpati}
		\rdg[wit={V3}]{sparśati}
		\rdg[wit={N23}]{pīḍyate}}//}\label{III42}
		\NotIn{N3,J5}\\!  %3.42 ; not in N3,J5,C2 but in G4,N24


\newpage
%[hp03_042]
\pada{\app{\lem[wit={ceteri}]{indhanāni}
		\rdg[wit={V3}]{yindhanāni}}
	\app{\lem[wit={ceteri}]{yathā}
		\rdg[wit={K3},alt={\om}]{\skp{\om}}} vahnis}
\pada{\app{\lem[wit={V3,V19,K3,V15,Jyo}]{tailavartiṃ}
		\rdg[wit={N3,C7,N19}]{tailavarti}% vartti N19
		\rdg[wit={C6,Gr2,J10}]{tailavartī}% varttī N23,C6,J10
		\rdg[wit={V1}]{tailāvṛtti}}
	\app{\lem[wit={ceteri}]{ca}
		\rdg[wit={V1}]{va}}
	\app{\lem[wit={ceteri}]{dīpakaḥ}
		\rdg[wit={V1}]{dīpikaḥ}}/}\\+ % V1 vowel sign i added later
\pada{tathā \app{\lem[wit={ceteri}]{soma}
		\rdg[wit={N19}]{sarva}}%
	kalā\app{\lem[wit={ceteri}]{pūrṇaṃ}
		\rdg[wit={J10}]{pūrṇa}
		\rdg[wit={J7,N19}]{pūrṇo}}}
\pada{\app{\lem[wit={ceteri}]{dehī dehaṃ}% dehāṃ V3
		\rdg[wit={C6,V15}]{dehaṃ dehī}
		\rdg[wit={N3},alt={\illeg}]{}} na
	\app{\lem[wit={Gr2,Gr3a,N19,V1,Jyo}]{muñcati}
		\rdg[wit={J10}]{mucyati}
		\rdg[wit={V15}]{muṃcyati}
		\rdg[wit={C6,V3}]{mucyate}
		\rdg[wit={N3}]{+\,+\,ti}}//}\label{III43}\myfn{%
	\getsiglum{Gr2,Gr3a} add here:
	\devnote{rasanāṃ veśayed ūrdhvaṃ pibet tat srāvitaṃ jalam};\\
	\getsiglum{V3} adds here:
	\devnote{tasmād idaṃ prakurvīta nityayuktaḥ samāhitaḥ}.
	}\\!  %3.43

%\endverse\startaltrecension{}
%%[hp03_042_1]
%\pada{\app{\lem[wit={J7,Gr3a}]{rasanāṃ}
%		\rdg[wit={N23}]{rasānāṃ}}
%	\app{\lem[wit={J7,Gr3a},alt={veśayed}]{veśaye}
%		\rdg[wit={N23}]{vasayed}
%		\rdg[wit={K3}]{ūrdhvam ā°}}%
%	\app{\lem[wit={ceteri},alt={ūrdhvaṃ}]{d ūrdhvaṃ}
%		\rdg[wit={K3}]{°veśet}}}
%\pada{pibet tat \app{\lem[wit={Gr3a}]{srāvitaṃ}
%		\rdg[wit={Gr2}]{sravitaṃ}} jalam/} % ta{{sya}}chrāvitaṃ V19
%\sgwit{Gr2,Gr3a}\\!
%
%%[hp03_042_2]
%\pada{tasmād idaṃ prakurvīta}
%\pada{nityayuktaḥ samāhitaḥ/}\label{III43_2} \sgwit{V3}\\!
%\endaltrecension
%\startverse

%\newpage
%[hp03_043]
\pada{\app{\lem[wit={ceteri}]{gomāṃsaṃ}
		\rdg[wit={J7,V19,J10}]{gomāṃsa}}
		bhakṣayen nityaṃ}
\pada{pibe%
	\app{\lem[wit={ceteri},alt={amara}]{d amara}
		\rdg[wit={C7}]{amṛta}}%
	\app{\lem[wit={ceteri}]{vāruṇīm}
		\rdg[wit={V3,N19,V15}]{vāruṇī}}/}\\+    % V3 bhakṣaye
\pada{kulīnaṃ
	\app{\lem[wit={ceteri},alt={tam}]{ta}
		\rdg[wit={J7}]{tum}}m ahaṃ
	\app{\lem[wit={ceteri}]{manye} % ahamanye N23
		\rdg[wit={Jyo}]{manya}
		\rdg[wit={V3}]{vidyāṃ}
		\rdg[wit={J10}]{viṃdyāṃ}}}
\pada{\app{\lem[wit={N3,C6,V3,J10,Jyo}]{itare}
		\rdg[wit={V15,V1}]{tv itare}
		\rdg[wit={N19}]{cetare}
		\rdg[wit={Gr2,Gr3a}]{netarān}}
	\app{\lem[wit={ceteri}]{kulaghātakāḥ}
		\rdg[wit={Gr2,Gr3a}]{kulaghātakān}
		\rdg[wit={N3}]{kuṣṭhaghātakāḥ}}//}\label{III44}\\!  %3.44

%\newpage
%[hp03_044]
\pada{gośabde\app{\lem[wit={ceteri}]{noditā jihvā}
		\rdg[wit={N23}]{nāditā jihvā}
		\rdg[wit={N3},alt={\illeg}]{}}}
\pada{tatpraveśo \app{\lem[wit={ceteri}]{hi}
		\rdg[wit={N23}]{di}}
	\app{\lem[wit={ceteri}]{tāluni}
		\rdg[wit={J10}]{tāluniṃ}}/}\\+
\pada{go\app{\lem[wit={ceteri}]{māṃsa}
		\rdg[wit={N19,V15,V1}]{māṃsaṃ}
		\rdg[wit={N23}]{māsaṃ}}%
	\app{\lem[wit={ceteri}]{bhakṣaṇaṃ}
		\rdg[wit={N3}]{bhakṣaṇe}} 
	\app{\lem[wit={N3,P11,V3,J7,Gr3a,V1,J10,Jyo}]{tat tu}
		\rdg[wit={N23}]{\_\,rttu}
		\rdg[wit={V15}]{tac ca}
		\rdg[wit={N19}]{caitat}
		\rdg[wit={C6}]{hy etan}}}
\pada{mahāpātakanāśanam//}\\!  %3.45


%\newpage
%[hp03_045]
\pada{jihvāpraveśasaṃbhūta}% jihva V19
\pada{\app{\lem[wit={J7,K3,N19,Jyo}]{vahninotpāditaḥ}% =P17
		\rdg[wit={V19},alt={°ditaṃ}]{vahninotpāditaṃ} % V19 om. va of vahni
		\rdg[wit={C6,C7},alt={°ditā}]{vahninotpāditā}
		\rdg[wit={N3},alt={°di\,+}]{vahninotpādi\,+}
		\rdg[wit={V3}]{vahninonnāpitā}
		\rdg[wit={J10}]{vahninottāpito}
		\rdg[wit={N23}]{vahnir utpāditaḥ}}
	\app{\lem[wit={J5,Gr2,Gr3a,N19,Jyo}]{khalu}% =P17;Gr1*
%		\rdg[wit={V17}]{daralu} % stemma point;
%		N26 vahnino[kṣ]āpitodarāt*, N9 vahni[natpi or: nāpi]tādaraṃ
		\rdg[wit={V3,J10}]{daraṃ}
		\rdg[wit={C6}]{surāḥ}
		\rdg[wit={N3},alt={\illeg}]{}}/}\\+
\pada{\app{\lem[wit={C6,V3,Gr2,K3,N19,J10,Jyo}]{candrāt sravati} % śravati N19, snavati? J10
		\rdg[wit={C7}]{candraḥ sravati}
		\rdg[wit={V19}]{candrā dravati}
		\rdg[wit={N3}]{+\,+\,+\,+\,\[t\]i}}
	\app{\lem[wit={J7,V19,K3,J10,Jyo}]{yaḥ sāraḥ}
		\rdg[wit={N23}]{yaḥ sāra}
		\rdg[wit={N3,C6,N19}]{yat sāraṃ}
		\rdg[wit={V3},prewit={(the same hemistich is inserted after \ref{III43})}]{yaḥ sāraṃ tasmād idam [m]akurvīta nityayuktaḥ samāhitaḥ}
		\rdg[wit={C7}]{yaḥ samyak}}}
\pada{\app{\lem[wit={ceteri}]{sā}
		\rdg[wit={K3}]{sa}}
		syā\app{\lem[wit={ceteri},alt={amaravāruṇī}]{d amaravāruṇī}
		\rdg[wit={J10}]{aṃmavāruṇī}}//}
	\NotIn{V1,V15}\myfn{%
In \getsiglum{V1} the second half is added in the margin sec. m.:
\devnote{tasmā[tsa]rati ya[tsā]raṃ sā syād amaravāruṇī}.}\\! % 3.48

\newpage
%[hp03_046]
\pada{\app{\lem[wit={V3,Gr3a,V15,Jyo}]{mūrdhnaḥ}
		\rdg[wit={J10}]{mūrdhneḥ}
		\rdg[wit={J7}]{mūrddhūḥ}
		\rdg[wit={J5,N19}]{mūrddhaṃ}
		\rdg[wit={N3}]{mūrddhvaḥ}
		\rdg[wit={V1}]{mūrddhva}
		\rdg[wit={N23}]{bhūrddhaḥ}
		\rdg[wit={C6}]{ūrdhvaṃ}}
ṣoḍaśa\app{\lem[wit={N3,C6,V3,J7,V19,V1,J10}]{padmapattra}
		\rdg[wit={J5,K3,C7,V15,Jyo}]{pattrapadma}%  ## 
		\rdg[wit={N19}]{patrapatra}
		\rdg[wit={N23},alt={\om}]{\skp{\om}}}galitaṃ % galitaḥ N19
		prāṇād avāptaṃ
	\app{\lem[wit={ceteri}]{haṭhād}
		\rdg[wit={V3}]{haṭhāṃ}}} \\+
\pada{\app{\lem[wit={ceteri}]{ūrdhvāsyo} % J8 °smo?
		\rdg[wit={N23}]{ūrdhvosyo}
		\rdg[wit={C7}]{ūrdhvosya}
		\rdg[wit={V3}]{varddhāsyo}}
	\app{\lem[wit={ceteri}]{rasanāṃ}
		\rdg[wit={N19}]{rasanā}
		\rdg[wit={N23}]{ramanā}}
	\app{\lem[wit={N3,C6,J7,Gr3a,V15,Jyo}]{niyamya}
		\rdg[wit={N23,N19}]{niyasya}
		\rdg[wit={V1}]{ca yāmya}
		\rdg[wit={V3,J10}]{vidhāya}}
	\app{\lem[wit={ceteri}]{vivare}% +P11
		\rdg[wit={N23}]{vicare}
		\rdg[wit={Gr3a}]{vivaraṃ}
		\rdg[wit={C6}]{vidhivat}} % °vac chaktiṃ C6
	\app{\lem[wit={ceteri}]{śaktiṃ}
		\rdg[wit={J7}]{śaktiḥ}} parāṃ % parā J10
	\app{\lem[wit={ceteri}]{cintayet} % citayet N19
		\rdg[wit={N23}]{cintayat}
		\rdg[wit={K3,C7,Jyo}]{cintayan}}/}\\+
\pada{\app{\lem[wit={N3,C6,V3,C7,N19,V15,V1,Jyo}]{utkallola}
		\rdg[wit={J10}]{uttakallola}
		\rdg[wit={J7,V19}]{tatkallola}
		\rdg[wit={K3}]{tatkalola}
		\rdg[wit={N23}]{taptalola}}%
	\app{\lem[wit={N3,C6,Gr2,Gr3a,N19,V15,V1,Jyo}]{kalājalaṃ}
		\rdg[wit={V3,J10}]{jalākulaṃ}}
	\app{\lem[wit={N3,C6,J7,Gr3a,N19,V1,Jyo}]{ca}
		\rdg[wit={V3,J10}]{su}
		\rdg[wit={N23}]{ya}
		\rdg[wit={V15},alt={\om}]{\skp{\om}}}
		vimalaṃ % vimala N3, vimalā C6
	\app{\lem[wit={ceteri}]{dhārāmṛtaṃ}
		\rdg[wit={Jyo}]{dhārāmayaṃ}}\marmas yaḥ pibet}\\+ % vimalā C6
\pada{\app{\lem[wit={ceteri}]{nirdoṣaḥ sa}
		\rdg[wit={V1}]{nirdoṣaṃ sa}
		\rdg[wit={N19}]{nirdoṣo 'sya}
		\rdg[wit={Jyo}]{nirvyādhiḥ sa}}
	mṛṇāla\app{\lem[wit={ceteri}]{komala}
		\rdg[wit={N23}]{komale}}%
	\app{\lem[wit={C6,V3,J7,Gr3a,J10},alt={tanur}]{tanu}% +J5,N24
		\rdg[wit={N23}]{tanu}
		\rdg[wit={N3,P11,N19,V15,V1,Jyo}]{vapur}}r
		yogī ciraṃ jīvati//}% jogī V19, yoga[ṃ] C8ac
	\myfn{\getsiglum{Jyo} has a different verse order from here.}\\!  %3.46

%\newpage
%[hp03_047]
\pada{\app{\lem[wit={ceteri}]{cumbantī}
		\rdg[wit={N23}]{vipitīṃ}}
	yadi \app{\lem[wit={ceteri}]{lambikāgram aniśaṃ jihvā}
		\rdg[wit={K3,C7}]{lampikāgram aniśaṃ jihvā}}
	\app{\lem[wit={ceteri}]{rasa} % also V3mg
		\rdg[wit={V3,J10}]{śiraḥ}}syandinī}\\+
\pada{\app{\lem[wit={ceteri}]{sakṣārā}
		\rdg[wit={N3,V19,N19}]{sākṣārā}
		\rdg[wit={J10}]{sakṣāra}}
	\app{\lem[wit={N3}]{kaṭukātha}% = VM
		\rdg[wit={J7,Gr3a,V15,Jyo}]{kaṭukāmla} % +V3marg
		\rdg[wit={N23}]{vaṭukāmla}
%		\rdg[wit={J8}]{kaṭukāmna} % sakṣīrodakatikta P11
		\rdg[wit={V1}]{kaṭukāsa}
		\rdg[wit={V3,J10}]{kaṭukādya}
		\rdg[wit={N19}]{kaṭutikta}
		\rdg[wit={C6}]{kaṭutyakta}}%
	\app{\lem[wit={ceteri}]{dugdha}
		\rdg[wit={J7}]{dugdhaṃ}
		\rdg[wit={N23}]{du}}%
	\app{\lem[wit={ceteri}]{sadṛśī}
		\rdg[wit={V19}]{sādṛśī}
		\rdg[wit={J7}]{sadṛśīṃ}
		\rdg[wit={N3,V1}]{sadṛśā}
		\rdg[wit={N19,V15}]{lavaṇā}}
	\app{\lem[wit={ceteri}]{madhvājya} % also V3mg
		\rdg[wit={V3,J10}]{madhvādya}
		\rdg[wit={N19}]{vaddhājya}}%
	\app{\lem[wit={ceteri}]{tulyā} % also V3pc, talyā C6
		\rdg[wit={V3}]{tulyāṃ}
		\rdg[wit={J10}]{tulyaṃ}}%
	\app{\lem[wit={J5,Gr2,Gr3a}]{thavā}% +N24
		\rdg[wit={N3,C6,V3,N19,V1,J10,Jyo}]{tathā}
		\rdg[wit={V15}]{savā}}/} \\+
\pada{vyādhīnāṃ haraṇaṃ % vyādhināṃ V1
	\app{\lem[wit={ceteri}]{jarāntakaraṇaṃ} % °karaṇa V15
		\rdg[wit={V19,K3}]{jvarāntakaraṇaṃ}
		\rdg[wit={C7}]{jvarāntaḥkaraṇaṃ}
		\rdg[wit={C6}]{jarāpraśamanaṃ}}
	\app{\lem[wit={C6,V3,J7,V15,J10}]{śāstrāgamodgīraṇaṃ}
		\rdg[wit={V1}]{śastrāṃgamodgīraṇaṃ}
		\rdg[wit={N3,Jyo}]{śāstrāgamodīraṇaṃ}
		\rdg[wit={N23}]{śāstrapramodīraṇaṃ}
		\rdg[wit={Gr3a,N19}]{śāstrāgamoddhāraṇaṃ}}} \\+
\pada{\app{\lem[wit={ceteri},alt={tasya syād}]{tasya syā}
		\rdg[wit={N23}]{tasyād}}d
	a\app{\lem[wit={ceteri},alt={amaratvam}]{maratva}
		\rdg[wit={N23}]{amarakṣam}
		\rdg[wit={V3}]{aramatvam}
		\rdg[wit={Gr3a}]{iha siddhir}}m
	aṣṭa\app{\lem[wit={N3,P11,V1}]{guṇavat}
		\rdg[wit={V15}]{guṇāvat}
		\rdg[wit={C6,V3,Gr2,V19,N19,J10,Jyo}]{guṇitaṃ}
		\rdg[wit={K3,C7}]{guṇitā}}
	\app{\lem[wit={C6,P11,K3,C7,V1,N19,V15,J10,Jyo}]{siddhāṅganā}
		\rdg[wit={N3,V3,N23,V19}]{siddhāṅgaṇā}% +J17
		\rdg[wit={J7}]{siddhāṅgānā}}%
	\app{\lem[wit={ceteri}]{karṣaṇam}
		\rdg[wit={N23}]{karṣaṇā}}//}\\!  %3.47

%\newpage
%[hp03_048]
\pada{\app{\lem[wit={ceteri}]{ekaṃ}
	\rdg[wit={C7}]{eka}
	\rdg[wit={N23}]{evaṃ}}
\app{\lem[wit={ceteri}]{sṛṣṭi} % śṛṣṭi N3
		\rdg[wit={N19}]{dṛṣṭi}}%
	\app{\lem[wit={ceteri}]{mayaṃ} % +P11; maya N23
		\rdg[wit={C6}]{midaṃ}
		\rdg[wit={N19}]{layaṃ}} bījaṃ}
\pada{ekā mudrā \app{\lem[wit={ceteri}]{ca}% eka V15
		\rdg[wit={C7,N19}]{tu}} khecarī/}\\+
\pada{eko
	\app{\lem[wit={ceteri}]{devo}
		\rdg[wit={N23}]{devā}
		\rdg[wit={N3}]{nirā°}}
	\app{\lem[wit={V3,V1,Jyo}]{nirālamba}% +YCM
		\rdg[wit={J7,Gr3a}]{nirālambaś}
		\rdg[wit={N23}]{nirāśambaś}
		\rdg[wit={C6,N19,V15}]{nirālambo}
		\rdg[wit={J10}]{nirālambaṃ}
		\rdg[wit={N3}]{°laṃbo deva}}}
\pada{\app{\lem[wit={N3,C6,V3,N19,V1,J10,Jyo}]{ekā}
		\rdg[wit={Gr3a}]{caikā}
		\rdg[wit={N23}]{cakā}
		\rdg[wit={J7}]{caiṣā}
		\rdg[wit={V15}]{hy ekā}}vasthā manonmanī//}
	\anm{= 4.44*1}\\!  %3.48

\endverse
\newpage
\startaltrecension{}
%[hp03_048_1]
\pada{\app{\lem[wit={Jyo}]{suṣiraṃ}
		\rdg[wit={J10}]{sukhiraṃ}
		\rdg[wit={V3}]{suciraṃ}} jñānajanakaṃ}
\pada{pañca\app{\lem[wit={J10,Jyo}]{srotaḥ}
		\rdg[wit={V3}]{śrotaḥ}}samanvitam/}\\+
\pada{\app{\lem[wit={Jyo}]{tiṣṭhate}
		\rdg[wit={V3}]{tiṣṭhaṃti}
		\rdg[wit={J10}]{tiṣṭhaṃtī}} % tiṣṭhati J10pc
		khecarī mudrā}
\pada{tasmin śūnye nirañjane//}
	\sgwit{V3,J10,Jyo} \anm{= 4.25*2}\\! % 3.52

%\newpage
%[hp03_049] % metre: Mandākrāntā
\pada{\app{\lem[wit={Gr2,N19,V15,V1},alt={pātāle yad}]{pātāle ya}
		\rdg[wit={C6}]{pātālād yad}
		\rdg[wit={V3,Jyo}]{yat prāleyaṃ}% +J17
		\rdg[wit={J10}]{yat prāleya}}%
	\app{\lem[wit={C6,Gr2},alt={viśati}]{d viśati}
		\rdg[wit={V1}]{vita}
		\rdg[wit={V15}]{vitanta}
		\rdg[wit={N19}]{inaya}
		\rdg[wit={V3}]{cāpihita}
		\rdg[wit={J10}]{pihita}
		\rdg[wit={Jyo}]{prahita}}
	\app{\lem[wit={V15,Jyo}]{suṣiraṃ}
		\rdg[wit={C6}]{suśiraṃ}
		\rdg[wit={V3,Gr2,J10}]{sukhiraṃ} % sukhira N23
		\rdg[wit={N19}]{sukhīraṃ}
		\rdg[wit={V1}]{stu[v]imaṃ/me}}
	\app{meru\lem[wit={C6}]{mūle tad asti}
		\rdg[wit={J7}]{mūle yad asti} % merū N23
		\rdg[wit={N19,V15}]{mūle tad asmin}
		\rdg[wit={N23}]{mūle pakṣasti}
		\rdg[wit={V1}]{mūlad}
		\rdg[wit={V3}]{mūrddhyataḥthyaṃ}
		\rdg[wit={J10}]{mūrdhni sthitaṃ}
		\rdg[wit={Jyo}]{mūrdhāntarasthaṃ}}}\\+  %
\pada{\app{\lem[wit={C6}]{tattvaṃ caitat}
		\rdg[wit={Gr2,N19}]{tadvac caitat} % tadvacaitat J7
		\rdg[wit={V15}]{taddac caitat}
		\rdg[wit={V3,J10,Jyo}]{tasmiṃs tattvaṃ}% =P17
		\rdg[wit={V1}]{asmi[ṃ]s tatvaṃ yat}}
	pravadati
	\app{\lem[wit={ceteri},alt={sudhīs}]{sudhī}
		\rdg[wit={N23}]{sudhās}}s
	\app{\lem[wit={ceteri}]{tan mukhaṃ}
	\rdg[wit={C6}]{tat sukhaṃ}}
	\app{\lem[wit={ceteri}]{nimnagānām}
		\rdg[wit={N23}]{niṣagmanāṃ}}/}\\+ % J10 gloss? korthaḥ nāḍīnāṃ
\pada{\app{\lem[wit={ceteri}]{candrāt sāraḥ}
		\rdg[wit={V1}]{candrasāro}
		\rdg[wit={V15}]{candrā sāraḥ}
		\rdg[wit={C6}]{candrāt sāraṃ}
		\rdg[wit={N19}]{candraḥ sāraḥ}}
	\app{\lem[wit={ceteri}]{sravati} % śravati N19,V15
		\rdg[wit={P11}]{grasati}
		\rdg[wit={N23}]{rapati}
		\rdg[wit={V1}]{[sra]vaṃtyai}}\myfn{\getsiglum{V15} jumps to Jālandharabandha (3.67) from here. For the lost part (3.50--66) \getsiglum{J14} is used instead.} % one illegible akṣara before
	\app{\lem[wit={ceteri},alt={vapuṣas}]{vapuṣa}
		\rdg[wit={J10}]{vapuṣes}
		\rdg[wit={V3}]{vapayuṣes}
		\rdg[wit={C6}]{vapuṣā}
		\rdg[wit={V15},alt={\om}]{\skp{\om}}}s tena
	\app{\lem[wit={ceteri},alt={mṛtyur}]{mṛtyu}
		\rdg[wit={V3,J10}]{mṛtyun}
		\rdg[wit={V15},alt={\om}]{\skp{\om}}}r narāṇāṃ}\\+
\pada{\app{\lem[wit={ceteri}]{taṃ}
		\rdg[wit={Jyo}]{tad}}
	\app{\lem[wit={ceteri}]{badhnīyāt}% yāta V3
		\rdg[wit={N23}]{cha\,\_\,yāt}
		\rdg[wit={V15},alt={\om}]{\skp{\om}}}
	\app{\lem[wit={C6}]{sukaraṇamṛdā}
		\rdg[wit={N19}]{pakaraṇamṛdā}
		\rdg[wit={V1}]{kakaraṇam amṛtaṃ}
		\rdg[wit={V3,J10}]{sukaraṇam atho}
		\rdg[wit={Jyo}]{sukaraṇam adho}
		\rdg[wit={J7}]{sukhakaram atho}
		\rdg[wit={N23}]{sukhakaraṇam artho}
		\rdg[wit={V15},alt={\om}]{\skp{\om}}}
	\app{\lem[wit={ceteri}]{nānyathā}
		\rdg[wit={N23}]{nāmarthā}
		\rdg[wit={V15},alt={\om}]{\skp{\om}}}
	\app{\lem[wit={C6,V3,Gr2,N19,J10,Jyo}]{kāya}
		\rdg[wit={V1}]{kārya}
		\rdg[wit={V15},alt={\om}]{\skp{\om}}}siddhiḥ//}\marma
\NotIn{N3,Gr3a}\myfn{%
\getsiglum{Gr4b} has this verse immediately after \ref{III39}.
\getsiglum{N3,Gr3a} have this in Ch. 4 (4.25).
%\devnote{pātālād vā vipati śikhare merumūle tadāstā,
%tatvaṃ caitat pravadati susaṃmukhaṃ nimnaśanāṃ/\\
%caṃdrāt srāvaḥ śravati vapuṣas tena mṛtyur narāṇāṃ,
%taṃ badhnīyāt svakaraṇamṛnā nānyathā kāryasiddhi//}\\
\getsiglum{Gr2} has this in both Ch. 3 and 4.} % but not in N2!
\anm{=\,4.25}\\!

\endaltrecension
\startverse
%\newpage
\outdent
\app{\lem[wit={J7,Gr3a}]{mūlabandhaḥ}
\rdg[wit={N3,C6,V3,N19,J14,V1,J10,Jyo}]{atha mūlabandhaḥ} % bandha V3,N19 ##
\rdg[wit={C7}]{atha mūle bandhaḥ}
\rdg[wit={N23},alt={\om}]{\skp{\om}}}%
	\myfn{\getsiglum{V3,Jyo} have the Mūlabandha section after the Uḍḍiyāna. Cf. 3.6.}//

%[hp03_050]
\pada{\app{\lem[wit={ceteri}]{pārṣṇi}
	\rdg[wit={N23}]{pādima}}bhāgena saṃpīḍya} % saṃpījya N23
\pada{yonim % yonīm V1
\app{\lem[wit={ceteri}]{ākuñcayed}% +J5
	\rdg[wit={N3}]{ākuṃcaned}
	\rdg[wit={N23}]{ākuṃ}}
\app{\lem[wit={ceteri}]{gudam}
	\rdg[wit={V1,J10}]{dṛḍhaṃ}
	\rdg[wit={N23},alt={\om}]{\skp{\om}}}/}\\+
\pada{apānam ūrdhvam ākṛṣya}
\pada{mūlabandho
\app{\lem[wit={ceteri}]{'yam ucyate}% +C7,VM
	\rdg[wit={K3}]{'yam īritaḥ}
	\rdg[wit={C6,V3}]{'yam iṣyate}% ##
	\rdg[wit={N3}]{mayiṣyate} % J5 jumps to the next verse: mūlabandhā(page break)di yoginaḥ.
	\rdg[wit={Jyo}]{'bhidhīyate}}//}\\!

%[hp03_051]
\pada{\app{\lem[wit={GrB,V1,Jyo}]{adhogatim} % better, =HR
	\rdg[wit={N3,Gr2,Gr3a,N19,J14,J10}]{adhogatam}} % lost J5,G4
\app{\lem[wit={N3,C6,Gr2,N19,J10}]{apānaṃ vai}
	\rdg[wit={Jyo}]{apānaṃ vā}% = vai, due to Sandhi
	\rdg[wit={V3}]{apānaṃ ca}
	\rdg[wit={Gr3a,J14}]{apānaṃ tu}
	\rdg[wit={V1}]{apānaivam}}}
\pada{\app{\lem[wit={ceteri}]{ūrdhvagaṃ}
	\rdg[wit={N3}]{mūrddhagaṃ}
	\rdg[wit={V3}]{vidyūrdhagaṃ}} kurute
\app{\lem[wit={N3,C6,V3,J14,V1,J10,Jyo}]{balāt}
	\rdg[wit={Gr2,Gr3a,N19}]{haṭhāt}}/}\\+
\pada{\app{\lem[wit={ceteri}]{ākuñcanena}
	\rdg[wit={J10}]{ākuñcya tena}}
\app{\lem[wit={ceteri}]{taṃ}% +C7,P11
	\rdg[wit={V19}]{ta}
	\rdg[wit={K3}]{te}
	\rdg[wit={C6}]{tu}}
\app{\lem[wit={ceteri}]{prāhur}
	\rdg[wit={N19}]{grāhyaṃ}}} % prāhu V3,V1,J10
\pada{\app{\lem[wit={ceteri}]{mūlabandhaṃ}
	\rdg[wit={J10}]{mūlabandho}}
\app{\lem[wit={C6,P11,Gr2,Gr3a}]{tu}% = source
	\rdg[wit={N3,V3,N19,J14,V1,J10,Jyo}]{hi}} yoginaḥ//}\\!

\newpage
%[hp03_052]
\pada{\app{\lem[wit={ceteri}]{gudaṃ}
	\rdg[wit={N19}]{gulpha}
	\rdg[wit={C6}]{pārṣṇi°}}
\app{\lem[wit={N3,V3,Gr3a,J10,Jyo}]{pārṣṇyā tu} % prob. V19, pāṣṇyā V3
	\rdg[wit={J7}]{pārśnī tu}
	\rdg[wit={N23}]{pādarmyāṃ tu}
	\rdg[wit={N19,V1}]{pārṣṇyā ca}
	\rdg[wit={J14}]{ca pārṣṇinā°} % ca pārṣṇināpīḍya J14
	\rdg[wit={C6}]{°nā gudam}}
	\app{\lem[wit={ceteri}]{saṃpīḍya} % pījya N23, °pīḍye V3
	\rdg[wit={C6,J14}]{āpīḍya}}}
\pada{\app{\lem[wit={N3,C6,V3,Gr2,J14,V1,J10,Jyo},alt={vāyum ā°}]{vāyum ā} % read pāyum?
	\rdg[wit={N19}]{vāyunā}
	\rdg[wit={Gr3a}]{yonim ā°}}kuñcayed
\app{\lem[wit={ceteri}]{balāt}
	\rdg[wit={J7}]{balat}}/}\\+
\pada{vāraṃ vāraṃ % 1st vāra V1; 2nd vāra V3,J10
\app{\lem[wit={N3,C6,V3,N19,J14,V1,J10,Jyo}]{yathā}
	\rdg[wit={Gr2,Gr3a}]{tathā}}
	cordhvaṃ}
\pada{samāyāti samīraṇaḥ//}\\!

%\newpage
%[hp03_053]
\pada{prāṇāpānau % prāṇa° V1
	\app{\lem[wit={ceteri}]{nādabindū}
	\rdg[wit={N3,V3,N19,J10}]{nādabindu}
	\rdg[wit={J14}]{tathā binduḥ}}}
\pada{mūlabandhena
\app{\lem[wit={ceteri}]{caikatām}
	\rdg[wit={C6,N19}]{caikatā}
	\rdg[wit={N23}]{cakataṃ}
	\rdg[wit={V3}]{caikataḥ}}/}\\+
\pada{gatvā yogasya % gatā J14; yagasya J10ac
\app{\lem[wit={N3,C6,V3,J7,V19,K3,V1,Jyo}]{saṃsiddhiṃ}
	\rdg[wit={N23,C7,N19,J14}]{saṃsiddhir}
	\rdg[wit={J10}]{saṃsiddhyaiḥ}}}
\pada{\app{\lem[wit={V3,Jyo}]{yacchato}
	\rdg[wit={C6}]{yakṣyato}
	\rdg[wit={N3}]{yichato}
	\rdg[wit={Gr3a,N19}]{gacchato} % gakṣato V19
	\rdg[wit={J7,J14}]{gacchate}
	\rdg[wit={N23}]{gacchatā}
	\rdg[wit={V1}]{prāpnoty e°}
	\rdg[wit={J10}]{pamāta}}
\app{\lem[wit={ceteri}]{nātra}
	\rdg[wit={V1}]{°va na}
	\rdg[wit={J10}]{tra na}} saṃśayaḥ//}\\! % V19 written in margin pr.m.


%[hp03_054]
\pada{apānaprāṇa%
\app{\lem[wit={ceteri},alt={°yor aikyaṃ}]{yor aikyaṃ} % +C7
	\rdg[wit={N23}]{°yor aikya}
	\rdg[wit={J10}]{°yor aikye}
	\rdg[wit={K3}]{°yoś caikyaṃ}
	\rdg[wit={J14}]{°yor ekyāt}}}
\pada{\app{\lem[wit={ceteri}]{kṣayo}
	\rdg[wit={N23}]{kṣayān}} mūtrapurīṣayoḥ/}\\+
\pada{yuvā bhavati vṛddho'pi}
\pada{satataṃ
	mūla\app{\lem[wit={ceteri}]{bandhanāt}
	\rdg[wit={V19}]{bandhataḥ}}//}\myfn{\getsiglum{N23} adds the following verse here:
	\nagari{bandhamūlaṃ yena tena tena vighnāṃ nivāritaḥ/ ajarāmaratāṃ yāti yathā pañcamukho haraḥ//}}\\!

%\newpage
%[hp03_055]
\pada{\app{\lem[wit={ceteri}]{apāne}
	\rdg[wit={Jyo}]{apāna}
	\rdg[wit={V3,J7}]{apānaṃ}}
\app{\lem[wit={ceteri}]{cordhvage jāte}
	\rdg[wit={V19}]{cordhvage yāte} % or yāne? V19
	\rdg[wit={J10}]{cordhvam āpāte}
	\rdg[wit={Jyo}]{ūrdhvage jāte}}}
\pada{\app{\lem[wit={C6,P11,Gr2,Gr3a,N19}]{saṃprāpte}% = GŚ keep; saṃprāprau P11
	\rdg[wit={V3}]{saṃyāte}
	\rdg[wit={N3,J5,V1,J10,Jyo}]{prayāte} % damaged G4
	\rdg[wit={J14}]{prajāte}}
\app{\lem[wit={N3,V3,Jyo}]{vahnimaṇḍalaṃ}
	\rdg[wit={C6,Gr2,Gr3a,N19,J14,V1}]{vahnimaṇḍale} % °la J7ac?
	\rdg[wit={J10}]{nābhimaṇḍalaṃ}}/}\\+
\pada{\app{\lem[wit={ceteri}]{tadānala}
	\rdg[wit={N19}]{tadānale}
	\rdg[wit={V1}]{tathānale}
	\rdg[wit={C7,J10}]{tathānala}}śikhā dīrghā} % śiṣā V19; dīryā V1
\pada{\app{\lem[wit={N3,C6,V3,Gr2,N19}]{vardhate vāyunāhatā}% hatāḥ C6
	\rdg[wit={Gr3a}]{baṃdhane vāyunāhatā} % bandhena C7
	\rdg[wit={J10}]{kriyate vāyunāhatāḥ}
	\rdg[wit={Jyo}]{jāyate vāyunāhatā}
	\rdg[wit={J14}]{vāyunā vardhate hi sā}
	\rdg[wit={V1}]{vāyunā preritā tathā}}//}\\!


%\newpage
%[hp03_056]
\pada{\app{\lem[wit={ceteri}]{tato}
	\rdg[wit={V1}]{yātā}}
\app{\lem[wit={C6}]{yātau}% yāttau J5
	\rdg[wit={V1,Jyo}]{yāto}% +M1
	\rdg[wit={J10}]{yāte}
	\rdg[wit={N3}]{yāmau}
	\rdg[wit={J7,Gr3a,J14}]{jātau}
	\rdg[wit={N23}]{jātā}
	\rdg[wit={V3}]{jāto}
	\rdg[wit={N19}]{vahnim}}\marmas
\app{\lem[wit={J7,Gr3a,J14,V1,Jyo}]{vahnyapānau}% +P11; °panau J7
	\rdg[wit={N3}]{vahnipānau}% °pātau J5
	\rdg[wit={J10}]{vahniyonau}
	\rdg[wit={C6}]{bāhyapānau}
	\rdg[wit={N23}]{baṃdhapānau}
	\rdg[wit={V3}]{vardhapānai}
	\rdg[wit={N19}]{apānai ca}}}
\pada{\app{\lem[wit={G4,C6,N19,Jyo}]{prāṇam uṣṇa}
	\rdg[wit={V3,J7}]{prāṇam uṣma}
	\rdg[wit={N23}]{prāṇamura}
	\rdg[wit={V19,C7,J14}]{prāṇamukta}
	\rdg[wit={K3}]{prāṇamuktaṃ}
	\rdg[wit={N3,J5,V1,J10}]{prāṇamūla}}
\app{\lem[wit={ceteri}]{svarūpakam}
	\rdg[wit={J10}]{svarūpakaḥ}
	\rdg[wit={J14,V1}]{svarūpakau}
	\rdg[wit={C7}]{svarūpavat}}/}\\+ % °rūpaṃke N23
\pada{\app{\lem[wit={Gr2,Gr3a,V1,Jyo}]{tenātyanta}
	\rdg[wit={N3}]{tenātyantaṃ}
	\rdg[wit={V3}]{tenābhyanta}
	\rdg[wit={J10}]{tenābhyantaḥ}
	\rdg[wit={C6}]{tenāyaṃna}
	\rdg[wit={J14}]{tatotyanta}
	\rdg[wit={N19}]{tailābhyaṃtaḥ}}%
\app{\lem[wit={ceteri}]{pradīptas tu} % °diptas V3
	\rdg[wit={V1}]{pradīpas tu}
	\rdg[wit={N19}]{pradīpāsau}}}
\pada{\app{\lem[wit={ceteri}]{jvalano dehajas tathā} % +P11; jvalanā N23
	\rdg[wit={C6}]{jvalato dehatas tadā}
	\rdg[wit={J10}]{kuto dehakṣayas tadā}}//}\\!

% V1 adds here vāmanāpanamū(?) at the end of the folio. The new folio begins with tena.

\newpage
%[hp03_057]
\pada{tena kuṇḍalinī suptā} % kuṃḍalanī N3
\pada{\app{\lem[wit={N3,C6,V3,Gr2,J14,V1,J10,Jyo}]{saṃtaptā} % sa pra° V3
	\rdg[wit={Gr3a,N19}]{satataṃ}}
\app{\lem[wit={N3,C6,Gr2,J14,J10,Jyo}]{saṃprabudhyate}% ddhy J7, yudhyate N23
	\rdg[wit={V1}]{saṃprabudhyati}
	\rdg[wit={V3}]{sa prabudhyate}
	\rdg[wit={K3}]{sā prabuddhyate}
	\rdg[wit={C7}]{sā prabodhyate}
	\rdg[wit={N19}]{saṃprabodhyate}
	\rdg[wit={V19}]{sānubodhyate}}/}\\+
\pada{\app{\lem[wit={ceteri}]{daṇḍāhatā bhujaṅgīva} % hatya N23
	\rdg[wit={J14}]{yathā daṃḍāhato bhogī}}}
\pada{\app{\lem[wit={N3,P11,Jyo}]{niśvasya} % ,postwit={C6,(=\,M1,M3)}
	\rdg[wit={J14}]{niḥśvasya}
	\rdg[wit={V3,V1,J10}]{viśvasya}
	\rdg[wit={K3}]{niścayam}
	\rdg[wit={C7}]{niścayād}
	\rdg[wit={Gr2,V19,N19}]{niścitam}}\marmas
\app{\lem[wit={ceteri}]{ṛjutāṃ vrajet} % ṛjvatāṃ V3
	\rdg[wit={N3}]{rujutāṃ vṛjet}
	\rdg[wit={J10}]{rijutām iyāt}}//}\\!

%\newpage
%[hp03_058]
\pada{bilaṃ
\app{\lem[wit={N3,Gr2,V19,N19,V1,Jyo}]{praviṣṭeva}
	\rdg[wit={C6}]{praviṣṭe ca}% +P11
	\rdg[wit={V3}]{praviṣṭaṃ ca}
	\rdg[wit={J10}]{praviṣṭaś ca}
	\rdg[wit={J14}]{praviṣṭe pa°}
	\rdg[wit={Gr3a},alt={\om}]{\skp{\om}}}
\app{\lem[wit={ceteri}]{tato}
	\rdg[wit={N23}]{to}
	\rdg[wit={J14}]{°vane}
	\rdg[wit={Gr3a},alt={\om}]{\skp{\om}}}}
\pada{\app{\lem[wit={ceteri}]{brahma}
	\rdg[wit={N23}]{tha\,\_}
	\rdg[wit={Gr3a},alt={\om}]{\skp{\om}}}%
\app{\lem[wit={ceteri}]{nāḍyantaraṃ}
	\rdg[wit={C6}]{nāḍyāntaraṃ}
	\rdg[wit={J10}]{nāḍyantare}
	\rdg[wit={Gr3a},alt={\om}]{\skp{\om}}} vrajet/}
	\lineom{ab}{Gr3a}\\+ %
\pada{\app{\lem[wit={ceteri}]{tasmān}
	\rdg[wit={K3}]{tato}}
\app{\lem[wit={ceteri}]{nityaṃ}
	\rdg[wit={N19}]{nityo}} mūlabandhaḥ}
\pada{kartavyo yogibhiḥ sadā//}\\!

%\newpage
\outdent
\app{\lem[wit={N23}]{atha uḍḍiyānabandhaḥ}
	\rdg[wit={J10}]{atha uḍḍiyānaṃ bandhaḥ}
	\rdg[wit={N19}]{atha uḍḍīyāṇabandhaḥ}
	\rdg[wit={Jyo}]{atha uḍḍīyānabandhaḥ}
	\rdg[wit={K3}]{uḍḍīyānabandhaḥ}
	\rdg[wit={J7}]{uḍḍiyāṇaṃ bandhaḥ}
	\rdg[wit={N3}]{athoḍḍīyāṇaṃ}
	\rdg[wit={V3}]{athoḍḍiyāṇaṃ}
	\rdg[wit={C6}]{athoḍiyānaṃ}
	\rdg[wit={C7,J14}]{athoḍḍīyānabandhaḥ}
	\rdg[wit={V1}]{athoḍyāṇabaṃdhaḥ}
	\rdg[wit={V19},alt={\om}]{\skp{\om}}}/%
%	\myfn{\getsiglum{V3,Jyo} have this section before the Mūlabandha.}


%[hp03_059]
\pada{\app{\lem[wit={N3,C6,V3,J7,V19,K3,Jyo}]{baddho}
	\rdg[wit={C7,N19}]{bandho}
	\rdg[wit={V1,J10}]{ūrdhvo}
	\rdg[wit={J14}]{ūrdhvaṃ}
	\rdg[wit={N23}]{vidrā}}
\app{\lem[wit={ceteri}]{yena suṣumṇāyāṃ} % su<ṣu>mnāyāṃ K3; °yā V3
	\rdg[wit={J10}]{kṣitaḥ suṣumṇāyāḥ}}}
\pada{\app{\lem[wit={ceteri}]{prāṇas}
	\rdg[wit={C6,N19,V1}]{prāṇam}}
\app{\lem[wit={N3,J7,V19,J14,J10,Jyo}]{tūḍḍīyate}
	\rdg[wit={V3}]{tūḍiyate}
	\rdg[wit={N23}]{tudīyate}
	\rdg[wit={K3}]{tūḍūyate}
	\rdg[wit={C7}]{tūyate}
	\rdg[wit={N19}]{uḍḍīyate}
	\rdg[wit={C6}]{uḍiyate}
	\rdg[wit={V1}]{uḍyayate}}
\app{\lem[wit={ceteri}]{yataḥ}
	\rdg[wit={C7}]{punaḥ}}/}\\+
\pada{\app{\lem[wit={ceteri}]{tasmād}
	\rdg[wit={J7}]{tasmātu}% sic
	\rdg[wit={J10}]{tasmāc ca}}
\app{\lem[wit={V19,C7,Jyo},post={\emph{m.c.}}]{uḍḍīyanākhyo}
	\rdg[wit={J7,J10}]{uḍḍiyānākhyo}
	\rdg[wit={V1}]{uḍḍiyāṇākhyo}
	\rdg[wit={N23}]{uddiyānākhyo}
	\rdg[wit={N19}]{uḍḍīyāṇākhyo}
	\rdg[wit={N3}]{uḍḍīyanākhye}
	\rdg[wit={V3}]{uḍiyāṇākhye}
	\rdg[wit={C6}]{uḍiyānākhyaṃ}
	\rdg[wit={K3}]{uḍḍīyamāno}
	\rdg[wit={J14}]{uḍḍīyabandho}}\marmas
\app{\lem[wit={ceteri}]{'yaṃ}
	\rdg[wit={K3}]{sau}
	\rdg[wit={C6}]{tad}
	\rdg[wit={J10},alt={\om}]{\skp{\om}}}}
\pada{yogibhiḥ
\app{\lem[wit={ceteri}]{samudāhṛtaḥ}
	\rdg[wit={C6,V3,N19}]{samudāhṛtaṃ}}//}\\!

%[hp03_060]
\pada{\app{\lem[wit={Gr3a,Jyo}]{uḍḍīnaṃ}
	\rdg[wit={V3}]{uḍīṇaṃ}
	\rdg[wit={C6,J14}]{uḍyānaṃ}
	\rdg[wit={N3,J7}]{uḍyāṇaṃ}
	\rdg[wit={N23,J10}]{uḍḍiyānaṃ}
	\rdg[wit={V1}]{uḍḍiyāṇaṃ}
	\rdg[wit={N19}]{uḍḍīyāṇaṃ}}
\app{\lem[wit={ceteri}]{kurute}
	\rdg[wit={J7}]{kṛyate}
	\rdg[wit={N19}]{kṛte}} yasmād}
\pada{\app{\lem[wit={C6,V3,Gr3a,N19,V1,J10,Jyo}]{aviśrāntaṃ}
	\rdg[wit={N3,P11}]{aviśrāṃta}% +J5
	\rdg[wit={J7,J14}]{aviśrānto}
	\rdg[wit={N23}]{aviśrāntā}} mahākhagaḥ/}\\+ % khaga V3
\pada{\app{\lem[wit={Gr2,J10}]{uḍḍiyānaṃ}
	\rdg[wit={Gr3a,Jyo}]{uḍḍīyānaṃ}
	\rdg[wit={N3,N19,V1}]{uḍḍīyāṇaṃ}
	\rdg[wit={V3,J14}]{uḍiyāṇaṃ}
	\rdg[wit={C6}]{uḍiyānaṃ}} tad
\app{\lem[wit={ceteri}]{eva}
	\rdg[wit={V19}]{evaṃ}
	\rdg[wit={N19}]{evaḥ}} syāt} % syā V3
\pada{\app{\lem[wit={N3,C6,V3,Gr2,N19,J14,V1,Jyo}]{tatra}
	\rdg[wit={J10}]{kṣetra}
	\rdg[wit={Gr3a}]{mūla}}\marmas bandho
\app{\lem[wit={J5,C6,J7}]{vidhīyate}
	\rdg[wit={ceteri}]{'bhidhīyate} % bhi[dh]īyate V19
	\rdg[wit={N23}]{nigadyate}}//}\\!


%\newpage
%[hp03_061]
\pada{\app{\lem[wit={ceteri}]{udare}
	\rdg[wit={J14}]{udaraṃ}
	\rdg[wit={V3}]{udarāt}}
\app{\lem[wit={C6,J7,C7,J14,V1,Jyo}]{paścimaṃ}
	\rdg[wit={N3,N23,J10}]{paścima}
	\rdg[wit={V3,V19,K3,N19}]{paścime}}
\app{\lem[wit={C6,Gr2,V19,C7,V1,J10,Jyo}]{tānaṃ}
	\rdg[wit={N3,K3,N19,J14}]{tāṇaṃ}
	\rdg[wit={V3}]{bhāge}}}\marmas
\pada{nābher % nābhed J7
\app{\lem[wit={ceteri}]{ūrdhvaṃ}\rdg[wit={J10}]{ūrdhve}}
\app{\lem[wit={ceteri}]{ca}
	\rdg[wit={N19,J10}]{tu}} kārayet/}\\+ % °rdhva-akāra° N23
\pada{\app{\lem[wit={Gr2,V19,J10}]{uḍḍiyāno}
	\rdg[wit={K3,C7,Jyo}]{uḍḍīyāno}
	\rdg[wit={V1}]{uḍḍiyāṇo}
	\rdg[wit={N3,N19}]{uḍḍīyāṇo}
	\rdg[wit={C6}]{uḍīyāno}
	\rdg[wit={J14}]{uḍiyāṇo}
	\rdg[wit={V3},alt={\om}]{\skp{\om}}} hy
\app{\lem[wit={N3,C6,Gr2,N19,J14,V1,J10,Jyo}]{asau}
	\rdg[wit={C7}]{asam}
	\rdg[wit={V19,K3}]{ayaṃ}
	\rdg[wit={V3},alt={\om}]{\skp{\om}}} bandho} % baṃdhā N23
\pada{mṛtyumātaṅga\app{\lem[wit={ceteri}]{kesarī}
	\rdg[wit={V3},alt={\om}]{\skp{\om}}}//} \lineom{cd}{V3}\\!

\newpage
%[hp03_062]
\pada{\app{\lem[wit={Gr2,V19,J10}]{uḍḍiyānaṃ} % uddi° N23
	\rdg[wit={V1}]{uḍḍiyāṇaṃ}
	\rdg[wit={K3,C7,Jyo}]{uḍḍīyānaṃ}
	\rdg[wit={N3,N19}]{uḍḍīyāṇaṃ}
	\rdg[wit={C6}]{uḍiyānaṃ}
	\rdg[wit={V3}]{uḍiyāṇaṃ}
	\rdg[wit={J14}]{uḍāyāṇaṃ}} tu
\app{\lem[wit={ceteri}]{sahajaṃ}
	\rdg[wit={J7}]{yaḥ sahate}}}
\pada{\app{\lem[wit={ceteri}]{guruṇā}
	\rdg[wit={V3}]{gurūṇāṃ}} kathitaṃ
\app{\lem[wit={N3,C6,V3,N19,V1,J10,Jyo}]{sadā}
	\rdg[wit={Gr2,Gr3a,J14}]{yathā}}\marma/}\\+
\pada{\app{\lem[wit={ceteri},alt={abhyased/-set}]{abhyased}%
%	\rdg[wit={N3,N19,V1}]{abhyased}
	\rdg[wit={N23}]{abhyāsen}
	\rdg[wit={V3}]{abhyāsāt}
	\rdg[wit={C6}]{abhyāsa°}}
\app{\lem[wit={N3},prewit={\textless\ astatadras tu}]{astatandras tu}% =source,+M3
	\rdg[wit={N19}]{asya taṃtrasya}
	\rdg[wit={J7,Gr3a}]{tad atandras tu}
	\rdg[wit={N23}]{na taṃdras tu}
	\rdg[wit={C6}]{°taḥ svatantras tu}
	\rdg[wit={V1}]{yo hy atandras}
	\rdg[wit={V3,J10,Jyo}]{satataṃ yas tu}
	\rdg[wit={J14}]{aniśaṃ yogī}}\marma} % +M4
\pada{\app{\lem[wit={ceteri}]{vṛddho}
	\rdg[wit={N23}]{vṛddhā}}'pi
\app{\lem[wit={N3,P11,V3,N19,V1,J10}]{taruṇo bhavet}
	\rdg[wit={C6,Gr2,Gr3a,J14,Jyo}]{taruṇāyate}}//}\\!


%[hp03_063]
\pada{\app{\lem[alt={\ante nābher \add},nosep]{}
	\rdg[wit={C6}]{pāṭhāntaram}}%
\app{\lem[wit={ceteri}]{nābher}
	\rdg[wit={J7}]{nābhed}} ūrdhvam
\app{\lem[wit={ceteri}]{adhaś cāpi}
	\rdg[wit={Gr3a}]{adho vāpi}
	\rdg[wit={V1}]{adhaḥkāya}
	\rdg[wit={C6}]{avasthāpya}}}
\pada{\app{\lem[wit={C6,Gr2,Gr3a,J10,Jyo}]{tānaṃ}
	\rdg[wit={N3,P11,V3,N19,J14,V1}]{tāṇaṃ}} % tāpyaṃ
	kuryāt
\app{\lem[wit={ceteri}]{prayatnataḥ}
	\rdg[wit={J10}]{ca yatnataḥ}}/}\\+
\pada{\app{\lem[wit={ceteri}]{ṣaṇmāsam}
	\rdg[wit={V1,J10}]{yogī sam°}}
\app{\lem[wit={N3,P11,Gr2,J10}]{abhyasan} %
	\rdg[wit={V3,K3,C7,N19,J14,V1,Jyo}]{abhyasen}% J5,G4??
	\rdg[wit={V19}]{abhyaseni}
	\rdg[wit={C6}]{ca samabhyān}} mṛtyuṃ} % matyu N23
\pada{\app{\lem[wit={ceteri}]{jayaty eva na saṃśayaḥ}
	\rdg[wit={C6}]{mūlaṃ jayaty asaṃśayaḥ}
	\rdg[wit={J14}]{jayati nātra vicāraṇā}}//}\\!


%\newpage
%[hp03_064]
\pada{sati % sa[t]i V19
\app{\lem[wit={ceteri}]{vajrāsane}
	\rdg[wit={N23}]{vajrāsanau}
	\rdg[wit={N3}]{vajrāsanaṃ}} pādau\marma} % +G7,G11; jānu M3,G5
\pada{\app{\lem[wit={ceteri},post={(dhārayad \getsiglum{J10})}]{karābhyāṃ dhārayed dṛḍham} % °ye<d>dṛ° J7,V19,V3
	\rdg[wit={V1}]{karābhyā dhārayaṃ dṛḍhaṃ}
	\rdg[wit={N3}]{karābhyāṃ kāraye dṛḍhaṃ}
	\rdg[wit={N23}]{karā\,\_\,sandhāraye dṛḍhe}}/}\\+
\pada{gulpha\app{\lem[wit={ceteri}]{deśa}% desa V3
	\rdg[wit={N19}]{deśe}
	\rdg[wit={N3}]{deśaṃ}}%
\app{\lem[wit={N3,C6,V3,Gr2,J14,V1,J10,Jyo}]{samīpe ca}
	\rdg[wit={K3,C7,N19}]{samīpaṃ ca}
	\rdg[wit={V19}]{samīpaṃ tu}}}\marmas
\pada{\app{\lem[wit={ceteri}]{kandaṃ}
	\rdg[wit={V19}]{kaṃdhaṃ}
	\rdg[wit={C7}]{skandaṃ}
	\rdg[wit={J14}]{gudaṃ}}
\app{\lem[wit={C6,Gr2,Gr3a,N19,J14,Jyo}]{tatra}
	\rdg[wit={V3,J10}]{tacca}
	\rdg[wit={N3}]{tava}
	\rdg[wit={V1}]{tasya}}
\app{\lem[wit={C6,Gr2,Gr3a,N19,J14,Jyo}]{prapīḍayet}
	\rdg[wit={N3,V3,V1,J10}]{prapīḍyate}}//}%
	\myfn{In \getsiglum{Jyo} this verse appears much later as 3.114 in the printed edition.}\\!


%[hp03_065]
\pada{\app{\lem[wit={ceteri}]{paścimaṃ tānam}
	\rdg[wit={P11,J14,V1}]{paścimaṃ tāṇam}
	\rdg[wit={N3,V3}]{paścimatāṇam}}
\app{\lem[wit={N3,C6,V3,J7,Gr3a,J14}]{udare}
	\rdg[wit={N23}]{udara}
	\rdg[wit={N19}]{udaraṃ}
	\rdg[wit={V1,J10}]{upari}}}
\pada{\app{\lem[wit={ceteri}]{kārayed}
	\rdg[wit={J10}]{pīḍayed}}
\app{\lem[wit={N3,C6,V3,J7,V1,J10}]{dhṛdaye †gale†} % kāraye hṛdaye V3, kārayed-hṛ°? V1
	\rdg[wit={N23}]{dhṛdaye gataiḥ}
	\rdg[wit={V19}]{udare hṛdi}
	\rdg[wit={C7,J14}]{cibukaṃ hṛdi}
	\rdg[wit={K3}]{cibukaṃ hṛdā}
	\rdg[wit={N19}]{vṛddhidaṃ śanaiḥ}}\marma/}\\+
\pada{śanaiḥ \app{\lem[wit={ceteri}]{śanair yathā}
	\rdg[wit={J14}]{śanair yataḥ}
	\rdg[wit={N23},alt={\om}]{\skp{\om}}}
\app{\lem[wit={N3,V3,Gr3a,V1}]{prāṇas}
	\rdg[wit={J14}]{prāṇaḥ}
	\rdg[wit={Gr2}]{prāṇās}
	\rdg[wit={C6,N19}]{prāṇaṃ}
	\rdg[wit={J10}]{prāṇo}}} % prāṇā/os J7
\pada{\app{\lem[wit={N3,C6,J7,V19,K3,N19}]{tunda}
	\rdg[wit={V3,V1}]{tuda}
	\rdg[wit={N23}]{taṃda}
	\rdg[wit={C7}]{tadā}
	\rdg[wit={J10}]{nāḍī}
	\rdg[wit={J14}]{skaṃda}}%
\app{\lem[wit={N3,P11,Gr2,V19,C7,J14,V1}]{saṃdhiṃ} % +C7
	\rdg[wit={V3,N19,J10}]{saṃdhi}
	\rdg[wit={K3}]{siṃddhiṃ}
	\rdg[wit={C6}]{siddhiṃ}}
\app{\lem[wit={N3,C6,V3,Gr2,N19,V1}]{na}
	\rdg[wit={Gr3a}]{ca}
	\rdg[wit={J10}]{ni°}
	\rdg[wit={J14}]{nir°}} gacchati\marma//} \NotIn{Jyo}\\!

%\newpage
%[hp03_066]
\pada{sarveṣām eva bandhānām} % caiva C6
\pada{\app{\lem[wit={ceteri}]{uttamo}
	\rdg[wit={J14}]{uttamas}
	\rdg[wit={N19}]{uttamaṃ}}
\app{\lem[wit={Gr2,Gr3a,J10,Jyo}]{hy uḍḍiyānakaḥ}
	\rdg[wit={V1}]{hy uḍḍiyāṇakaḥ}
	\rdg[wit={N3}]{hy uḍḍīyāṇakaḥ}
	\rdg[wit={N19}]{hy uḍḍīyāṇakaṃ}
	\rdg[wit={C6}]{hy uḍiyānakaḥ}
	\rdg[wit={V3}]{hy uḍiyāṇakaḥ}
	\rdg[wit={J14}]{tūḍiyānakaḥ}}/}\\+
\pada{\app{\lem[wit={Gr2,V19,J10,Jyo}]{uḍḍiyāne}
	\rdg[wit={K3,C7}]{uḍḍīyāne}
	\rdg[wit={N3,N19}]{uḍḍīyāṇe}
	\rdg[wit={V1}]{uḍḍiyāṇe}
	\rdg[wit={C6}]{uḍiyāne}
	\rdg[wit={V3,J14}]{uḍiyāṇe}}
\app{\lem[wit={ceteri}]{dṛḍhe}
	\rdg[wit={Gr2,Gr3a}]{kṛte}}
	\app{\lem[wit={ceteri}]{bandhe}
	\rdg[wit={C6}]{baddhe}}} % ##? 3 x e -> a N23
\pada{\app{\lem[wit={N3,P11,V3,N19,J10,Jyo}]{muktiḥ}
	\rdg[wit={V1}]{muktiṃ}
	\rdg[wit={C6,Gr2,Gr3a,J14}]{mūlaṃ}}
\app{\lem[wit={N3,P11,V3,N19,Jyo}]{svābhāvikī}
	\rdg[wit={J10}]{svābhāvakī}
	\rdg[wit={C6,Gr3a,J14,V1}]{svābhāvikaṃ}
	\rdg[wit={J7}]{svabhāvikaṃ}
	\rdg[wit={N23}]{bhāvikaṃ}} bhavet//}\\!


\newpage
\outdent
\app{\lem[wit={C6,C7,J14,V1,J10,Jyo}]{atha jālandharabandhaḥ}
	\rdg[wit={J7,K3}]{jālandharabandhaḥ}
	\rdg[wit={N23}]{atha nāśaṃdharabandhaḥ}
	\rdg[wit={N19}]{atha jālaṃdharībaṃdhaḥ}
	\rdg[wit={N3}]{atha jālāṃdharaḥ}
	\rdg[wit={V3}]{atha jālaṃdharaṃ}
	\rdg[wit={V19},alt={\om}]{\skp{\om}}}/

% V15 resumes with hṛdaye in Pada a. Ca. 18 verses are omitted.
%[hp03_067]
\pada{kaṇṭham ākuñcya \myfn{\getsiglum{V15} resumes here.}hṛdaye} % ākuṃci V3
\pada{sthāpayec % °ye Gr3a,N3,V3,J10, °yet* N19, °yed V15
\app{\lem[wit={C6,J7,Jyo}]{cibukaṃ dṛḍham}
	\rdg[wit={P11,V3,Gr3a,V1,J10}]{dṛḍham icchayā}% +M1,J5,G4 ##
	\rdg[wit={N3}]{dṛḍham īchayā}
	\rdg[wit={N19}]{dṛḍham icchayet}
	\rdg[wit={V15}]{dṛḍhaniścayāt}
	\rdg[wit={N23},alt={\om}]{\skp{\om}}}/}\\+
\pada{bandho jālandharākhyo'yaṃ} % baṃdha J10; jālāṃdharā° V3; °ākṣo N19
\pada{\app{\lem[wit={V15}]{amṛtāvyayakārakaḥ}% Marmasthāna
	\rdg[wit={P11,V3,N19}]{amṛtavyayakārakaḥ} % kāraka V3
	\rdg[wit={N3}]{amṛtāvapakārakaḥ}
	\rdg[wit={V19}]{amṛtākṣayakārakaḥ}
	\rdg[wit={C7}]{amṛtakṣayakārakaḥ}
	\rdg[wit={K3}]{amṛtākṣarakārakaḥ}
	\rdg[wit={Gr2},post={(mṛtaḥ \getsiglum{N23})}]{mṛtyor mṛtyuḥ paro mataḥ}
	\rdg[wit={C6}]{mṛtyumātaṃgakesarī}
	\rdg[wit={V1,J10,Jyo}]{jarāmṛtyuvināśakaḥ}}//}\label{Jaala1}\\!


%[hp03_068]
\pada{\app{\lem[wit={N3,C6,V3,N19,V15,V1,J10,Jyo}]{badhnāti hi} % baddhāti N19
	\rdg[wit={N23}]{badhnāti ha}
	\rdg[wit={Gr3a}]{badhnātīha}
	\rdg[wit={J7}]{badhnātīhṛ}}
\app{\lem[wit={C6,J7,V19,J10,Jyo},post={(sirā \getsiglum{Jyo})}]{śirā}
	\rdg[wit={V3,N23,K3,C7,N19,V15,V1}]{śiro}
	\rdg[wit={N3}]{śilā}}%
\app{\lem[wit={ceteri}]{jālam}
	\rdg[wit={V3}]{jālāṃ}}}
\pada{\app{\lem[wit={N3,C6,J7,Gr3a,N19,V15,J10,Jyo}]{adhogāmi}
	\rdg[wit={N23}]{adhogāmī}
	\rdg[wit={V3}]{madhyegāmi}
	\rdg[wit={V1}]{nādhāyāti}} nabhojalam/}\\+
\pada{tato jālandharo bandhaḥ} % jālāṃ° N3; °dharā V15; baṃdha V3, baṃdho C6
\pada{\app{\lem[wit={N3,C6,V3,J7,N19,V15,Jyo}]{kaṇṭha}
	\rdg[wit={N23,Gr3a,V1,J10}]{kaṇṭhe}}%
	\app{\lem[wit={ceteri}]{duḥkhaugha}}nāśanaḥ//}\\! % nāśanaṃ V3

%\newpage
%[hp03_069]
\pada{jālandhare kṛte bandhe} % jālādhare V19, jālāṃdhare N3,V3, °dhara? V15
\pada{kaṇṭhasaṃkocalakṣaṇe/}\\+
\pada{na pīyūṣaṃ
\app{\lem[wit={ceteri}]{pataty}
	\rdg[wit={V19}]{prayāty}
	\rdg[wit={N23}]{kṣaraty}} agnau}
\pada{na ca vāyuḥ
\app{\lem[wit={N3,C6,V3,Gr2,N19,V15,J10}]{pradhāvati}
	\rdg[wit={Gr3a,V1,Jyo}]{prakupyati}}//}\\!


%[hp03_070]
\pada{kaṇṭha\app{\lem[wit={ceteri}]{saṃkocanenaiva}
	\rdg[wit={V1}]{saṃkocane dehe}}}
\pada{\app{\lem[wit={N3,P11,J7,Gr3a,V15,J10,Jyo}]{dve nāḍyau}% ddhe J7ac, ddhau J7pc
	\rdg[wit={N19},postwit={\getsiglum{J7}\postcorr}]{dvau nāḍyau}
	\rdg[wit={C6}]{dvināḍyau}
	\rdg[wit={N23}]{\_\,nā\,\_}
	\rdg[wit={V1}]{nāḍyau ca}
	\rdg[wit={V3},alt={\om}]{\skp{\om}}}
\app{\lem[wit={N3,C6,V15,V1,J10,Jyo}]{stambhayed}% °bhaye N3,J10
	\rdg[wit={Gr2,Gr3a,N19}]{stambhite}
	\rdg[wit={V3},alt={\om}]{\skp{\om}}}
\app{\lem[wit={N3,V1,J10,Jyo}]{dṛḍhaṃ}
	\rdg[wit={P11,Gr2,V19,C7,N19}]{dhruvam}
	\rdg[wit={K3}]{dhruve}
	\rdg[wit={V15}]{dhṛvaṃ}
	\rdg[wit={C6}]{dhuram}
	\rdg[wit={V3},alt={\om}]{\skp{\om}}}/}\\+
\pada{\app{\lem[wit={N3,C6,J7,Gr3a,N19,V15,Jyo}]{madhyacakram}
	\rdg[wit={N23}]{madhyakram}
	\rdg[wit={V3}]{madhye cakram}
	\rdg[wit={J10}]{madhyaṃ cakram}
	\rdg[wit={V1},alt={\om}]{\skp{\om}}} idaṃ
\app{\lem[wit={ceteri}]{jñeyaṃ}
	\rdg[wit={N23}]{ya}
	\rdg[wit={V1},alt={\om}]{\skp{\om}}}}
\pada{ṣoḍaśādhārabandhanam//} \lineom{cd}{V1}\\!


%\newpage
%[hp03_071]
\pada{bandhatrayam idaṃ śreṣṭhaṃ}
\pada{\app{\lem[wit={N3,J7,N19}]{mahāsiddhair} % siddhai N19
	\rdg[wit={V1,J10,Jyo}]{mahāsiddhaiś}
	\rdg[wit={N23}]{mahāsiddhe}
	\rdg[wit={C6,V15}]{mahāsiddhi}
	\rdg[wit={V3}]{mahāsīha}}
\app{\lem[wit={N3,C6,V3,Gr2,Gr3a,N19}]{niṣevitam}% niśe° N3
	\rdg[wit={V1,J10,Jyo}]{ca sevitaṃ}
	\rdg[wit={V15}]{pradāyakaṃ}}/}\\+
\pada{sarveṣāṃ
\app{\lem[wit={N3,C6,V3,J7,Jyo}]{haṭha}% +J5
	\rdg[wit={N23,N19,V15,V1,J10}]{yoga} % yoma N19
	}tantrāṇāṃ}
\pada{\app{\lem[wit={ceteri}]{sādhanaṃ}
	\rdg[wit={N23}]{sāranaṃ}} yogino viduḥ//}
	\NotIn{Gr3a}\myfn{In \getsiglum{Jyo} this verse is found after \ref{III74}.}\\!

\newpage
%[hp03_072]
\startgray
\pada{adhastāt
\app{\lem[wit={V1,J10}]{kuñcanenāśu}
	\rdg[wit={Gr2}]{kuñcanenaiva}}}
\pada{kaṇṭha\app{\lem[wit={V1,J10}]{saṃkocane kṛte}
	\rdg[wit={Gr2}]{saṃkocanena ca}}/}\\+
\pada{\app{\lem[wit={V1}]{madhye}
	\rdg[wit={Gr2,J10}]{madhya}} paścimatānena}
\pada{syāt prāṇo brahmanāḍigaḥ//}
\sgwit{Gr2,V1,J10}\myfn{\getsiglum{N3,C6,V3,N19,V15,Jyo} have this verse in chp. 2.
\getsiglum{Gr2,V1,J10} have this in both chapters.
\getsiglum{Gr3a} does not have it at all.} \anm{= 2.46}\\!
\endgray

%\newpage
%[hp03_073]
\pada{mūlasthānaṃ
\app{\lem[wit={N3,C6,V3,V15,V1,Jyo}]{samākuñcya}
	\rdg[wit={Gr2,N19}]{samākṛṣya}
	\rdg[wit={Gr3a,J10},alt={\om}]{\skp{\om}}}}
\pada{\app{\lem[wit={Gr2,V15,Jyo}]{uḍḍiyānaṃ}
	\rdg[wit={V1}]{uḍḍiyāṇaṃ}
	\rdg[wit={N3,N19}]{uḍḍīyāṇaṃ}
	\rdg[wit={C6}]{uḍiyānaṃ}
	\rdg[wit={V3}]{uḍiyāṇaṃ}
	\rdg[wit={Gr3a,J10},alt={\om}]{\skp{\om}}} tu kārayet/} \lineom{ab}{Gr3a,J10}\\+
\pada{\app{\lem[wit={V3,J7,Gr3a,V15,Jyo}]{iḍāṃ ca piṅgalāṃ} % piṃgulāṃ V3
	\rdg[wit={N3,C6,N23,N19}]{iḍā ca piṅgalā}
	\rdg[wit={V1}]{iḍāpiṃgalāṃ}
	\rdg[wit={J10},alt={\om}]{\skp{\om}}}
\app{\lem[wit={ceteri}]{baddhvā}
	\rdg[wit={N19}]{baddhā}
	\rdg[wit={J10},alt={\om}]{\skp{\om}}}}
\pada{vāhayet
\app{\lem[wit={C6,Gr2,Gr3a,N19,V1}]{paścimaṃ}
	\rdg[wit={V3}]{paścimāṃ}
	\rdg[wit={N3,V15}]{paścimā}
	\rdg[wit={J10},alt={\om}]{\skp{\om}}
	\rdg[wit={Jyo}]{paścime}}
\app{\lem[wit={ceteri}]{patham}
	\rdg[wit={Jyo}]{pathi}
	\rdg[wit={J10},alt={\om}]{\skp{\om}}}//} \lineom{cd}{J10}\\!


%[hp03_074]
\pada{\app{\lem[wit={ceteri}]{anenaiva vidhānena}
	\rdg[wit={J10}]{brahmasthānasthito rodhaḥ}}}
\pada{\app{\lem[wit={Gr2,V15,V1,J10,Jyo}]{prayāti}
	\rdg[wit={N3,C6,V3,V19,C7}]{sevayet}% ##
	\rdg[wit={K3}]{[s]e[vay]e[t]}
	\rdg[wit={N19}]{vaśayet}}
\app{\lem[wit={V3,J7,V15,V1,J10,Jyo}]{pavano layam}
	\rdg[wit={N23}]{pavano lagaṃ}
	\rdg[wit={N3,C6,K3,N19}]{pavanālayam}% ##
	\rdg[wit={C7}]{pavanānalam}
	\rdg[wit={V19}]{paścimānalaṃ}}/}\\+
\pada{tato na jāyate
\app{\lem[wit={N3,C6,V3,J7,C7,N19,V15,V1,J10,Jyo}]{mṛtyur}
	\rdg[wit={V19,K3}]{mṛtyu}
	\rdg[wit={N23}]{mṛtyuṃ}}}
\pada{\app{\lem[wit={Gr2,K3,C7,N19,V1,Jyo}]{jarārogādikaṃ}% N24
	\rdg[wit={N3}]{jarārogādikas}
	\rdg[wit={C6,V3}]{jarārogādi kā}% +J5 kā vyathā ##
	\rdg[wit={V15}]{jarāmohādikaṃ}
	\rdg[wit={V19}]{jvaro rogādikas}
	\rdg[wit={J10}]{nāsya jarādikaṃ}}
\app{\lem[wit={ceteri}]{tathā}% +N24
	\rdg[wit={N3}]{tadā} % vyathā J5, damaged G4
	\rdg[wit={C6,V3}]{kathā}}//}\label{III74}\\!


\outdent
\app{\lem[wit={N3,C6,V3,N23,C7,N19,V15,V1,J10}]{atha}
	\rdg[wit={J7,Gr3a,Jyo},alt={\om}]{\skp{\om}}}
\app{\lem[wit={C6,V3,Gr2,K3,V15,V1,J10}]{viparītakaraṇī}
	\rdg[wit={N3}]{viparītakaraṇīṃ}
	\rdg[wit={N19}]{viparītakaraṇaṃ}
	\rdg[wit={C7}]{viparītakam}
	\rdg[wit={V19,Jyo},alt={\om}]{\skp{\om}}}/

%[hp03_075]
\grayscale
\pada{yat kiñcit
\app{\lem[wit={C6,V3,J7,Gr3a,N19,Jyo}]{sravate} % śravate N19
	\rdg[wit={N23}]{sravanaṃ}
	\rdg[wit={N3,V15,V1,J10},alt={\om}]{\skp{\om}}}
\app{\lem[wit={C6,J7,Gr3a,Jyo}]{candrād}
	\rdg[wit={V3}]{candra}
	\rdg[wit={N19}]{caṃdrāṃn}
	\rdg[wit={N23}]{ceda<<m>>}
	\rdg[wit={N3,V15,V1,J10},alt={\om}]{\skp{\om}}}}
\pada{amṛtaṃ
\app{\lem[wit={Gr2,Gr3a}]{divyarūpi ca}
	\rdg[wit={N19}]{divyarūpiṇaṃ}
	\rdg[wit={C6,Jyo}]{divyarūpiṇaḥ}% =N3
	\rdg[wit={V3}]{divyarūpagaḥ}
	\rdg[wit={N3,V15,V1,J10},alt={\om}]{\skp{\om}}}/}\\+
\pada{tat sarvaṃ grasate
\app{\lem[wit={C6,V3,Gr2,Gr3a,Jyo}]{sūryas}
	\rdg[wit={N19}]{roho}
	\rdg[wit={N3,V15,V1,J10},alt={\om}]{\skp{\om}}}}
\pada{tena \app{\lem[wit={C6,V3,J7,Gr3a,N19}]{piṇḍaṃ}
	\rdg[wit={N23}]{piḍaṃ}
	\rdg[wit={Jyo}]{piṇḍo}
	\rdg[wit={N3,V15,V1,J10},alt={\om}]{\skp{\om}}}
\app{\lem[wit={Gr2,K3,C7,N19}]{vināśi ca}
	\rdg[wit={V19}]{vinasyati}
	\rdg[wit={C6,V3}]{jarāyutaṃ}% = N3
	\rdg[wit={Jyo}]{jarāyutaḥ}
	\rdg[wit={N3,V15,V1,J10},alt={\om}]{\skp{\om}}}//}
	\NotIn{N3,V15,V1,J10}\myfn{%
	\getsiglum{Gr1r} has this pair of verses in Ch. 4: 
	\devnote{% J5 similar, G4 damaged
	yat kiṃcit sravate candrād amṛtaṃ divyarūpiṇaḥ/
	tat sarvaṃ grasate sūryas tena piṇḍaṃ jarāyutaṃ//
	tatrāsti karaṇaṃ divyaṃ sūryasya paribandhanaṃ/
	gurūpadeśato jñeyaṃ na tu śāstrārthakoṭibhiḥ//}
	}\\!
%	\anm{= 4.25*0}

%%% G7 omits these two verses too. G11,M3 have them at the end of this section. !!!

%\newpage
%[hp03_076]
\pada{\app{\lem[wit={C6,V3,Gr2,V19,C7,N19,Jyo}]{tatrāsti}
	\rdg[wit={K3}]{tato sti}
	\rdg[wit={N3,V15,V1,J10},alt={\om}]{\skp{\om}}} karaṇaṃ divyaṃ} % divya C7
\pada{sūryasya mukha%
\app{\lem[wit={ceteri}]{bandhanam} % entry in Marmasthāna
	\rdg[wit={Jyo}]{vañcanam}
	\rdg[wit={N3,V15,V1,J10},alt={\om}]{\skp{\om}}}\marma/}\\+
\pada{gurūpadeśato % guropa° N19
\app{\lem[wit={C6,V3,J7,Gr3a,N19,Jyo}]{jñeyaṃ}
	\rdg[wit={N23}]{\_\,yaṃ}
	\rdg[wit={N3,V15,V1,J10},alt={\om}]{\skp{\om}}}}
\pada{\app{\lem[wit={C6,V3,J7,Gr3a,N19,Jyo}]{na tu}
	\rdg[wit={N23}]{rttu}
	\rdg[wit={N3,V15,V1,J10},alt={\om}]{\skp{\om}}}
śāstrārthakoṭibhiḥ//} \NotIn{N3,V15,V1,J10}\\!%  \anm{= 4.25*0}
\endgray

\newpage
%[hp03_077]
\pada{\app{\lem[wit={ceteri}]{ūrdhvaṃ nābhir}
	\rdg[wit={N23}]{ūrdhvanābhor}
	\rdg[wit={Jyo}]{ūrdhvanābher}
	\rdg[wit={C6}]{ūrdhvaṃ nābher}}
\app{\lem[wit={C6,V3,V19,N19,V15,J10}]{adhas tālur}% adhaḥstālur V19, °tālūr V15
	\rdg[wit={N3,J7,K3,V1}]{adhas tālu}
	\rdg[wit={C7}]{adhas tālum}
	\rdg[wit={N23}]{asāluktar}
	\rdg[wit={Jyo}]{adhas tālor}}}
\pada{\app{\lem[wit={ceteri}]{ūrdhvaṃ}
	\rdg[wit={N23,V19}]{ūrdhva}
	\rdg[wit={V1}]{ūrdhvo}}
	bhānur adhaḥ śaśī/}\\+ % adho C6; śaśi V15
\pada{\app{\lem[wit={N3,Gr2,C7,V1,Jyo}]{karaṇī viparītākhyā}% +C7
	\rdg[wit={V3,V15}]{karaṇaṃ viparītākhyaṃ}
	\rdg[wit={C6,V19,K3,N19,J10},alt={\om}]{\skp{\om}}}}
\pada{guruvākyena
\app{\lem[wit={N3,V3,N23,V15,V1,Jyo}]{labhyate}
	\rdg[wit={J7,C7}]{gamyate}
	\rdg[wit={C6,V19,K3,N19,J10},alt={\om}]{\skp{\om}}}//} \lineom{cd}{C6,V19,K3,N19,J10}\\!


%\newpage
%[hp03_078]
\pada{karaṇī \app{\lem[wit={ceteri}]{viparītākhyā}
	\rdg[wit={C6}]{viparītākhyaṃ}
	\rdg[wit={N19}]{viparītākṣaṃ}
	\rdg[wit={C7,Jyo},alt={\om}]{\skp{\om}}}}
\pada{sarvavyādhi\app{\lem[wit={ceteri}]{vināśinī}
	\rdg[wit={N19}]{vināśanī}
	\rdg[wit={C6,V3}]{vināśanaṃ}
	\rdg[wit={C7,Jyo},alt={\om}]{\skp{\om}}}/} \lineom{ab}{C7,Jyo}\\+
\pada{nityam abhyāsayuktasya}
\pada{jaṭharāgni\app{\lem[wit={N3,J7,N19,V15,V1}]{vivardhanī}
	\rdg[wit={N23,Gr3a,J10,Jyo}]{vivardhinī}
	\rdg[wit={C6,V3}]{vivardhanaṃ}}//}\\!


%[hp03_079]
\pada{āhāro bahulas tasya} % bahu<la>s C7
\pada{saṃpādyaḥ sādhakasya % °pādya N3
\app{\lem[wit={ceteri}]{tu}
	\rdg[wit={N23,K3,Jyo}]{ca}}/}\\+
\pada{\app{\lem[wit={N3,C6,V3,V1,J10,Jyo}]{alpāhāro}
	\rdg[wit={Gr2,V19,C7,N19,V15}]{anāhāro}
	\rdg[wit={K3}]{anāhāre}}
\app{\lem[wit={ceteri}]{yadi bhaved} % bhuved? C7
	\rdg[wit={J10}]{nirāhāraḥ}}}
\pada{\app{\lem[wit={N3,V3,Gr3a,V15}]{agnir dehaṃ}
	\rdg[wit={P11,Gr2,N19}]{agnidehaṃ}
	\rdg[wit={V1}]{deham agnir}
	\rdg[wit={C6}]{agnidāho}
	\rdg[wit={Jyo}]{agnir daha°}
	\rdg[wit={J10}]{kṣudhālasya}}
\app{\lem[wit={ceteri}]{dahet} % dahe V3,N19
	\rdg[wit={P11,V15}]{haret}
	\rdg[wit={C6}]{bhavet}
	\rdg[wit={Jyo}]{°ti tat}
	\rdg[wit={J10}]{vaśe}}
\app{\lem[wit={ceteri}]{kṣaṇāt}
	\rdg[wit={N3}]{kramāt}% =HR; ##?
	\rdg[wit={J7}]{tataḥ}
	\rdg[wit={J10}]{bhavet}}//}\\!


%[hp03_080]
\pada{adhaḥ\app{\lem[wit={C6,Jyo}]{śirāś cordhva}
	\rdg[wit={N3,V3,V19,N19,V15,V1,J10}]{śiraś cordhva} % corddhaṃ V3, cordha V19
	\rdg[wit={J7,K3}]{śirā ūrdhva} % arddha J10; śiro J7pc, śirāḥ K3
	\rdg[wit={N23}]{śīrā ūrdhva}
	\rdg[wit={C7}]{śira ūrdhva}}%
\app{\lem[wit={N3,C6,Gr2,Gr3a,V15,Jyo}]{pādaḥ}
	\rdg[wit={V3,N19}]{pāda}
	\rdg[wit={V1,J10}]{pādau}}}
\pada{\app{\lem[wit={ceteri}]{kṣaṇaṃ syāt} % kṣaṇa N19,N3, kṣaṇaḥ C6
	\rdg[wit={V19}]{kṣīṇaṃ syāt}
	\rdg[wit={J10}]{lakṣaṇaṃ}} prathame dine/}\\+
\pada{\app{\lem[wit={N3,C6,V3,V15,V1,J10,Jyo}]{kṣaṇāc ca}
	\rdg[wit={Gr2}]{kṣaṇāt tu}
	\rdg[wit={C7,N19}]{kṣaṇārdhaṃ} % ṃ om. N19
	\rdg[wit={K3}]{kṣaṇārdhe}
	\rdg[wit={V19},alt={\om}]{\skp{\om}}} kiṃcid
\app{\lem[wit={ceteri}]{adhikam}
	\rdg[wit={N23}]{apika}
	\rdg[wit={V19},alt={\om}]{\skp{\om}}}} % kṣaṇārtu N23, apika N23 >> Śāradā?
\pada{\app{\lem[wit={ceteri}]{abhyasec ca}
	\rdg[wit={J7}]{abhyasetva}
	\rdg[wit={N23}]{bhyarccayec ca}
	\rdg[wit={V19},alt={\om}]{\skp{\om}}} dine dine//} \lineom{cd}{V19}\\!


%[hp03_081]
\pada{\app{\lem[wit={N3,C6,V3,V1}]{valiś ca}
	\rdg[wit={N23,Gr3a,N19,V15,J10,Jyo}]{valitaṃ}
	\rdg[wit={J7}]{calitaṃ}}
	\app{\lem[wit={ceteri}]{palitaṃ}
	\rdg[wit={C6}]{palitaś}} caiva}
\marma\pada{\app{\lem[wit={Jyo}]{ṣaṇmāsordhvaṃ na}% Marmasthāna
	\rdg[wit={N3,C6,V15}]{ṣaṇmāsārdhān na}% +M1,G5,6
	\rdg[wit={V3}]{ṣaṇmāsārdhaṃ na}
	\rdg[wit={Gr2,Gr3a}]{ṣaṇmāsārdhena} % +J5
	\rdg[wit={N19}]{ṣaṇmāsārdhe ca}
	\rdg[wit={V1,J10}]{ṣaṇmāsāt tu na}}
\app{\lem[wit={N3,C6,V3,V15,V1,J10,Jyo}]{dṛśyate}% +M1,G5,6
	\rdg[wit={Gr2,Gr3a,N19}]{naśyati}}/}\\+
\pada{\app{\lem[wit={ceteri}]{yāmamātraṃ tu}% jāma V19; mātras tu N19
	\rdg[wit={V15}]{yāmamātraṃ ca}
	\rdg[wit={C7}]{yo māsatraya}
	\rdg[wit={J10}]{māsatrayaṃ tu}} yo
\app{\lem[wit={ceteri}]{nityam}
	\rdg[wit={N23}]{gnibhyam}}}
\pada{\app{\lem[wit={ceteri}]{abhyaset}
	\rdg[wit={V19}]{aset}} sa
\app{\lem[wit={ceteri}]{tu}
	\rdg[wit={J7}]{su}
	\rdg[wit={N19}]{ca}}
\app{\lem[wit={ceteri}]{kālajit}
	\rdg[wit={N19}]{kālavit}}//}\\!

\outdent
\grau{
atratyā vajrolī 
\app{\lem[wit={V19}]{granthānte likhitā}% liṣitā V19
	\rdg[wit={K3}]{granthāntare likhitā vartate}
	\rdg[wit={C7}]{granthāntare tu likhitāsīt}}/
\app{\lem[wit={K3,C7}]{kramaprāptāpy atra tyaktā} % oṃ kramaprāptyāpy  C7
	\rdg[wit={V19},alt={\om}]{}}/
\app{\lem[wit={K3}]{asādhāraṇa}
	\rdg[wit={V19}]{asādhāraṇaṃ}
	\rdg[wit={C7}]{asāraṇa}}prāṇy\-anuṣṭheyatvāt tasyāḥ/ % prāṇuṣṭheyatvāt C7
	\sgwit{Gr3a}%
	\myfn{In \getsiglum{Gr3a} the Vajrolī section is found at the end of the work.}
}

\newpage
\outdent
\app{\lem[wit={ceteri}]{atha vajrolī}
		\rdg[wit={C6,J10},alt={read after \emph{vinā} of the next line}]{}}/

%[hp03_082]
\pada{svecchayā vartamāno'pi}
\pada{\app{\lem[wit={C6,V3,J7,V19,V1,J10,Jyo}]{yogoktair}
		\rdg[wit={N23}]{yogokair}
		\rdg[wit={N19}]{yogoktar}
		\rdg[wit={V15}]{yogokta}% ## = DYŚ
		\rdg[wit={N3}]{yogoktaṃ}
		\rdg[wit={C7}]{niyamair}}
	\app{\lem[wit={ceteri}]{niyamair vinā} % V3 niyamai
		\rdg[wit={C7}]{vividhais tathā}}/}\\+
\pada{\app{\lem[wit={C6,V19,V15,V1,J10,Jyo}]{vajrolīṃ yo}
		\rdg[wit={V3,Gr2,C7,N19}]{vajrolī yo}
		\rdg[wit={N3}]{vajrālī yo}} % yo (l.br.) yo J10
	\app{\lem[wit={ceteri}]{vijānāti}% +K4
		\rdg[wit={Gr2}]{bhijānāti}}}
\pada{sa yogī \app{\lem[wit={ceteri}]{siddhibhājanaṃ}
		\rdg[wit={N23},alt={°bhājanaḥ}]{siddhibhājanaḥ}
		\rdg[wit={J10}]{siddhimān bhavet}}\marma//}\\!

%[hp03_083]
\pada{tatra % tava N3
	\app{\lem[wit={ceteri}]{vastu}
		\rdg[wit={N3}]{castu}
		\rdg[wit={N19}]{bheda}}dvayaṃ
	\app{\lem[wit={ceteri},alt={vakṣ(y)e}]{vakṣye}%
%		\rdg[wit={V3,V19}]{vakṣe}
		\rdg[wit={J7}]{manye}
		\rdg[wit={N23}]{api}}} % dvayam api N23
\pada{durlabhaṃ yasya kasya
	\app{\lem[wit={ceteri}]{cit}
		\rdg[wit={V15}]{tu}}/}\\+
\pada{kṣīraṃ \app{\lem[wit={N3,P11,V3,V19,C7,N19,V15,V1,Jyo}]{caikaṃ}% vaikaṃ C7
		\rdg[wit={J10}]{caiva}
		\rdg[wit={C6,Gr2}]{ekaṃ}}
		dvitīyaṃ tu} % °tiyaṃ V3
\pada{nārī  % nāḍī C6
	\app{\lem[wit={ceteri}]{ca}
		\rdg[wit={C7}]{tu}} vaśavartinī//}\\! % vaśi°? N19


%[hp03_084]
\pada{\app{\lem[wit={N3,C6,J7,N19,V15,Jyo}]{mehanena}
		\rdg[wit={N23}]{mehanaiva}
		\rdg[wit={V19}]{mohanena}
		\rdg[wit={C7}]{mohanenā}
		\rdg[wit={V3}]{meḍhrenena}
		\rdg[wit={V1},post={\unm}]{meḍhreṇa}
		\rdg[wit={J10}]{mahānibhaṃ}}} % mehanena J10pc1, meḍhra J10pc2
	\app{\lem[wit={ceteri}]{śanaiḥ}
		\rdg[wit={V19}]{sadā}} samyag % kasyag V15
\pada{\app{\lem[wit={N3,C6,V3,N23,V19,C7,N19,V15,Jyo},alt={ūrdhvākuñcanam}]{ūrdhvākuñcana} % ā cancelled in N4?
		\rdg[wit={J7}]{ūrdhva kiṃcanam}
		\rdg[wit={J10}]{kṛtvā kuñcanam}
		\rdg[wit={V1}]{gudākuñcanam}}m abhyaset/}\\+
\pada{puruṣo \app{\lem[wit={ceteri}]{vāpi nārī vā}% vāti N19
		\rdg[wit={C7}]{vāpi vā nārī}
		\rdg[wit={Jyo}]{'py atha vā nārī}}}
\pada{\app{\lem[wit={N3,C6,V3,N23,C7,N19,V15,J10,Jyo}]{vajrolī} % °li V15
		\rdg[wit={V19,V1}]{vajrolīṃ}
		\rdg[wit={J7}]{vajrolīḥ}}%
	\app{\lem[wit={ceteri}]{siddhim āpnuyāt}
		\rdg[wit={J7}]{siddhibhājanam}
		\rdg[wit={N23}]{siddhibhājanaḥ}}//}\\!  %3.81

% om. N23
%[hp03_085]
\pada{\app{\lem[wit={N3,P11,V3,V19,N19,V15,Jyo}]{yatnataḥ} % yantrataḥ C7
		\rdg[wit={J7,V1,J10}]{prayatnataḥ}
		\rdg[wit={C6}]{prayatnāt}} % + J10pc
	\app{\lem[wit={N3,C6,V3,V19,C7}]{śaranālena}% sara° C6
		\rdg[wit={N19}]{śalanolena}
		\rdg[wit={V15}]{śatanārīṇāṃ}
		\rdg[wit={Jyo}]{śastanālena}
		\rdg[wit={J7,V1,J10}]{śironāle}}} % + J10pc
\pada{\app{\lem[wit={N3,C7,N19,V1,Jyo}]{phūtkāraṃ} % + J10pc
		\rdg[wit={V3}]{phutkāraṃ}
		\rdg[wit={V19,V15}]{pūtkāraṃ}
		\rdg[wit={J7,J10}]{phūtkāraḥ}
		\rdg[wit={C6}]{sphūtkāraṃ}}
	\app{\lem[wit={ceteri}]{vajra} % + J10pc
		\rdg[wit={J7,J10}]{kaṃbu}}%
	\app{\lem[wit={C6,V3,N19,V15,Jyo}]{kandare}
		\rdg[wit={N3,J7,V19,C7,V1,J10}]{kandhare}}/}\\+
\pada{śanaiḥ \app{\lem[wit={ceteri}]{śanaiḥ}
		\rdg[wit={J10}]{śanaḥ}}
	\app{\lem[wit={ceteri}]{prakurvīta}% +P11
		\rdg[wit={C6,J10}]{prakurvaṃti}}}
\pada{vāyusaṃcārakāraṇāt//} \NotIn{N23}\\!  % vāyu{{ḥ}} C7; vāyuma° J10ac, vāyoḥ C6


%[hp03_086]
\pada{\app{\lem[wit={C6,Gr2,V19,C7,N19,V15}]{nāryā}
		\rdg[wit={Jyo}]{nārī}
		\rdg[wit={N3}]{māryā}
		\rdg[wit={V3,V1,J10}]{bhāryā}}
	\app{\lem[wit={ceteri}]{bhage}
		\rdg[wit={N3}]{bhāge}}
	\app{\lem[wit={ceteri},alt={patad}]{pata} % patat V1
		\rdg[wit={J7}]{pated}
		\rdg[wit={N19}]{ca tad}}%
	\app{\lem[wit={N3,C6,Gr2,V19,C7,N19,V15,Jyo},alt={bindum}]{dbindu} % biṃduṃm N19
		\rdg[wit={V3}]{bindhuḥm}
		\rdg[wit={V1,J10}]{bindur}}}%
\pada{m abhyāsenordhvam
	\app{\lem[wit={ceteri}]{āharet}
		\rdg[wit={C7}]{āruhet}}/}\\+
\pada{calitaṃ
	\app{\lem[wit={N3,J5}]{ca svakaṃ}
		\rdg[wit={C6,P11,Gr2,N19}]{tu svakaṃ}% +G4
		\rdg[wit={V3}]{tu sukaṃ}
		\rdg[wit={V15,Jyo}]{ca nijaṃ}
		\rdg[wit={V1}]{patitaṃ}
		\rdg[wit={J10}]{calitaṃ}
		\rdg[wit={V19,C7},alt={\om}]{\skp{\om}}} bindum}
\pada{\app{\lem[wit={ceteri}]{ūrdhvam ākṛṣya rakṣayet}
		\rdg[wit={N3}]{ūrdhvam ākṛ\,+\,+\,+\,+}
		\rdg[wit={V15}]{ūrdhvam āhṛtya rakṣayet}
		\rdg[wit={N19}]{abhyāsenordhvam āharet}
		\rdg[wit={V19,C7},alt={\om}]{\skp{\om}}}//}
	\lineom{cd}{Gr3a}\\!

\newpage
%[hp03_087]
\pada{evaṃ
	\app{\lem[wit={V1,J10}]{rakṣati yo}
		\rdg[wit={J5,C6,V3,Gr2,N19}]{tu rakṣayed} % ## +Gr1
		\rdg[wit={V19,C7,Jyo}]{saṃrakṣayed}
		\rdg[wit={V15}]{surakṣayed}}
		binduṃ} % biṃdu V3,N4
\pada{mṛtyuṃ jayati yogavit/}
	\anm{\ref{VuIII88}--\ref{VuIII116}a lost \getsiglum{N3}}\\+
\pada{maraṇaṃ \app{\lem[wit={ceteri}]{bindu}
		\rdg[wit={N19}]{vida}
		\rdg[wit={V19},alt={\om}]{\skp{\om}}}pātena}
\pada{\app{\lem[wit={ceteri}]{jīvitaṃ} % jīvituṃ P11
		\rdg[wit={C6,J7,Jyo}]{jīvanaṃ}
		\rdg[wit={N23}]{jī}
		\rdg[wit={V19},alt={\om}]{\skp{\om}}}
	\app{\lem[wit={ceteri}]{bindudhāraṇāt}
		\rdg[wit={V15}]{bindurakṣaṇāt}
		\rdg[wit={N23,V19},alt={\om}]{\skp{\om}}}//}\label{VuIII88}
	\lineom{cd}{V19}\\!

%[hp03_088]
\pada{\app{\lem[resp=emend]{sugandhir}% +L2
		\rdg[wit={J5,GrB,Gr2,V19,V15}]{sugandhi}% +J5, yugaṃndhi N23
		\rdg[wit={N19,Jyo}]{sugandho}
		\rdg[wit={C7,V1,J10},alt={\om}]{\skp{\om}}} yogino
	\app{\lem[wit={C6,Gr2,N19}]{deho}
		\rdg[wit={V19,V15,Jyo}]{dehe}
		\rdg[wit={P11,V3}]{dehaṃ}% +J5
		\rdg[wit={C7,V1,J10},alt={\om}]{\skp{\om}}}}
\pada{jāyate bindu\app{\lem[wit={V3,N23,J7,V19,N19,Jyo}]{dhāraṇāt}% +P11
		\rdg[wit={C6,V15}]{rakṣaṇāt}
		\rdg[wit={C7,V1,J10},alt={\om}]{\skp{\om}}}/}\myfn{\getsiglum{V15} has this hemistich after the first half of the next verse.}
		\lineom{ab}{C7,V1,J10}\\+ % Haplography?
%		\sgwit{GrB,J7,V19,N19,V15,Jyo}
\pada{yāva\app{\lem[wit={N23,C7,J10,Jyo},alt={binduḥ}]{d binduḥ}
		\rdg[wit={GrB,J7,V19,N19,V15,V1}]{bindu}}
	\app{\lem[wit={J5,Gr2,V19,N19,V1,J10,Jyo}]{sthiro}% +J5, kṣīro G4?
		\rdg[wit={GrB,C7,V15}]{sthito}}
	\app{\lem[wit={ceteri}]{dehe}
		\rdg[wit={Gr2}]{deho}}}
\pada{tāva\app{\lem[wit={GrB,V19,C7,V1,J10},alt={mṛtyubhayaṃ kutaḥ}]{n mṛtyubhayaṃ kutaḥ}% +G4,P11
		\rdg[wit={Gr2,N19,Jyo}]{kālabhayaṃ kutaḥ}% +J5
		\rdg[wit={V15}]{jīvanam ucyate}}//}\\!

%[hp03_089]
\pada{\app{\lem[wit={ceteri}]{cittāyattaṃ}
		\rdg[wit={N23}]{cittamattaṃ}
		\rdg[wit={J5}]{manomayaṃ}
		\rdg[wit={C6,V3}]{manodhīnaṃ}}
		nṛṇāṃ % bhavet C6
	\app{\lem[wit={ceteri}]{śukraṃ}
		\rdg[wit={V3}]{śuklaṃ}}}
\pada{\app{\lem[wit={ceteri}]{śukrāyattaṃ}
		\rdg[wit={V3}]{śuklāyataṃ}
		\rdg[wit={C6}]{śukrādhīnaṃ}}
	\app{\lem[wit={V3,N19,V1,J10}]{hi}% +P11
		\rdg[wit={J5,C6,Gr2,V19}]{tu}% +J5,N24
		\rdg[wit={C7,V15,Jyo}]{ca}}
	\app{\lem[wit={ceteri}]{jīvitam} % +N23,P11
		\rdg[wit={C6,J7}]{jīvanaṃ}}/}\\+
\pada{tasmāc chukraṃ
	\app{\lem[wit={ceteri},alt={manaś}]{mana} % + J10pc
		\rdg[wit={J10}]{rajaś}
		\rdg[wit={C7}]{rakṣa°}}% + J7pc
	\app{\lem[wit={ceteri},alt={caiva}]{ś caiva}
		\rdg[wit={V1}]{caivaṃ}
		\rdg[wit={C7}]{°ṇīyaṃ}}}
\pada{\app{\lem[wit={ceteri}]{rakṣaṇīyaṃ}% °yaḥ C6
	\rdg[wit={C7}]{yogibhiś ca}} prayatnataḥ//}\\!


%[hp03_090]
\pada{\app{\lem[wit={C6,V3,Gr2,N19,V15,Jyo}]{ṛtumatyā} % rutu° N23,V3
		\rdg[wit={V19,C7,V1,J10}]{bindumadhye}}
	\app{rajo\lem[wit={C6,Gr2,V19,C7,N19,V15,V1,Jyo}]{'py evaṃ} % + J10pc
		\rdg[wit={V3}]{thevaṃ}
		\rdg[wit={J10}]{py eva}}}
\pada{\app{\lem[wit={Gr2}]{striyā}
		\rdg[wit={V19,N19,V15,V1,J10}]{bījaṃ}% ##
		\rdg[wit={J5}]{vīryaṃ}
		\rdg[wit={C7}]{jīvaṃ}
		\rdg[wit={Jyo}]{nijaṃ}
		\rdg[wit={V3}]{jayaṃ}
		\rdg[wit={C6}]{biṃduṃ}}\marmas
	\app{\lem[wit={ceteri}]{binduṃ}
		\rdg[wit={V3,J10}]{bindu}
		\rdg[wit={C6}]{rakṣe}}
	\app{\lem[wit={ceteri}]{ca}
		\rdg[wit={C6,N19,V1}]{tu}
		\rdg[wit={C7}]{pra°}}
	\app{\lem[wit={ceteri}]{rakṣayet}
		\rdg[wit={V3}]{rakṣayan}
		\rdg[wit={V19}]{taṃnnayet}
		\rdg[wit={C7}]{°pālayet}
		\rdg[wit={C6}]{yogavit}}/}\\+
\pada{\app{\lem[wit={ceteri}]{meḍhreṇā} % meṃḍhre° N19
		\rdg[wit={V19,C7,V15}]{meḍhreṇa}
		\rdg[wit={N23}]{meḍhrā}
		\rdg[wit={J10}]{meḍhrām ā}}% $$ J10 is partly based on Gr2?
	\app{\lem[wit={ceteri},alt={karṣayed}]{karṣaye}
		\rdg[wit={V3}]{karṣayad}
		\rdg[wit={J10}]{kuṃcayed}}d ūrdhvaṃ}
\pada{samyagabhyāsa%
	\app{\lem[wit={Gr2,V19,C7,V1}]{yogataḥ}
		\rdg[wit={P11,V3,N19,V15,G4}]{yogavān}% ##
		\rdg[wit={J10,Jyo}]{yogavit}
		\rdg[wit={J5,C6}]{pāṭavāt}}//}\\!

\startgray
%[hp03_091]
\pada{ayaṃ yogaḥ puṇyavatāṃ}
\pada{\app{\lem[wit={ceteri}]{dhanyānāṃ}
		\rdg[wit={Jyo}]{dhīrāṇāṃ}}
	tattva\app{\lem[wit={C6,J7,V19,C7,V15,V1}]{śālinām} % tattva in V19 unclear
		\rdg[wit={V3,N19}]{śālinaṃ}
		\rdg[wit={N23}]{sattināṃ}
		\rdg[wit={J10,Jyo}]{darśinām}}/} \lineom{ab}{Gr1r}\\+
\pada{nirmatsarāṇāṃ
	\app{\lem[wit={P11,V3,N23,V19,N19,V15,V1}]{sidhyeta} % sidhyaita N23
		\rdg[wit={J7}]{siddheta}
		\rdg[wit={Jyo},post={(but sidhyeta in mss?)}]{vai sidhyen}
		\rdg[wit={J10}]{siddhet}
		\rdg[wit={C6}]{siddhānāṃ}}}
\pada{na tu matsara\app{\lem[wit={C6,V3,Gr2,V19,V15,V1,Jyo}]{śālinām}
		\rdg[wit={N19}]{śālinaṃ}
		\rdg[wit={J10}]{śīlinām}}//} \lineom{cd}{Gr1r,C7}\myfn{%
		In \getsiglum{V15} Pāda b and d are transposed; \getsiglum{Jyo} has this verse at the end of the Sahajolī section.}\\!
\endgray

\newpage
\outdent
\graus{\app{\lem[wit={J7,J10}]{atha sahajolī}% Jyo-mss
	\rdg[wit={Jyo}]{atha sahajoliḥ}}/
	\sgwit{J7,J10,Jyo}}


%[hp03_092]
\pada{\app{\lem[wit={C6,V19,C7,V1,J10}]{sahajolī}
		\rdg[wit={V3,Gr2,N19,V15,Jyo}]{sahajoliś}} % Jyo-ed
%		\rdg[wit={N4}]{sahajolīś}
	\app{\lem[wit={C6,V19,C7}]{cāmarolī}
%		\rdg[wit={N4}]{cāmarolīr}
		\rdg[wit={V3,N19}]{cāmaroli}
		\rdg[wit={V15,Jyo}]{cāmarolir}
		\rdg[wit={J10}]{vāmarolī}
		\rdg[wit={V1}]{cāmarolī ca}
		\rdg[wit={Gr2}]{cāmaroliś ca}}}
\pada{\app{\lem[wit={ceteri}]{vajrolyā} % vajrāḷyā V15
		\rdg[wit={V19,C7}]{vajrolyante}
		\rdg[wit={C6}]{vajrolī}}
	\app{\lem[wit={C6,V3,Gr2,N19,V15,V1}]{eva bhedataḥ}
		\rdg[wit={J10}]{ekabhedataḥ} % eka -> nāma J10pc
		\rdg[wit={Jyo}]{bheda ekataḥ}
		\rdg[wit={V19}]{prakīrtitā}
		\rdg[wit={C7}]{pracodyate}}/}\\+
\grau{\pada{\app{\lem[wit={J7,V19,N19,V15,V1,J10},alt={jaleṣu bhasma}]{jaleṣu\marmas bhasma}
		\rdg[wit={Jyo}]{jale subhasma}
		\rdg[wit={C7}]{jale bhasmani}}
	\app{\lem[wit={V19,C7,N19,V15,V1,J10,Jyo}]{nikṣipya}
		\rdg[wit={J7}]{niḥkṣipya}}}
\pada{\app{\lem[wit={J7,V19,C7,V15,V1,J10,Jyo}]{dagdha}
		\rdg[wit={N19}]{dagdhaṃ}}gomaya%
	\app{\lem[wit={J7,V19,N19,V15,V1,J10,Jyo}]{sambhavaṃ}
		\rdg[wit={C7}]{sambhave}}//} 
		\lineom{cd}{Gr1r,C6,V3,N23}}\\!


%[hp03_093]
\pada{\app{\lem[wit={ceteri}]{vajrolīmaithunād}
		\rdg[wit={V15}]{vajroḷimithunād}} ūrdhvaṃ}
\pada{strī\app{\lem[wit={J7,N19,V1,J10,Jyo}]{puṃsoḥ}
		\rdg[wit={V3}]{puṃso}
		\rdg[wit={N23}]{puṃsā}
		\rdg[wit={V15}]{puṃsau}
		\rdg[wit={C6,V19,C7}]{puṃsoś}}
	\app{\lem[wit={Gr2,N19,V15,V1,J10,Jyo}]{svāṅga}
		\rdg[wit={V3}]{svāṃgu}
		\rdg[wit={C6,V19,C7}]{cāṃga}}lepanam/}\\+  %lāpanaṃ C6
\pada{\app{\lem[wit={ceteri}]{āsīnayoḥ} % °yot N19
		\rdg[wit={V15}]{anenaiva}}
	\app{\lem[wit={ceteri}]{sukhenaiva}
		\rdg[wit={J10}]{mukhenaiva}}}
\pada{mukta%
	\app{\lem[wit={C6,Gr2,C7,V15,V1,Jyo}]{vyāpārayoḥ}
		\rdg[wit={N19,J10}]{vyāpārayo}
		\rdg[wit={V3}]{vyāpāramo}
		\rdg[wit={V19}]{vyāpārala°}}
	\app{\lem[wit={Gr2,Jyo}]{kṣaṇāt}
		\rdg[wit={C6,V3,V19,C7,N19,V15,V1,J10}]{kṣaṇaṃ}}//}\\!% ##

%[hp03_094]
\pada{\app{\lem[wit={V3,N23,N19,V15,Jyo},alt={sahajolir}]{sahajoli}
		\rdg[wit={C6,J7,V19,C7,V1,J10}]{sahajolī}% sahayolīr V19; °jolīr P11
		}r iyaṃ proktā}
\pada{\app{\lem[wit={P11,V3,Jyo}]{śraddheyā}% = DYŚ
		\rdg[wit={Gr1r,C6,V19,C7,V1}]{śraddhayā}% +M1,M3,G7
		\rdg[wit={J10}]{sādhyeyā}
		\rdg[wit={V15}]{siddhaye}
		\rdg[wit={Gr2,N19}]{sevyate}} yogibhiḥ sadā/}\\+
\grau{\pada{ayaṃ śubhakaro yogo} % From hier 2.5 verses om. in N23
\pada{\app{\lem[wit={V3,J7,V15,J10}]{bhoge}
		\rdg[wit={C6,N19,V1,Jyo}]{bhoga}
		\rdg[wit={C7}]{yoga}
		\rdg[wit={V19},alt={\gap}]{}}
	\app{\lem[wit={V3,J7,V15,J10}]{bhukte}
		\rdg[wit={Jyo}]{yukto}
		\rdg[wit={N19}]{mukte}
		\rdg[wit={V19,C7,V1}]{mukti}
		\rdg[wit={C6}]{yoge}}%
	\app{\lem[wit={C6,V3,J7,N19,V15,J10,Jyo}]{'pi muktidaḥ}
		\rdg[wit={C7,V1}]{vimuktidaḥ}
		\rdg[wit={V19}]{pradāyakaḥ}}//}\label{III94}\\+
	\lineom{cd}{Gr1r,N23} \anm{cf. \ref{III101}cd}}\\!  % M1 omits too, but M3 has it.

%\newpage
\outdent
\graus{\app{\lem[wit={J7,J10}]{atha amarolī}
	\rdg[wit={Jyo}]{athāmarolī}
	\rdg[wit={V15}]{āthamāroḷi}
	\rdg[wit={V19,C7}]{tatrāmarolī}}/ \sgwit{J7,Gr3a,V15,J10,Jyo}}


%[hp03_095]
\pada{\app{\lem[wit={V3,V19,C7,V15,V1,Jyo}]{pittolbaṇatvāt} % °lvana° V19
		\rdg[wit={C6}]{pītvā aṇut}
		\rdg[wit={N19}]{virttaṇatvāḍyat}
		\rdg[wit={J10}]{vihāya nityāṃ}
		\rdg[wit={J7}]{vihāya nīv\,..\,ḥ}}
	\app{\lem[wit={C7,V1,Jyo}]{prathamāmbu}
		\rdg[wit={C6,N19,V15,J10}]{prathamāṃ ca}
		\rdg[wit={J7}]{prathamaṃ ca}
		\rdg[wit={V3}]{prathamaṃ vi}
		\rdg[wit={V19},post={\unm}]{prathamāṃ}}%
	\app{\lem[wit={ceteri}]{dhārāṃ}
		\rdg[wit={V19},alt={\om}]{\skp{\om}}}}\\+
\pada{vihāya
	\app{\lem[wit={V19,V15,V1,J10,Jyo}]{niḥsāratayāntya}
		\rdg[wit={C6}]{niḥsāratapāṃśu}
		\rdg[wit={C7}]{niḥsārabhayāntya}
		\rdg[wit={J7}]{niḥsāralayāṃtya}
		\rdg[wit={V3}]{niḥsārayāṃtya}
		\rdg[wit={N19}]{niḥsmāratayāṃtya}}dhārām/}\\+
\pada{\app{\lem[wit={ceteri}]{niṣevyate}
		\rdg[wit={C6}]{niṣevite}
		\rdg[wit={V1}]{niḥsevyate}
		\rdg[wit={V3}]{nikhyevyate}}
	śītalamadhya\app{\lem[wit={V3,N19,V15,J10,Jyo}]{dhārā}
		\rdg[wit={C6,J7,C7,V1}]{dhārāṃ}
		\rdg[wit={V19}]{dhārāḥ}}}\\+
\pada{\app{\lem[wit={V3,V19,C7,N19}]{kāpālikaiḥ}
		\rdg[wit={J7,V15,V1,J10}]{kapālikaiḥ}
		\rdg[wit={C6}]{kapālakaiḥ}
		\rdg[wit={Jyo}]{kāpālike}}
	\app{\lem[wit={C6,V3,V1},alt={khaṇḍamatair}]{khaṇḍamatai}
		\rdg[wit={N19}]{khaṃḍamitair}
		\rdg[wit={V15,Jyo}]{khaṃḍamate}
		\rdg[wit={V19,C7}]{kaṃṭhamaṭhair}
		\rdg[wit={J7,J10}]{kuṃṭhamatair}}% kaṃṭha J10pc
	\app{\lem[wit={V19,C7,N19},alt={amaryāḥ}]{r amaryāḥ}
		\rdg[wit={C6}]{amaryā}
		\rdg[wit={V3}]{aryā}
		\rdg[wit={J10}]{amedhyā}
		\rdg[wit={V1}]{amedhya}
		\rdg[wit={J7}]{amedhyāṃ}
		\rdg[wit={Jyo}]{'marolī}
		\rdg[wit={V15}]{'maroḷi}}//} \NotIn{N23}% M3 omits too
	\myfn{\getsiglum{J7} seems to have supplied this verse and the next one from a ms belonging to the {\textepsilon}-group.}\\!

\newpage
%[hp03_096]
\pada{\app{\lem[wit={J7,V19,C7,J10,Jyo}]{amarīṃ}
		\rdg[wit={V3,N19,V15,V1}]{amarī}
		\rdg[wit={C6}]{amariṃ}}
	\app{\lem[resp=emend]{yat}
		\rdg[wit={C6,V3,V19,C7,N19,V15,V1,Jyo}]{yaḥ} % + j10pc
		\rdg[wit={J7,J10}]{yo}}
	\app{\lem[wit={ceteri}]{piben} % paven V19
		\rdg[wit={C7}]{piban}} nityaṃ}
\pada{\app{\lem[wit={G4,C6,V3,V19,N19,V15,Jyo},post={(naśyaṃ \getsiglum{G4,N19,V15})}]{nasyaṃ kurvan}% kurvana V3, naśyaṃ kurvan G4, 
		\rdg[wit={C7}]{na saṃkurvan}
		\rdg[wit={J5,V1}]{nasyaṃ kuryād}% nasya J5, naśyaṃ V1
		\rdg[wit={J7}]{tasya kuryā}
		\rdg[wit={J10}]{tasthaṃ kuryād}}\marmas % nāśāraṃdhrā J10pc
	dine dine/}\\+
\pada{\app{\lem[wit={V19,C7}]{vajrolīṃ cā}
		\rdg[wit={V3,N19,V15,V1}]{vajrolī cā}
		\rdg[wit={J7,J10,Jyo}]{vajrolīm a}
		\rdg[wit={C6}]{vajrolī ka}}%
	\app{\lem[wit={V3}]{bhyaset seyam}
		\rdg[wit={V19,C7,V15}]{bhyasec ceyam}
		\rdg[wit={N19}]{bhyasec ceya} % ceya ama°
		\rdg[wit={V1}]{bhyasen nityaṃ} % nityaṃ a°
		\rdg[wit={J7}]{bhyaset satve}
		\rdg[wit={J10}]{bhyasec chattve} % bhyased eva J10pc
		\rdg[wit={Jyo}]{bhyaset samyak}
		\rdg[wit={C6}]{thyate seyam}}}
\pada{\app{\lem[wit={ceteri}]{amarolīti}
		\rdg[wit={Jyo}]{sāmarolīti}
		\rdg[wit={V15}]{amaroḷīṃ tu}}
	\app{\lem[wit={ceteri}]{kathyate}
		\rdg[wit={V15}]{kalpayet}
		\rdg[wit={J10}]{kasyate}}//}\myfn{%
	In \getsiglum{Jyo} the verse \ref{VuIII98} is found after this}
	\NotIn{N23}\label{III96}\\! % M3 omits too


\startgray
%[hp03_097]
\pada{\app{\lem[wit={N23,V19,C7,N19,V15,J10,Jyo}]{puṃso}
		\rdg[wit={C6,J7}]{puṃsor}
		\rdg[wit={V3,V1}]{puṃsāṃ}}
	\app{\lem[wit={ceteri}]{binduṃ}
		\rdg[wit={V3,N19,V15,J10}]{bindu}}
	\app{\lem[wit={Gr2}]{samākṛṣya}
		\rdg[wit={ceteri}]{samākuñcya}}}\label{III99}
\pada{samyag abhyāsa% N19 om. sa
	\app{\lem[wit={C6,V19,C7,V15,Jyo}]{pāṭavāt}
		\rdg[wit={V3,Gr2,N19,J10}]{pāṭavān}
		\rdg[wit={V1}]{pāravān}}/}\\+  %3.97
\pada{yadi nārī rajo rakṣed} % rakṣe N19
\pada{\app{\lem[wit={V3,J7,N19,V1,Jyo}]{vajrolyā}
		\rdg[wit={V19}]{vajrolyāṃ}
		\rdg[wit={C7}]{vajrolya}
		\rdg[wit={C6}]{vajrolī}
		\rdg[wit={V15}]{vajroḷi}
		\rdg[wit={J10}]{saṃyoge}
		\rdg[wit={N23},alt={\om},post={(\ref{III97}d--\ref{III101}a om. prob. by eye-skip)}]{\skp{\om}}}
	\app{\lem[wit={V3}]{sā hi}
		\rdg[wit={C6,J7,N19}]{saha}
		\rdg[wit={V19,V15,V1,Jyo}]{sāpi}
		\rdg[wit={C7}]{syāpi}
		\rdg[wit={J10}]{cāpi}
		\rdg[wit={N23},alt={\om}]{\skp{\om}}}
		yoginī//}\label{III97} \NotIn{Gr1r} \\!
	%\anm{\getsiglum{N23} om. 97d--100c by haplogr.?}

%[hp03_098]
\pada{tasyāḥ kiṃcid rajo nāśaṃ} % tasyā N4
\pada{na gacchati % gakṣati; gacchaṃti N19
		na saṃśayaḥ/}\\+
\pada{\app{\lem[wit={C6,V19,C7,V15,V1,J10,Jyo}]{tasyāḥ}
		\rdg[wit={N19}]{yasyāḥ}
		\rdg[wit={V3}]{asyāḥ}
		\rdg[wit={Gr2},alt={\om}]{\skp{\om}}}
	\app{\lem[wit={ceteri}]{śarīre}
		\rdg[wit={C7,V15}]{śarīra}
		\rdg[wit={Gr2},alt={\om}]{\skp{\om}}}
	\app{\lem[wit={C6,V3,C7,N19,V15,V1}]{nādas tu}
		\rdg[wit={J10}]{nādas tat}
		\rdg[wit={V19}]{nādātmā}
		\rdg[wit={Jyo}]{nādaś ca}
		\rdg[wit={Gr2},alt={\om}]{\skp{\om}}}}
\pada{\app{\lem[wit={V3,V19,C7,N19,V15,V1,Jyo}]{bindutām eva}
		\rdg[wit={J10}]{bindus tam eva}
		\rdg[wit={C6}]{vyaṃjatām eva}
		\rdg[wit={Gr2},alt={\om}]{\skp{\om}}} gacchati//}
	\NotIn{Gr1r}	\lineom{cd}{J7}\\!

%\newpage
%[hp03_099]
\pada{sa bindus tad rajaś caiva} % svabiṃdu? V15
\pada{\app{\lem[wit={ceteri}]{ekī}
		\rdg[wit={C7}]{hy ekī}
		\rdg[wit={N23},alt={\om}]{\skp{\om}}}%
	\app{\lem[wit={C6,V3,J7,V19,C7,V15,Jyo}]{bhūya}
		\rdg[wit={N19,J10}]{bhūyaḥ}
		\rdg[wit={V1}]{bhūtaḥ}
		\rdg[wit={N23},alt={\om}]{\skp{\om}}}
	\app{\lem[wit={C7}]{svadehajaṃ}
		\rdg[wit={V19}]{sadehajaṃ}
		\rdg[wit={C6,J10}]{svadehajaiḥ}
		\rdg[wit={V3,J7,N19,V15,V1}]{svadehajau}
		\rdg[wit={Jyo}]{svadehagau}
		\rdg[wit={N23},alt={\om}]{\skp{\om}}}\marma/}\\+
\pada{\app{\lem[wit={C6,V3,N19,V15,V1,J10}]{vajrolyā}
		\rdg[wit={J7,V19,C7,Jyo}]{vajrolya}
		\rdg[wit={N23},alt={\om}]{\skp{\om}}}bhyāsayogena} % +N4
\pada{sarva% sarvāṃ C6
	\app{\lem[wit={C7,J10}]{siddhiḥ}
		\rdg[wit={V3,V1}]{siddhi}
		\rdg[wit={C6,J7,V19,N19,V15,Jyo}]{siddhiṃ}
		\rdg[wit={N23},alt={\om}]{\skp{\om}}} % +J10pc,M1
	\app{\lem[wit={V3,V1,J10}]{prajāyate}% = DYŚ
		\rdg[wit={J7,V19,C7}]{prakurvate}
		\rdg[wit={N19,V15}]{prakurvataḥ}% +M1,P11 ##
		\rdg[wit={Jyo}]{prayacchataḥ}
		\rdg[wit={C6}]{prayacchati}
		\rdg[wit={N23},alt={\om}]{\skp{\om}}}//}
	\NotIn{Gr1r}\\!
\endgray


%[hp03_100]
\pada{\app{\lem[wit={Gr1r},postwit={(rakṣed ākuṃbhanonordhaṃ \getsiglum{J5}, rakṣaṇe kuṃcanenorddhva \getsiglum{N24}, \textit{damaged} \getsiglum{G4})}]{rakṣed ākuñcanenordhvaṃ}
	\rdg[wit={Jyo}]{rakṣed ākuñcanād ūrdhvaṃ}
	\rdg[wit={J7}]{mehenākuñcanād ūrdhvaṃ}
	\rdg[wit={J10}]{meḍhrām ākuṃcanād ūrdhvaṃ }}}
\pada{\app{\lem[wit={Jyo}]{yā rajaḥ sā hi yoginī}
	\rdg[wit={J5}]{yā rajaḥ saha yoginī}
	\rdg[wit={J7,J10}]{rajasāpi hi yoginaḥ}}/} \sgwit{Gr1r,J7,J10,Jyo}\myfn{\getsiglum{J7} has this hemistich between \ref{III96} and \ref{III99}.}\\+
% J5: rakṣed ākuṃbhanonordhaṃ yā rajaḥ saha yoginī
% N24: rakṣaṇe kuṃcanenorddhva yo radhaḥ saha yogavit
% G4: (damaged)


\pada{\app{\lem[wit={ceteri}]{atītānāgataṃ}
		\rdg[wit={C6}]{atītānāgate}
		\rdg[wit={V15}]{atītānāgatiṃ}
		\rdg[wit={N19}]{atītānāṃ gatiṃ}
		\rdg[wit={N23},alt={\om}]{\skp{\om}}} vetti}
\pada{\app{\lem[wit={ceteri}]{khecarī ca}% +J10pc
		\rdg[wit={V19},post={(one syllable missing)}]{khecarī}
		\rdg[wit={C7}]{khecarīṃ la°}
		\rdg[wit={J10}]{khecaraś ca}
		\rdg[wit={N23},alt={\om}]{\skp{\om}}}
	\app{\lem[wit={ceteri}]{bhaved dhruvam}
		\rdg[wit={C7}]{°bhate dhruvam}
		\rdg[wit={J7}]{prajāyate}
		\rdg[wit={N23},alt={\om}]{\skp{\om}}}//}\\!


\newpage
%[hp03_101]
\pada{dehasiddhiṃ % dehe C6
	\app{\lem[wit={ceteri}]{ca}
		\rdg[wit={V1}]{tu}
		\rdg[wit={N23},alt={\om}]{\skp{\om}}}
	\app{\lem[wit={ceteri}]{labhate}
		\rdg[wit={C6}]{labhyeta}
		\rdg[wit={N23},alt={\om}]{\skp{\om}}}}
\pada{\app{\lem[wit={J7,V19,C7,Jyo}]{vajrolyabhyāsa}
		\rdg[wit={C6,V3,N23,N19,V15,V1,J10}]{vajrolyābhyāsa}% +N4; ā N23pc
		}yogataḥ/}\\+ % +N24, yogavit G4, om. J5
\pada{ayaṃ \app{\lem[wit={G4,N24}]{śubhakaro}
	\rdg[wit={Jyo}]{puṇyakaro}} yogo}
\pada{\app{\lem[wit={Jyo}]{bhoge bhukte'pi muktidaḥ}
	\rdg[wit={G4}]{bhāvamukthivimukthidaḥ}
	\rdg[wit={N24}]{bhyāsyayuktasya muktida}}//}\\+
	\anm{cd in \getsiglum{G4,N24,Jyo}; cf. \ref{III94}cd}\\!

\endverse
\startaltrecension{}\normalsize

\pada{\app{\lem[wit={V19,C7,N19,V15,V1,J10}]{tasmād ayaṃ}
		\rdg[wit={C6,V3}]{yasmād ayaṃ}}
	\app{\lem[wit={C6,V3,V19,C7,N19,V15}]{sādhakāya}
		\rdg[wit={V1}]{sādhako'yaṃ}
		\rdg[wit={J10}]{sādhakānāṃ}}}
\pada{\app{\lem[wit={V3,J10}]{bhoge}
		\rdg[wit={C6,C7,N19,V15,V1}]{bhoga}
		\rdg[wit={V19}]{yoga}}
	\app{\lem[resp=emend]{bhukte}
		\rdg[wit={V3},postwit={\getsiglum{V19}\postcorr?}]{bhukti}
		\rdg[wit={N19}]{mukte}
		\rdg[wit={V19,C7,V1,J10}]{mukti}
		\rdg[wit={V15}]{yukto}
		\rdg[wit={C6}]{yoge}}%
	\app{\lem[wit={C6,V3,N19,V15}]{'pi muktidaḥ} % <<pi>> V3
		\rdg[wit={V19,C7,V1,J10}]{vimuktidaḥ}}\marma//}
	\lineom{ab}{Gr1r,Gr2,Jyo}\label{III101}\\+
\pada{tasmāt puṇyavatām %
	\app{\lem[wit={C6,Gr2,J10}]{eva}
		\rdg[wit={V3,V19,C7,N19,V15,V1}]{evam}}}
\pada{\app{\lem[wit={C6,V3,Gr2,N19,V15,V1}]{ayaṃ yogaḥ}
		\rdg[wit={V19,C7}]{eṣa yogaḥ}
		\rdg[wit={J10}]{yogo'yaṃ}}
	\app{\lem[wit={C6,V3,Gr2,V19,C7,N19,V15,V1}]{prasidhyati}
		\rdg[wit={J10}]{saṃprasidhyati}}}//%
	\myfn{In \getsiglum{N23} chap. 3 ends with this verse (the 100th!). Chap. 4 contains only 29 verses, which are the remaining verses of the usual Chap. 3. Chap. 5 corresponds to the usual Chap. 4.}
%\myfn{\getsiglum{V19} adds here: \devnote{iti haṭhayogapradīpikāyāṃ paṃcama upadeśaḥ// 5 // samāptoyaṃ graṃthaḥ// saṃvat 1707 jyeṣṭha kṛṣṇa 4 bhṛgau liṣitam idaṃ// // śubhaṃ// //}; \getsiglum{C7} \devnote{iti śrīmadātmārāmaviracitāyāṃ pañcamoyam upadeśaḥ// 5 // śubham astu sarvajagatām//}; \getsiglum{P23} \devnote{iyaṃ vajrolī trayodaśe patre śakticālanāt pūrvaṃ jñātavyā// //iti śrīātmārāmamunīṃdraviracitāyāṃ haṭhadīpikāyaṃ(!) paṃcamopadeśaḥ// 5 //}}
	\lineom{cd}{Gr1r,Jyo}\\!
\endaltrecension
\startverse


\outdent
\app{\lem[wit={C6,V3,N19,V1,J10,Jyo}]{atha}
\rdg[wit={Gr2,Gr3a,V15},alt={\om}]{\skp{\om}}}
\app{\lem[wit={C6,V3,J7,Gr3a,N19,V1,Jyo}]{śakticālanam}
	\rdg[wit={N23}]{śaktiyānaṃ}
	\rdg[wit={J10}]{śakti}
	\rdg[wit={V15},alt={\om}]{\skp{\om}}}/

\startgray
%[hp03_102]
\pada{\app{\lem[wit={C6,V3,Gr3a,V1,J10,Jyo}]{kuṭilāṅgī}
	\rdg[wit={J7,N19,V15}]{kuṃḍalāṅgī}
	\rdg[wit={N23}]{kundalīgī}} kuṇḍalinī} % kuḍalinī V15
\pada{bhujaṅgī
\app{\lem[wit={C6,V3,J7,N19,V15,V1,J10,Jyo}]{śaktir īśvarī}
	\rdg[wit={V19,K3}]{śaktir aiśvarī}
	\rdg[wit={N23}]{śaktir asvarī}
	\rdg[wit={C7}]{śaktivardhinī}}/}\\+
\pada{\app{\lem[wit={C6,V3,Gr2,N19,V15,V1,J10,Jyo}]{kuṇḍaly} % kuṃḍalasaṃdhanī N23
	\rdg[wit={Gr3a}]{kuṭily}}
\app{\lem[wit={ceteri}]{arundhatī}
	\rdg[wit={V1}]{ā[ku]ṃḍalī}
	\rdg[wit={J10}]{āceti ruṃ°}}
\app{\lem[wit={V1}]{ceti} % c[e]ti V1
	\rdg[wit={V3}]{veti}
	\rdg[wit={N19}]{vati}
	\rdg[wit={V15}]{caiva}
	\rdg[wit={C6,Jyo}]{caite}
	\rdg[wit={Gr2,V19,C7}]{devī}
	\rdg[wit={K3}]{dīvī}
	\rdg[wit={J10}]{dhaṃti}}}
\pada{\app{\lem[wit={ceteri}]{śabdāḥ paryāyavācakāḥ} % śabdā V3; °vācakā V3
	\rdg[wit={V19,C7}]{śabdaḥ paryāyavācakaḥ}}//} \NotIn{Gr1r}\\!

%[hp03_103]
\pada{\app{\lem[wit={ceteri}]{udghāṭayet}
	\rdg[wit={N19}]{udghāṭayati}}
\app{\lem[wit={ceteri}]{kapāṭaṃ}
	\rdg[wit={C7}]{kapālaṃ}}
\app{\lem[wit={ceteri}]{tu}
	\rdg[wit={N19},alt={\om}]{\skp{\om}}}}
\pada{yathā
\app{\lem[wit={ceteri}]{kuñcikayā}% kuṃcīkayā V3
	\rdg[wit={C6}]{kaṃcukayā}} haṭhāt/}\\+
\pada{kuṇḍalinyā tathā yogī}
\pada{mokṣadvāraṃ
\app{\lem[wit={ceteri}]{vibhedayet}
	\rdg[wit={N23}]{prabhedayet}
	\rdg[wit={J7}]{nirodhayet}}//}\myfn{This verse and the next one are transposed in \getsiglum{N19}.} \NotIn{Gr1r}\\!

%[hp03_104]
\pada{yena \app{\lem[wit={C6,V3,N19,V15,V1,J10,Jyo}]{mārgeṇa}
	\rdg[wit={Gr2,Gr3a}]{dvāreṇa}} gantavyaṃ}
\pada{brahmasthānaṃ nirāmayam/}\\+ % °khyānaṃ N23, sthāna N19
\pada{mukhen\app{\lem[wit={ceteri}]{ācchādya} % sukhenā° K3
	\rdg[wit={V19}]{ākṣādya/ājñādya}
	\rdg[wit={N19}]{āvādya}}
\app{\lem[wit={N23,C7,J10}]{taddvāraṃ}
	\rdg[wit={V3,J7,N19,V15,V1,Jyo}]{tadvāraṃ}
	\rdg[wit={C6}]{taṃ dvāraṃ}
	\rdg[wit={V19}]{nadvāraṃ}
	\rdg[wit={K3}]{tedvāraṃ}}}
\pada{prasuptā parameśvarī//} \NotIn{Gr1r}\\!

\newpage
%[hp03_105]
\pada{\app{\lem[wit={Gr2,N19,V15}]{kandordhvaṃ}
	\rdg[wit={V19,C7,V1,J10,Jyo},post={(kandho° \getsiglum{V19}\antecorr)}]{kandordhve}
	\rdg[wit={V3}]{kandorddha}
	\rdg[wit={C6}]{kaṃṭhorddhaṃ}
	\rdg[wit={K3}]{kuṇḍovvo}} kuṇḍalī
\app{\lem[wit={ceteri},alt={śaktiḥ/śaktir}]{śaktiḥ}
	\rdg[wit={V3}]{śakti}}}
\pada{\app{\lem[wit={C6,V3,V15,V1,J10,Jyo}]{suptā}
	\rdg[wit={Gr2,K3,N19}]{buddhā}
	\rdg[wit={V19,C7}]{baddhā}} mokṣāya yoginām/}\\+ % yoginaṃ N23
\pada{bandhanāya ca
\app{\lem[wit={ceteri}]{mūḍhānāṃ}
	\rdg[wit={J7}]{mūrkhāṇāṃ}}}
\pada{yas tāṃ vetti sa yogavit//} \NotIn{Gr1r}\\! % vartti N23; taṃ J10ac

%\newpage
%[hp03_106]
\pada{\app{\lem[wit={ceteri}]{ambhodhi}
		\rdg[wit={Gr1r,Jyo},alt={\om}]{\skp{\om}}}%
	\app{\lem[wit={V3,Gr2,V15,V1,J10}]{śailadvīpānām} % aṃbhodha N23
		\rdg[wit={C6}]{śailordvagānām}
		\rdg[wit={N19}]{plauladvīpānām}
		\rdg[wit={Gr3a}]{dvīpaśailānām}
		\rdg[wit={Gr1r,Jyo},alt={\om}]{\skp{\om}}}} % dvepa K3
\pada{\app{\lem[wit={ceteri}]{ādhāraḥ}
		\rdg[wit={J7}]{ādharaḥ}
		\rdg[wit={N19}]{ādhāraṃ}
		\rdg[wit={Gr1r,Jyo},alt={\om}]{\skp{\om}}} śeṣakuṇḍalī/}
		\lineom{ab}{Gr1r,Jyo}\\+
\pada{aśeṣayoga\app{\lem[wit={ceteri}]{tantrāṇām} % +J10pc
	\rdg[wit={J10}]{jagatām}
	\rdg[wit={Gr1r,V1,Jyo},alt={\om}]{\skp{\om}}}}
\pada{ādhāraḥ
\app{\lem[wit={ceteri}]{kuṇḍalī tathā}
	\rdg[wit={V19}]{kuṇḍalī yathā}
	\rdg[wit={V15}]{śeṣakuṇḍalī}
	\rdg[wit={Gr1r,V1,Jyo},alt={\om}]{\skp{\om}}}//}
	\lineom{cd}{Gr1r,V1,Jyo} \anm{cf. 3.1}\\!

%[hp03_107]
\pada{kuṇḍalī
\app{\lem[wit={V3,Gr2,Gr3a,N19,Jyo}]{kuṭilākārā}
	\rdg[wit={V15}]{kuṃḍilākārā}
	\rdg[wit={V1}]{kuṃḍalākārā}
	\rdg[wit={J10}]{kuṭilākarī}}}
\pada{sarpavat parikīrtitā/}\\+
\pada{sā śaktiś cālitā yena} % ye<<na>> J7
\pada{sa mukto nātra saṃśayaḥ//} \NotIn{Gr1r} \anm{= 4.77*1}\\!
\endgray

%\newpage
%[hp03_108]
\pada{gaṅgāyamunayor madhye} % jamunāyor V3,J10
\pada{\app{\lem[wit={ceteri}]{bālaraṇḍā}% vā in mss; bā V15,V1,K3
	\rdg[wit={Jyo}]{bālaraṇḍāṃ}}
\app{\lem[wit={ceteri}]{tapasvinī}
	\rdg[wit={N19}]{tapaśvinī}
	\rdg[wit={V19}]{tapaścānī}
	\rdg[wit={Jyo}]{tapasvinīm}
	\rdg[wit={C6,P11}]{sarasvatī}}/}\\+
\pada{balātkāreṇa gṛhṇīyāt} % gṛhnī° Gr2,V19,N19
\pada{tad viṣṇoḥ paramaṃ padam//}\\! % padamaṃ N23

\endverse
\startaltrecension{}
%[hp03_108_1]
\pada{iḍā bhagavatī gaṅgā}
\pada{piṅgalā yamunā nadī/}\\+ % jamunā V3
\pada{iḍāpiṅgalayor madhye} % ilā V1; piṃgalāyor V3,J10
\pada{bālaraṇḍā
	\app{\lem[wit={C6,V3,J7,V1,J10}]{sarasvatī}
	\rdg[wit={Jyo}]{ca kuṇḍalī}}//}
\sgwit{C6,V3,J7,V1,J10,Jyo}\\!
%\NotIn{P11,N23,Gr3a,N19,V15}
\endaltrecension
\startverse

%[hp03_109]
\pada{\app{\lem[wit={ceteri}]{pucchaṃ}
	\rdg[wit={K3,J10,Jyo}]{pucche}}
\app{\lem[wit={C6,V3,Gr2,Gr3a,N19,J10,Jyo}]{pragṛhya}
	\rdg[wit={V15}]{nigṛhya}
	\rdg[wit={V1}]{gṛhya}}
\app{\lem[wit={C6,J7,Gr3a}]{bhujagīṃ}
	\rdg[wit={V3,N23}]{bhujaṃgī}
	\rdg[wit={J10}]{bhujaṃgīṃ}
	\rdg[wit={V1}]{bhujaṃgīva}
	\rdg[wit={N19},alt={\illeg}]{\skp{\illeg}}}}
\pada{suptām % \illeg N19
\app{\lem[wit={C6,V3,J7,V19,K3,V15,J10}]{udbodhayed}
	\rdg[wit={V1,Jyo}]{udbodhayec}
	\rdg[wit={C7}]{uddyotayed}
	\rdg[wit={N23}]{udrodhyamed}
	\rdg[wit={N19},alt={\illeg}]{\skp{\illeg}}}
\app{\lem[wit={P11,Gr2}]{abhīḥ}% abhī P11
	\rdg[wit={V15}]{abhiḥ}
	\rdg[wit={Gr3a}]{api}
	\rdg[wit={V3,V1,J10,Jyo}]{ca tām}
	\rdg[wit={C6}]{balāt}
	\rdg[wit={N19},alt={\illeg}]{\skp{\illeg}}}/}\\+
\pada{nidrāṃ vihāya sā
\app{\lem[wit={ceteri}]{ṛjvī}
	\rdg[wit={J7}]{ṛjvīṃ} % ṃ not m
	\rdg[wit={V3}]{rujvī}
	\rdg[wit={N19}]{rajvī}
	\rdg[wit={Jyo}]{śaktir}}}
\pada{\app{\lem[wit={K3,C7,V15,V1,J10,Jyo},alt={ūrdhvam/ūrddham}]{ūrdhvam}
	\rdg[wit={N23}]{urddham}
	\rdg[wit={C6,V3,V19}]{mūrddham}
	\rdg[wit={N19}]{kurddham}}
\app{\lem[wit={ceteri}]{uttiṣṭhate}
	\rdg[wit={N19}]{ākṛṣyate}}
\app{\lem[wit={ceteri}]{haṭhāt}% +P11
	\rdg[wit={C6}]{kṣaṇāt} % +V23
	}//}\\!

\newpage
\startgray
%[hp03_110]
\pada{\app{\lem[wit={C6,Gr2,Gr3a,N19}]{paristhitā caiva}
	\rdg[wit={V15}]{paristhitasyaiva}
	\rdg[wit={V1}]{paristhitā [sai]va}
	\rdg[wit={V3}]{pṛṣṭisthitasyaiva}
	\rdg[wit={J10}]{avasthitasya}
	\rdg[wit={Jyo}]{avasthitā caiva}}
\app{\lem[wit={ceteri},post={(kaṇā° \getsiglum{N23})}]{phaṇāvatī sā}
	\rdg[wit={C7}]{phaṇāvatīva sā}
	\rdg[wit={J10}]{phaṇāryayāṃtīyaṃ}}}\\+
\pada{\app{\lem[wit={ceteri}]{prātaś ca sāyaṃ}
	\rdg[wit={V15}]{prātas tu sāyaṃ}
	\rdg[wit={K3}]{sāyaṃ ca prātaḥ}}
	praharārdha\app{\lem[wit={ceteri}]{mātraṃ}
	\rdg[wit={V3}]{rātraṃ}}/}\\+
\pada{\app{\lem[wit={ceteri}]{prapūrya}
	\rdg[wit={N23}]{prapūrvva}
	\rdg[wit={V1}]{prasūrya}
	\rdg[wit={C6,J10}]{prasārya}}
\app{\lem[wit={P11,V3,Gr2,N19,V15,V1,Jyo}]{sūryāt}
	\rdg[wit={V19}]{sauryā}
	\rdg[wit={C7}]{saudhā}
	\rdg[wit={C6}]{sācāryya}
	\rdg[wit={K3}]{tesau}
	\rdg[wit={J10}]{ryāṣṇut}}
\app{\lem[wit={ceteri}]{paridhāna} % pa[ri] .. na V1
	\rdg[wit={V3}]{paridhāya}
	\rdg[wit={P11}]{mavidhāna}
	\rdg[wit={C6}]{vidhāna}}%
\app{\lem[wit={V3,Gr2,N19,V15,V1}]{yuktā}
	\rdg[wit={Gr3a}]{muktā}
	\rdg[wit={C6,J10,Jyo}]{yuktyā}}}\\+ % better? +P11
\pada{pragṛhya
\app{\lem[wit={J7,Gr3a}]{niryāty avicālinī sā}
	\rdg[wit={N23}]{niyāt* pavicālinī sā}
	\rdg[wit={V15},post={\unm}]{niryātya paricālanīyā}
	\rdg[wit={V3}]{niryāt paricālanīyā}
	\rdg[wit={N19}]{niryāt paricālanīyāt}
	\rdg[wit={C6,Jyo}]{nityaṃ paricālanīyā}
	\rdg[wit={V1}]{teyā paricālanīy[ai]}
	\rdg[wit={J10}]{paricālanīyā}}//}\marma
	\NotIn{Gr1r}\\!

%\newpage
%[hp03_111]
\pada{\app{\lem[wit={C6,Gr2,Gr3a,J10}]{vitastipramitaṃ dīrghaṃ} % vitasthi N23
	\rdg[wit={V3,N19}]{vitastipramitaṃ dairghyaṃ}
	\rdg[wit={V15,V1}]{vitastipramita-dairghyaṃ}
	\rdg[wit={Jyo}]{ūrdhvaṃ vitastimātraṃ tu}}}
\pada{\app{\lem[wit={J7,N19}]{vistāre}
	\rdg[wit={C6,V3,N23,Gr3a,V15,V1,J10,Jyo}]{vistāraṃ}} caturaṅgulam/}\\+
\pada{\app{\lem[wit={ceteri}]{mṛdulaṃ}
	\rdg[wit={V19}]{mṛlaṃ}}
\app{\lem[wit={ceteri}]{dhavalaṃ}
	\rdg[wit={C7}]{pavanaṃ}} proktaṃ}
\pada{\app{\lem[wit={V15,V1,J10}]{veṣṭanāmbara}
	\rdg[wit={J7}]{veṣṭanāṃvala}
	\rdg[wit={N23,N19}]{vaṣṭanāṃcara}
	\rdg[wit={C6}]{vaṣṭanāṃba}
	\rdg[wit={N19}]{vaṣṭanāṃ}
	\rdg[wit={V3}]{veṣṭatāṃvara}
	\rdg[wit={Jyo}]{veṣṭitāmbara}
	\rdg[wit={Gr3a}]{veṣṭanādhāra}}lakṣaṇam//} \NotIn{Gr1r}\\!
\endgray

%\newpage
%[hp03_112]
\myfn{\getsiglum{Jyo} has 3.64 before this verse.}%
\pada{\app{\lem[wit={ceteri}]{vajrāsana}
	\rdg[wit={C6,Jyo}]{vajrāsane}}sthito yogī}
\pada{cālayitvā
\app{\lem[wit={C6,V3,Gr2,Gr3a,N19}]{tu}
	\rdg[wit={V15,J10,Jyo}]{ca}% +J5
	\rdg[wit={V1},alt={\om}]{\skp{\om}}} kuṇḍalīm/}\\+% °lī N23,N19,C6,V3
\pada{\app{\lem[alt={\ante kuryād \add},nosep]{}
	\rdg[wit={N23,K3,C7},postwit={(as header \getsiglum{C7})}]{sūryabhedāt}}
\app{\lem[wit={C6,V3,V1,J10,Jyo}]{kuryād}
	\rdg[wit={Gr2,V19,C7,N19,V15}]{sūryād}
	\rdg[wit={K3}]{tathā}}
\app{\lem[wit={ceteri}]{anantaraṃ}
	\rdg[wit={N23}]{vanara}
	\rdg[wit={K3}]{sūryāt}}
\app{\lem[wit={V15,J10}]{bhastrīṃ}
	\rdg[wit={N23,Gr3a}]{bhastrī}% +J5
	\rdg[wit={J7}]{bhasrī}
	\rdg[wit={V3,N19}]{bhastri}
	\rdg[wit={C6,Jyo}]{bhastrāṃ}
	\rdg[wit={V1},alt={\illeg}]{\skp{\illeg}}
	\rdg[wit={K3},alt={\om}]{\skp{\om}}}}
\pada{\app{\lem[wit={ceteri}]{kuṇḍalīm āśu bodhayet} % āsu V3
	\rdg[wit={K3},alt={\om}]{\skp{\om}}}//}\\!

%\newpage
%[hp03_113]
\pada{\app{\lem[wit={ceteri}]{bhānor}
	\rdg[wit={K3},alt={\om}]{\skp{\om}}}
\app{\lem[wit={ceteri}]{ākuñcanaṃ kuryāt}
	\rdg[wit={V19}]{ākuñcanaṃ pu(?)ryāt}
	\rdg[wit={V1}]{ākuṃcanaivaṃ}% āku(ṃ)nacaivaṃ V1
	\rdg[wit={J10}]{ākuñcanenaiva}
	\rdg[wit={K3},alt={\om}]{\skp{\om}}}}
\pada{kuṇḍalīṃ % kuṃḍalī Gr2,V15,N3,V3, kuṇḍalīś K3;
\app{\lem[wit={ceteri}]{cālayet} % cālayat N23
	\rdg[wit={N3}]{bodhayet}}
\app{\lem[wit={ceteri}]{tataḥ}
	\rdg[wit={J10}]{tadā}}/}\\+
\pada{\app{\lem[wit={ceteri}]{mṛtyu}
	\rdg[wit={J10}]{mṛtyor}}%
\app{\lem[wit={ceteri}]{vaktra}
	\rdg[wit={V3}]{vaktraṃ}}gatasyāpi}
\pada{tasya mṛtyubhayaṃ kutaḥ//}\label{VuIII116}\\!

\newpage
\startgray
%[hp03_114]
\pada{nāsā\app{\lem[wit={V3,Gr2,N19,V15,J10,Jyo}]{dakṣiṇamārgavāhi} % nāśā N19; +P11
	\rdg[wit={C6}]{dakṣiṇavāhimārga}
	\rdg[wit={K3,C7}]{dakṣiṇavartmavāhi}
	\rdg[wit={V19}]{paścimavartmavāhi}
	\rdg[wit={V1}]{da\,..\,ṇa[vā]\,..\,mārgeṇa}}%
\app{\lem[wit={V3,N19,V1,J10,Jyo}]{pavanāt}% +P11
	\rdg[wit={V15}]{pavanot}
	\rdg[wit={C6,J7,Gr3a}]{pavano}
	\rdg[wit={N23}]{pavana}}
\app{\lem[wit={V3,N19,V15,V1,J10,Jyo}]{prāṇo}% +P11
	\rdg[wit={N23}]{prāṇe}
	\rdg[wit={V19}]{ghrāṇo}
	\rdg[wit={C6,J7,K3,C7}]{ghrāṇe}}
\app{\lem[wit={V3,K3,V15,V1,Jyo}]{'tidīrghīkṛtaś}
	\rdg[wit={J7}]{'tidīrghīkṛteś}
	\rdg[wit={N19,J10}]{tidīrghākṛtiś}
	\rdg[wit={N23}]{tidīrghākṛtaś}
	\rdg[wit={V19},post={(°kṛtaś \emph{pc}?)}]{tirghīkṛtiś}
	\rdg[wit={C6}]{na dīrghīkṛtaḥ}% pi P11
	\rdg[wit={C7}]{ca dīrghīkṛtaś}}}\\+
\pada{\app{\lem[wit={C6,V15,V1,J10}]{candrāmbhaḥ} % caṃdrāṃbhaḥ; 2nd ṃ unclear V1
	\rdg[wit={Gr2,Jyo}]{candrābhaḥ}
	\rdg[wit={Gr3a}]{candrāṃtaḥ}
	\rdg[wit={V3}]{caṃdrāṃgāt}
	\rdg[wit={N19}]{caṃdrād[vā]}}%
\app{\lem[wit={C6,Gr2,N19,J10,Jyo}]{paripūritāmṛtatanuḥ}
	\rdg[wit={V15}]{paripūrṇatāmṛtatanuḥ}
	\rdg[wit={V3}]{paripūritāmṛtyutanuḥ}
	\rdg[wit={V1}]{paripūritā\,..\,..\,..\,..}
	\rdg[wit={Gr3a}]{paripūrya pūritatanuḥ}}
\app{\lem[wit={ceteri}]{prāg}% prāghgh° J7,V15; prāk V1
	\rdg[wit={C6,V19}]{prā}} ghaṇṭikāyās % °kāyā N23; ghaṇṭi illeg. V1
\app{\lem[wit={C6,V3,N23,V19,K3,N19,J10}]{tathā}
	\rdg[wit={C7,Jyo}]{tataḥ}
	\rdg[wit={J7}]{tadā}
	\rdg[wit={V15}]{sadā}
	\rdg[wit={V1},alt={\illeg}]{\skp{\illeg}}}/}\\+
\pada{\app{\lem[wit={J7,Gr3a}]{bhindan}% siñcan! Amaraughaśāsana
	\rdg[wit={N23}]{bhidan}
	\rdg[wit={N19,V15}]{chindan}% ##
	\rdg[wit={V3,J10}]{chinnat}
	\rdg[wit={Jyo}]{chittvā}
	\rdg[wit={C6}]{chaṃdaḥ}
	\rdg[wit={V1},alt={\illeg}]{\skp{\illeg}}}
	kālaviśāla% kāla illeg. V1
\app{\lem[wit={ceteri}]{vahni}
	\rdg[wit={V15}]{pāśa}
	\rdg[wit={N23},alt={\om}]{\skp{\om}}}%
\app{\lem[wit={V3,J7,K3,V1}]{vaśagān}
	\rdg[wit={J10}]{vaśagāt}
	\rdg[wit={V19,C7,V15}]{vaśagā}
	\rdg[wit={N19}]{vaśanān}
	\rdg[wit={Jyo}]{vaśagaṃ}
	\rdg[wit={C6}]{pavanān}}
\app{\lem[wit={ceteri}]{bhrū}
	\rdg[wit={V15}]{bhū}
	\rdg[wit={N23}]{tū}
	\rdg[wit={V3}]{bhṛṃ}
	\rdg[wit={J10}]{prāg}}randhranāḍī% raṃdhranā_n N23
\app{\lem[wit={ceteri},alt={gaṇān/gaṇāṃs}]{gaṇāṃs} % gaṇān* J7,P23; gaṇāṃs Gr3
	\rdg[wit={J10}]{gaṇāt}
	\rdg[wit={Jyo}]{gataṃ}}}\\+
\pada{\app{\lem[wit={V3,Gr2,Gr3a,V1}]{taṃ}% P11
	\rdg[wit={C6,N19,V15,J10,Jyo}]{tat}}
	kāyaṃ kurute punar navataraṃ % kāryaṃ, navattaraṃ J10; °tara V3
\app{\lem[wit={C6,V3,J7,V19,K3}]{jīrṇa}
	\rdg[wit={C7,N19}]{jīrṇaṃ} % chiṃjīrṇaṃ N19
	\rdg[wit={J10,Jyo}]{chinna}
	\rdg[wit={V15}]{chinnaṃ}
	\rdg[wit={V1}]{kṛnta}
	\rdg[wit={N23}]{bhasma}}drumaskandhavat//}\marma
	\NotIn{N3}\myfn{In \getsiglum{Jyo} this verse is found after \ref{VuIII121} together with the next one and has no commentary.}
	\\!

%\newpage
%[hp03_115]
\pada{kuṇḍalīṃ % lī Gr2,N19,V15,V3, °liṃ J10
cālayi\app{\lem[wit={J7,Gr3a,N19,Jyo},alt={°tvā tu}]{tvā tu} % cāla-i-tvā V19
	\rdg[wit={N23}]{°tvācca}
	\rdg[wit={C6,V15,J10}]{°tvātha}
	\rdg[wit={V3}]{°tvādhaḥ}
	\rdg[wit={V1},alt={\illeg}]{\skp{\illeg}}}}
\pada{\app{\lem[wit={K3,C7,V15,V1}]{kuryād bhastrīṃ}
	\rdg[wit={V3,V19,N19}]{kuryād bhastrī} % kuryā bhastri V3
	\rdg[wit={J10}]{kuryād bhastrāṃ}
	\rdg[wit={Gr2}]{bhasrī kuryād}
	\rdg[wit={C6,Jyo}]{bhastrāṃ kuryād}}
	viśeṣataḥ/}\\+
\pada{evam
\app{\lem[wit={C7,V15}]{abhyasato}
	\rdg[wit={J10,Jyo}]{abhyasyato}
	\rdg[wit={V3}]{abhyasyatā}
	\rdg[wit={C6,Gr2,V19,K3,N19}]{abhyāsato}
	\rdg[wit={V1}]{..\,..\,syat.}} nityaṃ}
\pada{\app{\lem[wit={ceteri}]{yaminaḥ śaṅkate yamaḥ}
	\rdg[wit={Jyo}]{yamino yamabhīḥ kutaḥ}}//} \NotIn{N3}\\!

%\newpage
%[hp03_116]
\pada{\app{\lem[wit={V3,Gr2,K3,N19,V1}]{tadābhyaset}
	\rdg[wit={J10}]{tadābhyasyet}
	\rdg[wit={C6,V19,V15}]{tad abhyaset}
	\rdg[wit={C7}]{tam abhyaset}}
\app{\lem[wit={ceteri}]{sūryabhedam} % °bhedan*m J10
	\rdg[wit={V15}]{sūryabhede}
	\rdg[wit={C7}]{sūryabījam}}}
\pada{\app{\lem[wit={C6,V3,Gr2,K3,C7,V1,J10}]{ujjāyīṃ} % °yī N23,V3
	\rdg[wit={N19}]{ujjāī}
	\rdg[wit={V15}]{ujjāyāṃ}
	\rdg[wit={V19}]{ujrākhyām}}
\app{\lem[wit={ceteri}]{cāpi}
	\rdg[wit={V15}]{vāpi}
	\rdg[wit={V1}]{[vā]\,..}
	\rdg[wit={V19}]{api}} śītalīm/}\\+ % śītalī N23,N19,V15; sītalī V3
\pada{evam
abhyāsa\app{\lem[wit={ceteri}]{yuktasya}
	\rdg[wit={J10}]{yogena}}}
\pada{\app{\lem[wit={Gr2,Gr3a}]{yamas tu}
	\rdg[wit={V15}]{śramas tu}
	\rdg[wit={N19,V1}]{śamino}% C6pc
	\rdg[wit={C6,V3,J10}]{śamano}}
\app{\lem[wit={ceteri}]{yaminaḥ}
	\rdg[wit={V3}]{yaminaṃ}} kutaḥ//} % kva ca C6
	\NotIn{N3,Jyo}\\!
\endgray

%\newpage
%[hp03_117]
\pada{muhūrtadvayaparyantaṃ} % mahūrtta V3; paryaṃta V15
\pada{\app{\lem[wit={Gr2,Gr3a}]{nirbharaṃ}
	\rdg[wit={N3,C6,V3,V15}]{nirbhayaś}
	\rdg[wit={N19}]{nirbhayaṃś}
	\rdg[wit={V1,J10,Jyo}]{nirbhayaṃ}} % +M1,G7 ##? ṃ unsichtbar V1
\app{\lem[wit={N3,C6,V3,Gr2,N19,V15,V1,Jyo}]{cālanād asau}
	\rdg[wit={Gr3a}]{calanād asau}
	\rdg[wit={J10}]{vā diśodiśa}}/}\\+
\pada{ūrdhvam % ūrdham V3
\app{\lem[wit={ceteri}]{ākṛṣyate}
	\rdg[wit={V15}]{ākṛte}
	\rdg[wit={Gr3a},alt={\om}]{\skp{\om}}} kiṃcit}
\pada{\app{\lem[wit={C6,Gr2,V15}]{suṣumṇāgatakuṇḍalī}
	\rdg[wit={J5,N19}]{suṣumnā kuṇḍalīgatā}% +J5,M1,M3
	\rdg[wit={N3}]{suṣumnā kuṃḍalīgataḥ}% +G7
	\rdg[wit={P11}]{suṣumṇāṃ kuṃḍalīgatāṃ}
	\rdg[wit={V3}]{suṣumnāṃ kuṇḍalī gatā}
	\rdg[wit={Jyo}]{suṣumnāyāṃ samudgatā}
	\rdg[wit={J10}]{suṣumṇāyāḥ samuddhṛtaḥ}
	\rdg[wit={Gr3a,V1},alt={\om}]{\skp{\om}}}//} \lineom{cd}{Gr3a}\\!

\newpage
%[hp03_118]
\pada{\app{\lem[wit={N3,C6,V3,Gr2,N19,V15,J10,Jyo}]{tena kuṇḍalinī tasyāḥ} % kuṃḍalanī N3, kuṇḍalīnī N23; tasyā V3
	\rdg[wit={V1},alt={\om}]{\skp{\om}}}}
\pada{suṣumṇāyāḥ % °yā N3,C6,V3
\app{\lem[wit={V3,Gr2,N19,V15,V1,J10}]{samuddhṛtā} % tāḥ N19, samudhṛtā J5
	\rdg[wit={N3}]{samudbhutā}
	\rdg[wit={C6,P11,Jyo}]{mukhaṃ dhruvam}}/}\\+ % = GŚ
\pada{\app{\lem[wit={ceteri}]{jahāti}
	\rdg[wit={J10}]{na yāti}} tasmāt prāno'yaṃ}
\pada{suṣumṇāṃ vrajati % suṣumṇā C6
\app{\lem[wit={V15,V1,Jyo}]{svataḥ}
	\rdg[wit={N3,P11,V3,N19}]{svanaḥ} % svana V3
	\rdg[wit={C6,Gr2}]{svayam}
	\rdg[wit={J10}]{niścalaḥ}}//} \NotIn{Gr3a}\\!

%\newpage
%[hp03_119]
\pada{\app{\lem[wit={N3,C6,V3,J7,N19,V15,V1,J10,Jyo}]{tasmāt}
	\rdg[wit={N23}]{kasmāt}} saṃcālayen nityaṃ}
\pada{\app{\lem[wit={C6,V3}]{śabdagarbhām}% gabhāṃ C6
	\rdg[wit={N3}]{śabdagaṃdhām}
	\rdg[wit={Gr2,N19,V15}]{śambhugarbhām}
	\rdg[wit={Jyo}]{sukhasuptām}
	\rdg[wit={J10}]{suṣasuptām}
	\rdg[wit={V1},alt={\illeg}]{\skp{\illeg}}}\marmas
\app{\lem[wit={ceteri}]{arundhatīm} % °tī N23,N19,N3, ddhaṃtī V3
	\rdg[wit={C6}]{sarasvatīṃ}}/}\\+
\pada{\app{\lem[wit={Gr2,N19}]{yasyāḥ}
	\rdg[wit={N3,C6,V3,V15,Jyo}]{tasyāḥ}% ##
	\rdg[wit={J10}]{tasyāṃ}
	\rdg[wit={V1}]{[ya]\,..}}
\app{\lem[wit={Gr2,V15}]{saṃcālanenāśu}
	\rdg[wit={N19,J10}]{saṃcālayenāśu}
	\rdg[wit={N3,C6,V3,Jyo}]{saṃcālanenaiva}% sacāla° N3
	\rdg[wit={V1}]{..\,..\,lanen.\,..}}}
\pada{yogī \app{\lem[wit={N3,C6,J7,N19,V15,V1,Jyo},alt={rogaiḥ/rogair}]{rogaiḥ}% raugaiḥ C6
%	\rdg[wit={J7,N19,V15}]{rogair}
	\rdg[wit={N23}]{[r]. .air}
	\rdg[wit={V3}]{rogoḥ}
	\rdg[wit={J10}]{rogāt}}
\app{\lem[wit={N3,C6,V3,V1,J10,Jyo}]{pramucyate}
	\rdg[wit={Gr2,N19,V15}]{vimucyate}}//} \NotIn{Gr3a}\\!

%[hp03_120]
\pada{yena \app{\lem[wit={C6,V3,N23,V15,V1,J10,Jyo}]{saṃcālitā}
	\rdg[wit={N19}]{saṃcalitā}
	\rdg[wit={N3}]{saṃcalatā}
	\rdg[wit={J7}]{sa cālitā}} śaktiḥ} % śakti V3
\pada{sa yogī \app{\lem[wit={N3,C6,V3,N19,V15,V1,J10,Jyo}]{siddhi}
	\rdg[wit={Gr2}]{mukti}}% .. ddhi V1 ##
\app{\lem[wit={ceteri}]{bhājanam}
	\rdg[wit={C6}]{bhājanaḥ}
	\rdg[wit={V1}]{..\,janaḥ}}/}\\+
\pada{kim atra bahunoktena}
\pada{kālaṃ \app{\lem[wit={ceteri}]{jayati}
	\rdg[wit={J10}]{vrajati}} līlayā//} \NotIn{Gr3a}\\! % jayalati? V1


%\newpage
\startgray
%[hp03_121]
\pada{\app{\lem[wit={Gr3a,V1,Jyo}]{brahmacaryaratasyaiva}
	\rdg[wit={J7}]{brahmacarye ca tasyaiva}
	\rdg[wit={N23}]{brahmacatasyaiva}
	\rdg[wit={N19}]{brahmacaryarataś caiva}
	\rdg[wit={V3,V15}]{brahmacaryavratasyaiva}
	\rdg[wit={C6}]{brahmacaryavrataṃ}
	\rdg[wit={J10}]{brahmadharmaratasyaiva}}}
\pada{nityaṃ
\app{\lem[wit={J7,C7,Jyo}]{hitamitāśinaḥ}% °śanaḥ C7ac
	\rdg[wit={V3,N23,V19,N19}]{hitamitāśanaḥ} % °sanaḥ V19
	\rdg[wit={C6}]{hitamitāśanaṃ}
	\rdg[wit={V15}]{hitamitāśanaiḥ}
	\rdg[wit={K3,J10},post={(°śanaḥ \getsiglum{K3}\postcorr)}]{mitahitāśinaḥ}
	\rdg[wit={V1},alt={\illeg}]{\skp{\illeg}}}/}\\+
\pada{\app{\lem[wit={C6,J7,K3,C7,N19,V15,Jyo}]{maṇḍalād}% written often maṇḍalāt*
	\rdg[wit={V3,N23,J10}]{maṃḍalā}
	\rdg[wit={V19}]{maṃḍalī}
	\rdg[wit={V1},alt={\illeg}]{\skp{\illeg}}} dṛśyate siddhiḥ} % siddhiṃ N19; siddhi V3
\pada{\app{\lem[wit={C6,J7,V15,Jyo}]{kuṇḍalya}
	\rdg[wit={V3,V19,C7,N19,J10}]{kuṇḍalyā}
	\rdg[wit={K3}]{kuṇḍalā}
	\rdg[wit={N23}]{kuṇḍali}
	\rdg[wit={V1},alt={\illeg}]{\skp{\illeg}}}bhyāsa%
\app{\lem[wit={C6,Gr2,Gr3a,V1,J10}]{yogataḥ}% V1 uncertain
	\rdg[wit={V3,N19,V15,Jyo}]{yoginaḥ}}//}\label{VuIII121}
	\NotIn{N3}\\!
\endgray

%\newpage
%[hp03_122]
\pada{\app{\lem[wit={N3,C6,V3,Gr2,N19,V15,V1}]{abhyāsa}
	\rdg[wit={Jyo}]{abhyāsān}
	\rdg[wit={J10}]{abhyāsā}}%
\app{\lem[wit={C6,V3,V15,Jyo}]{niḥsṛtāṃ}
	\rdg[wit={V1}]{niḥsṛtā}
	\rdg[wit={J10}]{niḥśritāṃ}
	\rdg[wit={N3}]{nisṛtā}
	\rdg[wit={N19}]{nibhṛtāṃ}
	\rdg[wit={Gr2}]{sahitaṃ}}
\app{\lem[wit={N3,C6,N19,J10,Jyo}]{cāndrīṃ}
	\rdg[wit={V3,V15,V1}]{cāndrī}
	\rdg[wit={Gr2}]{candraṃ}}}
\pada{vibhūtyā saha % vibhūbhyā N23
\app{\lem[wit={N19,V15,V1,J10,Jyo}]{miśrayet}
	\rdg[wit={C6,V3}]{miśritāṃ}
	\rdg[wit={N3}]{mīśritaṃ}
	\rdg[wit={N23}]{micchayet}
	\rdg[wit={J7}]{mūrchayet}}/}\\+
\pada{\app{\lem[wit={C6,N19,V15}]{taddhāraṇaṃ}
	\rdg[wit={Gr2}]{taddhāraṇā}
	\rdg[wit={V3}]{tadvāraṇaṃ}
	\rdg[wit={V1}]{tad[v/dh].\,..\,..}
	\rdg[wit={N3}]{tad dhārayed}
	\rdg[wit={J10}]{tāṃ dhārayed}
	\rdg[wit={Jyo}]{dhārayed}}
\app{\lem[wit={C6,V3,N19,V15}]{tūttamāṃge}
	\rdg[wit={J7}]{cottamāṅge}
	\rdg[wit={N23}]{cottamāṃga}
	\rdg[wit={N3,J10}]{uttamāṅge}
	\rdg[wit={Jyo}]{uttamāṅgeṣu}
	\rdg[wit={V1},alt={\illeg}]{\skp{\illeg}}}}
\pada{\app{\lem[wit={ceteri}]{divya}
	\rdg[wit={C6,V3}]{dīrgha}}%
\app{\lem[wit={N3,C6,V3,N19,V15,V1}]{dṛṣṭipradāyakaṃ}
	\rdg[wit={Gr2}]{dṛṣṭipradāyinī}
	\rdg[wit={J10}]{dṛṣṭipradāyinīṃ}
	\rdg[wit={V1}]{dṛṣṭiḥ prajāyate}}//}\label{VuIII98} \NotIn{Gr3a,Jyo}\myfn{\getsiglum{Jyo} has this verse in the Vajrolī section, immediately after \ref{III96}.}\\!

\newpage
\startgray
%[hp03_123]
\pada{\app{\lem[wit={N23,V19,C7,Jyo}]{dvā}
	\rdg[wit={C6,J7,K3,N19,V15,V1,J10}]{dvi}}saptatisahasrāṇāṃ} % sahasrānā V19
\pada{nāḍīnāṃ mala\app{\lem[wit={J10,Jyo}]{śodhane}% +M1
	\rdg[wit={C6,Gr2,Gr3a,N19,V15,V1}]{śodhanam}}/}\\+
\pada{\app{\lem[wit={Gr3a,V15,Jyo}]{kutaḥ}
	\rdg[wit={N19}]{kṛta}
	\rdg[wit={J7}]{gudaḥ}
	\rdg[wit={V1,J10}]{guda}
	\rdg[wit={C6}]{aṃtaḥ}
	\rdg[wit={N23},alt={\om}]{\skp{\om}}}
\app{\lem[wit={J7,Gr3a,V15,J10,Jyo}]{prakṣālanopāyaḥ}
	\rdg[wit={N19,V1},alt={°pāyaṃ}]{prakṣālanopāyaṃ}
	\rdg[wit={C6}]{prakṣālano vāyuḥ}
	\rdg[wit={N23},alt={\om}]{\skp{\om}}}}
\pada{\app{\lem[wit={J7}]{kuṇḍalyabhyāsato vinā}
	\rdg[wit={N23,Gr3a}]{kuṇḍalyābhyāsato vinā}
	\rdg[wit={C6,N19,V15}]{kuṇḍalyabhyāsanād ṛte}
	\rdg[wit={Jyo}]{kuṇḍalyabhyasanād ṛte}
	\rdg[wit={J10}]{kuṇḍalyabhyāsa iṣyate}
	\rdg[wit={V1}]{ku\,..\,..\,[bhyā]\,..\,[mā]\,..\,..}}//} \NotIn{N3,V3}\\!

%\newpage
\outdent
iti śakticālanam/ \sgwit{N19,V15,V1,J10}
\endgray

%\newpage
%[hp03_124]
\pada{iti mudrā
\app{\lem[wit={ceteri}]{daśa}
	\rdg[wit={N3}]{dabhā}
	\rdg[wit={Gr3a}]{nava}} proktā}
\pada{ādināthena śambhunā/}\\+
\pada{\app{\lem[wit={N3,K3,C7,Jyo}]{ekaikā tāsu}
	\rdg[wit={N19}]{ekaikatāsu}
	\rdg[wit={V19}]{ekaiva tāsu}
	\rdg[wit={Gr2},post={\unm}]{ekaikāpi su°}
	\rdg[wit={V15}]{karaṇe sarva}
	\rdg[wit={J10}]{kāraṇe sarva}
	\rdg[wit={C6,V3}]{kāraṇaṃ sarva}
	\rdg[wit={V1}]{ka\,..\,..\,sarva}}
\app{\lem[wit={N3,Gr2,Gr3a,N19,Jyo}]{yamināṃ}
	\rdg[wit={V15,V1,J10}]{siddhānām}
	\rdg[wit={C6,V3}]{siddhīnām}}}
\pada{\app{\lem[wit={N3,Gr2,Gr3a,N19,Jyo}]{mahāsiddhipradāyinī}
	\rdg[wit={C6,V3,V15,V1,J10},postwit={(\getsiglum{V1} partly illegible)}]{ekaikāpi kṣamaiva sā}
%	\rdg[wit={V1}]{e[k]. .. [p]i [kṣamai] .. ..}
	}//}\label{III124}\myfn{Verse order of \getsiglum{Jyo}: \ref{III127} \rightarrow\ \ref{III128} \rightarrow\ \ref{III125} \rightarrow\ \ref{III126} \rightarrow\ \ref{III124}}\\!

%[hp03_125]
\pada{rājayogaṃ vinā % °yoge N23, yoga N19
\app{\lem[wit={ceteri}]{pṛthvī}
	\rdg[wit={J10}]{pṛthvīṃ}
	\rdg[wit={V15}]{siddhī}
	\rdg[wit={N19}]{vṛddhir}}}
\pada{rājayogaṃ vinā % yoga N19
\app{\lem[wit={ceteri}]{niśā}
	\rdg[wit={J10}]{niśāṃ}
	\rdg[wit={N23}]{nyathā}}/}\\+
\pada{rājayogaṃ vinā mudrā} % yoga N19
\pada{vicitrāpi na
\app{\lem[wit={ceteri}]{rājate}
	\rdg[wit={C6,Jyo}]{śobhate}}//}\label{III125}\\! % rojate N23

%\newpage
%[hp03_126]
\pada{\app{\lem[wit={ceteri}]{mārutasya vidhiṃ}% vidhi J7
	\rdg[wit={V15,V1,J10}]{mārutābhyasanaṃ}}
\app{\lem[wit={ceteri}]{sarvaṃ}% sarva J7
	\rdg[wit={C6}]{sarvāṃ}
	\rdg[wit={N3}]{sarve}
	\rdg[wit={K3,C7}]{siddhiṃ}
	\rdg[wit={J10}]{kiṃcin}}}
\pada{manoyuktaṃ
\app{\lem[wit={ceteri}]{samabhyaset}
	\rdg[wit={V1,J10}]{samācaret}}/}\\+ % yukta N23
\pada{itaratra na kartavyā} % italantra na N23; ityatatra C6; kartavyaṃ N23
\pada{manovṛttir % manārvvartti N23
\app{\lem[wit={N3,C6,Gr2,V19,V15,J10,Jyo}]{manīṣiṇā}
	\rdg[wit={V3,K3,C7,N19}]{manīṣiṇām}
	\rdg[wit={V1}]{..\,[nī]\,..\,ṇ.}}//}\label{III126}\\!

%\newpage
%[hp03_127]
\pada{\app{\lem[wit={N3,C6,V3,J7,Gr3a}]{khilāpi}% +C6,P11 >> Marmasthana
	\rdg[wit={N23}]{sthirāpi}
	\rdg[wit={N19,V15}]{calāpi}
	\rdg[wit={Jyo}]{iyaṃ tu}
	\rdg[wit={V1,J10}]{vināpi}}\marmas
\app{\lem[wit={ceteri}]{madhyamā}
	\rdg[wit={J10}]{madhyamāṃ}
	\rdg[wit={V1},alt={\illeg}]{\skp{\illeg}}}
\app{\lem[wit={ceteri}]{nāḍī}
	\rdg[wit={V1}]{..\,ḍīṃ}}}
\pada{dṛḍhābhyāsena % driḍhā° J10
\app{\lem[wit={ceteri}]{yoginām}
	\rdg[wit={C6}]{yoginā}
	\rdg[wit={V3}]{yoginaṃ}
	\rdg[wit={J10}]{yoginaḥ}}/}\\+
\pada{\app{\lem[wit={C6,V3,V19,K3,J10,Jyo}]{āsana} % asana N3
	\rdg[wit={Gr2,C7,N19,V15,V1}]{āsanaṃ}}prāṇa%
\app{\lem[wit={N3,N23,N19,V15,V1,Jyo}]{saṃyāma}
	\rdg[wit={V3}]{saṃyama}
	\rdg[wit={C6,K3,C7}]{saṃyāmair}
	\rdg[wit={J7,V19}]{saṃyāmai}
	\rdg[wit={J10}]{saṃyamair}}}%
\pada{mudrābhiḥ % °bhi V3, °niḥ C6
\app{\lem[wit={ceteri}]{saralā}
	\rdg[wit={V19}]{na calā}
	\rdg[wit={V15}]{sabalā}
	\rdg[wit={N19}]{śavalā}} bhavet//}\label{III127}\\!

\newpage
%[hp03_128]
\pada{\app{\lem[wit={N3}]{upāsane}
	\rdg[wit={Gr2}]{upāsanaṃ}
	\rdg[wit={V19,C7}]{upāsana}
	\rdg[wit={K3}]{tathāsana}
	\rdg[wit={V1}]{abhyāse\,..}
	\rdg[wit={C6,V3,V15}]{abhyāseṣu}% P11
	\rdg[wit={J10}]{abhyāsena}
	\rdg[wit={Jyo}]{abhyāse tu}}
\app{\lem[wit={ceteri}]{vinidrāṇāṃ} % °ṇā N3
	\rdg[wit={J10}]{hi mudrāṇāṃ}}}
\pada{\app{\lem[wit={Gr2,Gr3a}]{rājayogaḥ}
	\rdg[wit={N3}]{rājayoga}% +G4
	\rdg[wit={V1}]{anuddhṛta}
	\rdg[wit={V15}]{anuddhata}
	\rdg[wit={C6}]{anudbhūta}
	\rdg[wit={V3}]{manudṛta} % anudruta P11
	\rdg[wit={Jyo}]{mano dhṛtvā}
	\rdg[wit={J10}]{tad udeti}}
\app{\lem[wit={J7}]{samudrakaḥ}
	\rdg[wit={N3}]{samudravat}% +G4, samudbhavān J5
	\rdg[wit={N23}]{samūcakaḥ}
	\rdg[wit={V19}]{samāhnakaḥ}
	\rdg[wit={C7}]{samahnakaḥ}
	\rdg[wit={K3}]{samāhakaḥ}
	\rdg[wit={C6,V15,V1}]{samādhināṃ}
	\rdg[wit={J10,Jyo}]{samādhinā}
	\rdg[wit={V3}]{samādhiṣu}}/}\marma% P11
	\myfn{%
	\getsiglum{N3}: \devnote{upāsane vinidrāṇāṃ rājayogasamudravat}\\
	\getsiglum{Gr2,Gr3a}: \devnote{upāsanavinidrāṇāṃ rājayogaḥ samudrakaḥ} (\devnote{samāhnakaḥ} \getsiglum{Gr3a})\\
	\getsiglum{N19,V15}: \devnote{abhyāseṣu vinidrāṇāṃ manodhṛta-}(or: \devnote{anuddhṛta-}?)\devnote{samādhināṃ}}\\+
\pada{\app{\lem[wit={N3,C6,V3,Gr2,Gr3a,V15,Jyo}]{rudrāṇī}
	\rdg[wit={V1,J10}]{mudrāṇāṃ}}
\app{\lem[wit={Gr2,Gr3a}]{sā}
	\rdg[wit={N3,C6,V3,V15,V1,J10}]{cā}
	\rdg[wit={Jyo}]{vā}} parā mudrā}
\pada{\app{\lem[wit={ceteri}]{bhadrāṃ} % bhadrā V3, mudrā(ṃ) C6,P11
	\rdg[wit={N23}]{bhavāṃ}
	\rdg[wit={N3}]{sadā}} siddhiṃ % siddhi V19,N3,J10
\app{\lem[wit={ceteri}]{prayacchati}
	\rdg[wit={V19}]{prayakṣati}}//}\label{III128} \NotIn{N19}\\!


\startgray
%[hp03_129]
\pada{\app{\lem[wit={ceteri}]{upadeśaṃ}% °dehaṃ C6
	\rdg[wit={V1}]{upadeśe}
	\rdg[wit={N19}]{upadeśo}} hi mudrāṇāṃ}
\pada{yo \app{\lem[wit={C6,V3,J7,Gr3a,N19,J10}]{dhatte}% dadyāt J7pc
	\rdg[wit={V15,Jyo}]{datte}
	\rdg[wit={N23}]{dartte}
	\rdg[wit={V1}]{..\,[tte]}}
\app{\lem[wit={P11,V3,Gr3a,V1,J10,Jyo}]{sāṃpradāyikam}
	\rdg[wit={V15},alt={°yikāṃ}]{sāṃpradāyikāṃ}
	\rdg[wit={Gr2},alt={°yikaḥ}]{sāṃpradāyikaḥ}
	\rdg[wit={N19},alt={°yakaṃ}]{sāṃpradāyakaṃ}
	\rdg[wit={C6}]{sāṃpradāyakaḥ}}/}\\+
\pada{sa \app{\lem[wit={V15,J10,Jyo}]{eva śrī}
	\rdg[wit={P11,J7,Gr3a,N19,V1}]{evāstu}% ##
	\rdg[wit={V3}]{evastu}
	\rdg[wit={N23}]{evavāca}
	\rdg[wit={C6}]{vāstava}}guruḥ svāmī} % guru N23,V1,J10, <gu>ruḥ V19
\pada{sākṣādīśvara % sākhyād N23, sakṣād V19; eṣa N19
\app{\lem[wit={ceteri}]{eva}
	\rdg[wit={N19}]{eṣa}}
\app{\lem[wit={ceteri}]{saḥ} % sa V3
	\rdg[wit={N23}]{ca}}//} \NotIn{N3}\\!

%[hp03_130]
\pada{tasya vākyaparo % °parā N23
\app{\lem[wit={Gr2,Gr3a,V15,Jyo}]{bhūtvā}
	\rdg[wit={C6,V3,N19,J10}]{nityaṃ}}} % nitya V3; mudrāṃ M1,M3,G7
\pada{\app{\lem[wit={Gr3a,V15}]{yo'bhyasyati}% +P11,M1,M3
	\rdg[wit={C6}]{yo bhyasati}
	\rdg[wit={N23}]{yo bhyaset su°}
	\rdg[wit={J7}]{yo bhyaseta}
	\rdg[wit={N19}]{yo bhyasena}% yobhyāsena G7
	\rdg[wit={V3}]{yomabhyaset}
	\rdg[wit={J10}]{athābhyāsa}
	\rdg[wit={Jyo}]{mudrābhyāse}} samāhitaḥ/}\\+ % °hita V3
\pada{aṇimādi\app{\lem[wit={ceteri}]{guṇaiśvaryaṃ}
	\rdg[wit={V15,Jyo}]{guṇaiḥ sārdhaṃ}}} % guṇai V15
\pada{\app{\lem[wit={ceteri}]{jāyate}% jayate G7
	\rdg[wit={J10,Jyo}]{labhate}}
kāla\app{\lem[wit={Gr3a,Jyo}]{vañcanam}% M1,M3
	\rdg[wit={Gr2}]{vañcanāt}
	\rdg[wit={C6,V3,N19,V15,J10}]{vañcakaḥ}}//} % ## G7
	\NotIn{N3,V1}\\!
\endgray

%\newpage
\outdent
%\begin{col}[hp03_col]
\app{\lem[wit={N23,V1}]{iti svātmārāma}
	\rdg[wit={V3}]{iti śrīsvātmārāma}
	\rdg[wit={N3}]{ti śrīsadgurusvātmārāma}
	\rdg[wit={J10}]{ity ātmārāma}
	\rdg[wit={J7,N19,V15},post={(ciṃtāmaṇinā \getsiglum{V15})}]{iti śrīsahajānaṃdasaṃtānaciṃtāmaṇisvātmārāma}
	\rdg[wit={C6,Gr3a}]{iti}}%
\app{\lem[wit={V3,Gr2,J10}]{yogīndra} % yogiṃdra V3
	\rdg[wit={N3}]{yogeṃdra}
	\rdg[wit={N19,V15,V1}]{yoginā}
	\rdg[wit={C6,Gr3a},alt={\om}]{\skp{\om}}}%
\app{\lem[wit={N3,V3,Gr2,N19,V15,V1,J10}]{viracitāyāṃ}
	\rdg[wit={C6,Gr3a},alt={\om}]{\skp{\om}}}
\app{\lem[wit={N3,V3,J7,C7,N19,V15,V1,J10}]{haṭhapradīpikāyāṃ}
	\rdg[wit={C6,K3}]{śrīhaṭhapradīpikāyāṃ}
	\rdg[wit={V19}]{haṭhayogavidyāyāṃ}
	\rdg[wit={N23},alt={\om}]{\skp{\om}}}
\app{\lem[alt={\ante tṛtīyo° \add},nosep]{}
	\rdg[wit={V15}]{mudrāvidhānaṃ}}%
\app{\lem[wit={N3,C6,V3,J7,N19,V15}]{tṛtīyopadeśaḥ} % tṛtiyo V3
	\rdg[wit={V19}]{tṛtīya upadeśaḥ}
	\rdg[wit={K3,C7}]{tṛtīyoyam upadeśaḥ}
	\rdg[wit={V1,J10}]{tṛtīyo dhyāyaḥ}
	\rdg[wit={N23},postwit={(ch. 3 ended with Vajrolī)}]{caturthopadeśa}}// 3 //
%\end{col}

\end{ekdverse}
\end{ekdosis}
%\end{otherlanguage}
\newpage

\bigskip
\bigskip
%\section*{List of sigla}

% N23,J7, V19,K3,C7, P15(up to 13a),N19,V15, V1,N3,V3,J10,Jyo
% J6,N9,V17 (for the Khecaryabhyāsakrama only)

\begin{tabular}{lllp{8cm}}
\multicolumn{4}{l}{\textbf{List of Sigla}} \\
\\
\getsiglum{N3} & N3 & Gr1 & one folio is missing (\ref{VuIII88}--\ref{VuIII116}a)
[\getsiglum{Gr1r} is a reconstruction of Gr1 from the other mss of the group for the missing part of \getsiglum{N3}]\\
\getsiglum{C6} & C6 & Gr4b & contaminated with Gr3?\\
\getsiglum{P11} & P11 & Gr4b & consulted only when the reading of \getsiglum{C6} is unusual from the stemmatic point of view\\
\getsiglum{V3} & V3 & Gr6b\\
\getsiglum{N23} & N23 & Gr2\\
\getsiglum{J7} & J7 & Gr2\\
\getsiglum{V19} & V19 & Gr3\\
\getsiglum{K3} & K3 & Gr3 & the Vajrolī section is lost\\
\getsiglum{C7} & C7 & Gr3 & one folio is missing (3.11d--3.19c)\\
\getsiglum{J6} & J6 & Gr6a & collated only for 3.32*1--33*19\\
\getsiglum{P15} & P15 & Gr4c & lost after 3.13a\\
\getsiglum{N19} & N19 & Gr4c\\
\getsiglum{V15} & V15 & Gr4c & 3.49c--3.67 omitted; contaminated with Gr3?\\
\getsiglum{J14} & J14 & Gr4c & collated for 3.50--66 only as substitute for \getsiglum{V15}\\
\getsiglum{V1} & V1 & Gr4c/4d & \\
\getsiglum{J10} & J10 & Gr4d\\
\getsiglum{N9} & N9 & Gr6c & collated only for 3.32*1--33*19\\
\getsiglum{V17} & V17 & Gr6c & as above\\
\getsiglum{Jyo} & Jyo & Gr4a &  Brahmānanda's version, based on the edition 1972 \\
\end{tabular}

\end{document}


